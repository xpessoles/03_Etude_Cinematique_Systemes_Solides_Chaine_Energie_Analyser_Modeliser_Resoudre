\documentclass[11pt,oneside]{article}
\usepackage[T1]{fontenc}
\usepackage[utf8]{inputenc}
%\DeclareUnicodeCharacter{00A0}{ }
\usepackage[adobe-utopia]{mathdesign}

\usepackage{amsmath}
\usepackage[francais]{babel}
\usepackage[dvips]{graphicx}
%\usepackage{here}
\usepackage{framed}
\usepackage[normalem]{ulem}
\usepackage{fancyhdr}
\usepackage{titlesec}
\usepackage{vmargin}

\usepackage{amsmath}
\usepackage{ifthen}
\usepackage{multirow}
\usepackage{multicol} % Portions de texte en colonnes

%\usepackage{xltxtra} % Logo XeLaTeX
%\usepackage{pst-solides3d}
\usepackage{color}
%\usepackage{colortbl}
\usepackage{titletoc} % Pour la mise en forme de la table des matières

%\usepackage[crop=off]{auto-pst-pdf}
%\usepackage{bclogo}


%\usepackage{longtable}
%\usepackage{flafter}%floatants après la référence
%\usepackage{pst-solides3d}
%\usepackage{pstricks}
%\usepackage{minitoc}
%\setcounter{minitocdepth}{4}
%\usepackage{draftcopy}% "Brouillon"
%\usepackage{floatflt}
%\usepackage{psfrag}
%\usepackage{listings} % Permet d'insérer du code de programmation
%\usepackage{lmodern}
%\usepackage[adobe-utopia,uppercase=upright,greeklowercase=upright]{mathdesign}
%\usepackage{minionpro}
%\usepackage{pifont}
%\usepackage{amssymb}
%\usepackage[francais]{varioref}

\setmarginsrb{1.5cm}{1cm}{1cm}{1.5cm}{1cm}{1cm}{1cm}{1cm}

\definecolor{gris25}{gray}{0.75}
\definecolor{bleu}{RGB}{18,33,98}
\definecolor{bleuf}{RGB}{42,94,171}
\definecolor{bleuc}{RGB}{231,239,247}
\definecolor{rougef}{RGB}{185,18,27}
\definecolor{rougec}{RGB}{255,230,231}
\definecolor{vertf}{RGB}{103,126,82}
\definecolor{vertc}{RGB}{220,255,191}
\definecolor{violetf}{RGB}{112,48,160}
\definecolor{violetc}{RGB}{230,224,236}
\definecolor{jaunec}{RGB}{220,255,191}

\usepackage[%
    pdftitle={Exercice},
    pdfauthor={Xavier Pessoles},
    colorlinks=true,
    linkcolor=blue,
    citecolor=magenta]{hyperref}




% \makeatletter \let\ps@plain\ps@empty \makeatother
%% DEBUT DU DOCUMENT
%% =================
\sloppy
\hyphenpenalty 10000

\begin{document}

%------------- En tetes et Pieds de Pages ------------
\pagestyle{fancy}
\renewcommand{\headrulewidth}{0pt}

\fancyhead{}
\fancyhead[L]{%
\begin{minipage}[c]{1.6cm}
\includegraphics[width=1.4cm]{png/logo_jh_ptsi.png}%
\end{minipage}
\rule{2cm}{.5pt}
}

\fancyhead[C]{\rule{12cm}{.5pt}}

\fancyhead[R]{%
\begin{minipage}[c]{3cm}
\begin{flushright}
\footnotesize{\textit{\textsf{Sciences Industrielles\\ pour l'Ingénieur}}}%
\end{flushright}
\end{minipage}
}

\renewcommand{\footrulewidth}{0.2pt}

\fancyfoot[C]{\footnotesize{\bfseries \thepage}}
\fancyfoot[L]{\footnotesize{2011 -- 2012} \\ X. \textsc{Pessoles}}
\fancyfoot[R]{\footnotesize{Exercice de colle 7}}


\begin{center}
 \Large\textsc{Exercice de colle 7}
\end{center}

\section{Cours}

\begin{minipage}[c]{.55\linewidth}
Soit les liaisons suivantes :
\begin{itemize}
\item cylindre--plan;
\item pivot;
\item glissière.
\end{itemize}
\end{minipage}\hfill
\begin{minipage}[c]{.55\linewidth}
Donner :
\begin{itemize}
\item les degrés de liberté;
\item les degrés de liaison; 
\item le schéma 2D;
\item le schéma 3D;
\item le paramétrage.
\end{itemize}
\end{minipage}

\section{Modélisation d'un système mécanique}
Cette pince permet de sectionner automatiquement de petits câbles, sur une chaîne de production. L’énergie nécessaire à la coupe est fournie par de l’air comprimé, délivré par une centrale pneumatique.

L'air comprimé agit sur le piston 6. Le déplacement de la tige de piston 5 entraîne la rotation des mâchoires 7. Les couteaux 16, liés aux mâchoires 7, assurent la coupe du câble. La coupe étant effectuée, le ressort de rappel 17 repousse le piston 6 en position initiale. Le ressort 11 maintient le contact entre les galets 14 (en liaison pivot avec les mâchoires 7) et la tige de piston 5 (extrémité conique). L’ouverture des mâchoires est ainsi assurée, lors du retour en position initiale.

\vspace{.5cm}

\begin{minipage}[c]{.55\linewidth}
Pour le système suivant donner :
\begin{enumerate}
\item les classes d'équivalences cinématiques;
\item le graphe des liaisons;
\item le schéma cinématique minimal et le paramétrage associé;
\item le schéma d'architecture.
\end{enumerate}
\end{minipage}\hfill
\begin{minipage}[c]{.4\linewidth}
%\includegraphics[width=.95\textwidth]{png/fig1}
\end{minipage}\hfill

\vspace{.5cm}


\begin{center}
\includegraphics[width=.9\textwidth]{png/fig2}
\end{center}

\begin{center}
\includegraphics[width=.8\textwidth]{png/fig1}
\end{center}
 \end{document}
