\documentclass[11pt,oneside]{article}
\usepackage[T1]{fontenc}
\usepackage[utf8]{inputenc}
%\DeclareUnicodeCharacter{00A0}{ }
\usepackage[adobe-utopia]{mathdesign}

\usepackage{amsmath}
\usepackage[francais]{babel}
\usepackage[dvips]{graphicx}
%\usepackage{here}
\usepackage{framed}
\usepackage[normalem]{ulem}
\usepackage{fancyhdr}
\usepackage{titlesec}
\usepackage{vmargin}

\usepackage{amsmath}
\usepackage{ifthen}
\usepackage{multirow}
\usepackage{multicol} % Portions de texte en colonnes

%\usepackage{xltxtra} % Logo XeLaTeX
%\usepackage{pst-solides3d}
\usepackage{color}
%\usepackage{colortbl}
\usepackage{titletoc} % Pour la mise en forme de la table des matières

%\usepackage[crop=off]{auto-pst-pdf}
%\usepackage{bclogo}


%\usepackage{longtable}
%\usepackage{flafter}%floatants après la référence
%\usepackage{pst-solides3d}
%\usepackage{pstricks}
%\usepackage{minitoc}
%\setcounter{minitocdepth}{4}
%\usepackage{draftcopy}% "Brouillon"
%\usepackage{floatflt}
%\usepackage{psfrag}
%\usepackage{listings} % Permet d'insérer du code de programmation
%\usepackage{lmodern}
%\usepackage[adobe-utopia,uppercase=upright,greeklowercase=upright]{mathdesign}
%\usepackage{minionpro}
%\usepackage{pifont}
%\usepackage{amssymb}
%\usepackage[francais]{varioref}

\setmarginsrb{1.5cm}{1cm}{1cm}{1.5cm}{1cm}{1cm}{1cm}{1cm}

\definecolor{gris25}{gray}{0.75}
\definecolor{bleu}{RGB}{18,33,98}
\definecolor{bleuf}{RGB}{42,94,171}
\definecolor{bleuc}{RGB}{231,239,247}
\definecolor{rougef}{RGB}{185,18,27}
\definecolor{rougec}{RGB}{255,230,231}
\definecolor{vertf}{RGB}{103,126,82}
\definecolor{vertc}{RGB}{220,255,191}
\definecolor{violetf}{RGB}{112,48,160}
\definecolor{violetc}{RGB}{230,224,236}
\definecolor{jaunec}{RGB}{220,255,191}

\usepackage[%
    pdftitle={TD Cinématique},
    pdfauthor={Xavier Pessoles},
    colorlinks=true,
    linkcolor=blue,
    citecolor=magenta]{hyperref}



% \makeatletter \let\ps@plain\ps@empty \makeatother
%% DEBUT DU DOCUMENT
%% =================
\sloppy
\hyphenpenalty 10000

\newcommand{\Pointilles}[1][3]{%
\multido{}{#1}{\makebox[\linewidth]{\dotfill}\\[\parskip]
}}


\colorlet{shadecolor}{orange!15}

\newtheorem{theorem}{Theorem}

\newenvironment{theo}
  {\begin{snugshade}\begin{leftbar}\begin{theorem}}
  {\end{theorem}\end{leftbar}\end{snugshade}}


\renewenvironment{leftbar}[1][\hsize]
{%
    \def\FrameCommand
    {%
        {\color{bleuf}\vrule width 3pt}%
        \hspace{0pt}%must no space.
        \fboxsep=\FrameSep\colorbox{bleuc}%
    }%
    \MakeFramed{\hsize#1\advance\hsize-\width\FrameRestore}%
}
{\endMakeFramed}




\begin{document}


\newboolean{prof}
\setboolean{prof}{true}
%------------- En tetes et Pieds de Pages ------------
\pagestyle{fancy}
\renewcommand{\headrulewidth}{0pt}

\fancyhead{}
\fancyhead[L]{%
\begin{minipage}[c]{1.6cm}
\includegraphics[width=1.4cm]{png/logo_jh_ptsi.png}%
\end{minipage}
\rule{2cm}{.5pt}
}

\fancyhead[C]{\rule{12cm}{.5pt}}

\fancyhead[R]{%
\begin{minipage}[c]{3cm}
\begin{flushright}
\footnotesize{\textit{\textsf{Sciences Industrielles\\ pour l'Ingénieur}}}%
\end{flushright}
\end{minipage}
}

\renewcommand{\footrulewidth}{0.2pt}

\fancyfoot[C]{\footnotesize{\bfseries \thepage}}
\fancyfoot[L]{\footnotesize{2011 -- 2012} \\ X. \textsc{Pessoles}}
\ifthenelse{\boolean{prof}}{%
\fancyfoot[R]{\footnotesize{TD -- CI 2 : Cinématique -- P}}
}{%
\fancyfoot[R]{\footnotesize{TD -- CI 2 : Cinématique}}
}


%\begin{center}
%\textit{Centre d'intérêt}
%\end{center}

\begin{center}
 \huge\textsc{CI 2 -- Cinématique : Modélisation, prévision et vérification du comportement cinématiques des systèmes}
\end{center}

\begin{center}
 \LARGE\textsc{Chapitre 1 -- Modélisation des systèmes mécaniques} 
\end{center}

\begin{center}
 \large\textsc{Robot manipulateur} 
\end{center}

Le plan d'ensemble proposé représente l'avant projet d'une pince de manutention d'un robot manipulateur, à une échelle réduite, en vue de face et en vue de gauche.
Le carter 1 est solidaire d’une colonne de guidage dans laquelle se situe un vérin, qui par l’intermédiaire d’une bielle 2 fait pivoter la fourche 3 autour de l’axe 4. La tige 5 est alors entraînée en translation par l’intermédiaire de deux galets 6 et 6’, encastrés dans la fourche 3. Ceci permet le pincement de la tôle 11 entre le mors 7 et le carter 1. (Attention, la nomenclature donnée ci-dessous utilise l’ancienne désignation des matériaux).

\begin{minipage}[c]{.5\linewidth}
\begin{center}
\includegraphics[width=.95\textwidth]{png/img1}
\end{center}
\end{minipage}\hfill
\begin{minipage}[c]{.5\linewidth}
On étudie le mécanisme en phase de serrage de la tôle en tenant compte des remarques suivantes.
On considérera que les liaisons sont parfaites (sans jeu) sauf pour les cas suivants :
\begin{itemize}
\item au vu des jeux et des dimensions des pièces 2 et 3, on admettra que la liaison L2/3 est une linéaire annulaire d’axe $(D, \vect{z})$;
\item il existe également un jeu axial important entre les pièces 5 et 7, qui autorise une légère translation suivant $\vect{z}$;
\item on considérera qu’il y a un jeu entre les pièces \{6 ;6’\} et l’arbre 5, de telle sorte qu’on ne tiendra pas compte des éventuels contacts plans entre ces pièces.
\end{itemize}
\end{minipage}

\paragraph{}
\textit{Dans quel sens faut-il déplacer la tige du vérin 10 pour commander le serrage de la tôle ?}

\paragraph{}
\textit{Déterminer les classes d’équivalence de ce mécanisme (colorier le plan d’ensemble dans les deux vues, il n’est pas nécessaire d’écrire les listes de pièces par groupe).}

\paragraph{}
\textit{Identifier les liaisons et effectuer le graphe des liaisons (respecter les couleurs précédentes).}

\paragraph{}
\textit{Tracer le schéma cinématique plan  du mécanisme : en vue de face, dans la position représentée sur le plan d’ensemble, et à l’échelle de ce plan.}

\paragraph{}
\textit{Tracer le schéma cinématique spatial  en perspective cavalière.}

\begin{center}
\includegraphics[width=\textwidth]{png/plan}
\end{center}

\end{document}