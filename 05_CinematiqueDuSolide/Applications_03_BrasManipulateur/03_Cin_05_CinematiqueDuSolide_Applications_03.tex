\documentclass[10pt,oneside]{article}
\usepackage[T1]{fontenc}
\usepackage[utf8]{inputenc}
%\DeclareUnicodeCharacter{00A0}{ }
\usepackage[adobe-utopia]{mathdesign}

\usepackage{amsmath}
\usepackage[francais]{babel}
\usepackage[dvips]{graphicx}
%\usepackage{here}
\usepackage{framed}
\usepackage[normalem]{ulem}
\usepackage{fancyhdr}
\usepackage{titlesec}
\usepackage{vmargin}

\usepackage{amsmath}
\usepackage{ifthen}
\usepackage{multirow}
\usepackage{multicol} % Portions de texte en colonnes

%\usepackage{xltxtra} % Logo XeLaTeX
%\usepackage{pst-solides3d}
\usepackage{color}
%\usepackage{colortbl}
\usepackage{titletoc} % Pour la mise en forme de la table des matières

%\usepackage[crop=off]{auto-pst-pdf}
%\usepackage{bclogo}


%\usepackage{longtable}
%\usepackage{flafter}%floatants après la référence
%\usepackage{pst-solides3d}
%\usepackage{pstricks}
%\usepackage{minitoc}
%\setcounter{minitocdepth}{4}
%\usepackage{draftcopy}% "Brouillon"
%\usepackage{floatflt}
%\usepackage{psfrag}
%\usepackage{listings} % Permet d'insérer du code de programmation
%\usepackage{lmodern}
%\usepackage[adobe-utopia,uppercase=upright,greeklowercase=upright]{mathdesign}
%\usepackage{minionpro}
%\usepackage{pifont}
%\usepackage{amssymb}
%\usepackage[francais]{varioref}

\setmarginsrb{1.5cm}{1cm}{1cm}{1.5cm}{1cm}{1cm}{1cm}{1cm}

\definecolor{gris25}{gray}{0.75}
\definecolor{bleu}{RGB}{18,33,98}
\definecolor{bleuf}{RGB}{42,94,171}
\definecolor{bleuc}{RGB}{231,239,247}
\definecolor{rougef}{RGB}{185,18,27}
\definecolor{rougec}{RGB}{255,230,231}
\definecolor{vertf}{RGB}{103,126,82}
\definecolor{vertc}{RGB}{220,255,191}
\definecolor{violetf}{RGB}{112,48,160}
\definecolor{violetc}{RGB}{230,224,236}
\definecolor{jaunec}{RGB}{220,255,191}


%Si le boolen xp est vrai : compilation pour xabi
%Sinon compilation Damien
\newboolean{xp}
\setboolean{xp}{true}

\newboolean{prof}
\setboolean{prof}{false}

\def\xxtitre{\ifthenelse{\boolean{xp}}{
CI 3 -- CIN : Étude du comportement cinématique des systèmes}{
}}

\def\xxsoustitre{\ifthenelse{\boolean{xp}}{
Chapitre 5 -- Cinématique du solide indéformable}{
}}


\def\xxauteur{\ifthenelse{\boolean{xp}}{
\noindent 2013 -- 2014 \\
Xavier \textsc{Pessoles}}{
}}


\def\xxpied{\ifthenelse{\boolean{xp}}{
CI 3 : CIN -- Cours \\
Ch 5 : Cinématique du solide -- \ifthenelse{\boolean{prof}}{P}{E}%
}{
}}

\usepackage[%
    pdftitle={CIN : Cinématique du solide},
    pdfauthor={Xavier Pessoles},
    colorlinks=true,
    linkcolor=blue,
    citecolor=magenta]{hyperref}


\usepackage{pifont}
\sloppy
\hyphenpenalty 10000


\begin{document}






% \makeatletter \let\ps@plain\ps@empty \makeatother
%% DEBUT DU DOCUMENT
%% =================




%------------- En tetes et Pieds de Pages ------------


\pagestyle{fancy}
\ifthenelse{\boolean{xp}}{%
\renewcommand{\headrulewidth}{0pt}}{%
\renewcommand{\headrulewidth}{0.2pt}} %pour mettre le trait en haut
%\renewcommand{\headrulewidth}{0.2pt}

\fancyhead{}
\fancyhead[L]{%
\ifthenelse{\boolean{xp}}{%
\noindent\begin{minipage}[c]{2.6cm}%
\includegraphics[width=2cm]{png/logo_ptsi.png}%
\end{minipage}%
}{%
\footnotesize{\textit{\textsf{Lycée François Premier}}}
}}

\ifthenelse{\boolean{xp}}{%
\fancyhead[C]{\rule{12cm}{.5pt}}}{
}


\fancyhead[R]{%
\noindent\begin{minipage}[c]{3cm}
\begin{flushright}
\footnotesize{\textit{\textsf{Sciences Industrielles \\ de l'ingénieur}}}%
\end{flushright}
\end{minipage}
}


\ifthenelse{\boolean{xp}}{%
\fancyhead[C]{\rule{12cm}{.5pt}}}{
}

\renewcommand{\footrulewidth}{0.2pt}

\fancyfoot[C]{\footnotesize{\bfseries \thepage}}
\fancyfoot[L]{%
\begin{minipage}[c]{.2\linewidth}
\noindent\footnotesize{{\xxauteur}}
\end{minipage}
\ifthenelse{\boolean{xp}}{}{%
\begin{minipage}[c]{.15\linewidth}
\includegraphics[width=2cm]{png/logoCC.png}
\end{minipage}}
}


\fancyfoot[R]{\footnotesize{\xxpied}}



\begin{center}
 \Large\textsc{\xxtitre}

\end{center}

\begin{center}
 \large\textsc{\xxsoustitre}
\end{center}

\begin{center}
 \Large\textsc{Exercice d'application}
\end{center}
\begin{flushright}
D'après ressources de Stéphane Génouël\footnote{\url{http://stephane.genouel.free.fr/}} et Renan Bonnard.
\end{flushright}

\vspace{.25cm}

\subsection*{Etude d'une bras manipulateur}

\vspace{.25cm}


\begin{minipage}[c]{.45\linewidth}
La figure ci-dessous représente un bras manipulateur permettant de déplacer des objets.
Ce mécanisme est constitué :
\begin{itemize}
\item d'un bâti S0;
\item d'un solide S1 entraîné en rotation par un moteur M1;
\item d'un solide S2 entraîné en rotation par un moteur M2;
\item d'un solide S3 entraîné en translation par un vérin V1;
\item d'une pince située à l’extrémité du vérin permettant de saisir l’objet.
\end{itemize}

On note : 
\begin{itemize}
\item $1/0$ la rotation d'axe $(A,\vect{z_0})$.
\item $2/1$ la rotation d'axe $(B,\vect{x_1})$.
\item $3/2$ la translation rectiligne de direction $\vect{z_2}$.
\item $\vect{AB}=a\vect{y_1}$.
\end{itemize}

\end{minipage}\hfill%
\begin{minipage}[c]{.5\linewidth}

\begin{center}
\includegraphics[width=.7\textwidth]{png/bras1}
\end{center}

\begin{center}
\includegraphics[width=\textwidth]{png/parametrage}
\end{center}

$$\vect{BC}=\lambda\vect{z_2}$$

\end{minipage}

\subparagraph{}
Proposer un schéma cinématique du bras manipulateur.
%\ifthenelse{\boolean{prof}}{
%\begin{corrige}
%\end{corrige}}{}
%
%
%\subparagraph{}
%Donner le paramétrage du système.
%\ifthenelse{\boolean{prof}}{
%\begin{corrige}
%\end{corrige}}{}


\subparagraph{}
Quelle est la trajectoire du point $B$ appartenant au solide 2 par rapport à 0.
\ifthenelse{\boolean{prof}}{
\begin{corrige}
\end{corrige}}{}

\subparagraph{}
Calculer $\vectv{C}{3}{2}$, $\vectv{C}{2}{1}$, $\vectv{C}{1}{0}$, $\vectv{B}{2}{0}$, $\vectv{C}{3}{0}$.
\ifthenelse{\boolean{prof}}{
\begin{corrige}
\end{corrige}}{}

\subparagraph{}
Calculer $\vect{\Gamma(B,2/0)}$ et $\vect{\Gamma(C,2/0)}$.
\ifthenelse{\boolean{prof}}{
\begin{corrige}
\end{corrige}}{}








\end{document}

