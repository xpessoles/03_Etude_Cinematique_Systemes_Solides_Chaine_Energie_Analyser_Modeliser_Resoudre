\documentclass[10pt,fleqn]{article} % Default font size and left-justified equations
\usepackage[%
    pdftitle={Cycle 6 : Vérification des performances mécaniques},
    pdfauthor={Xavier Pessoles}]{hyperref}
    
%%%%%%%%%%%%%%%%%%%%%%%%%%%%%%%%%%%%%%%%%
% Original author:
% Mathias Legrand (legrand.mathias@gmail.com) with modifications by:
% Vel (vel@latextemplates.com)
% License:
% CC BY-NC-SA 3.0 (http://creativecommons.org/licenses/by-nc-sa/3.0/)
%%%%%%%%%%%%%%%%%%%%%%%%%%%%%%%%%%%%%%%%%

%----------------------------------------------------------------------------------------
%	VARIOUS REQUIRED PACKAGES AND CONFIGURATIONS
%----------------------------------------------------------------------------------------

\usepackage[top=2.5cm,bottom=2cm,left=2cm,right=2cm,headsep=40pt,a4paper]{geometry} % Page margins

\usepackage{graphicx} % Required for including pictures
\graphicspath{{images/}} % Specifies the directory where pictures are stored

\usepackage{lipsum} % Inserts dummy text

\usepackage{tikz} % Required for drawing custom shapes

\usepackage[french]{babel} % English language/hyphenation
\frenchbsetup{StandardLists=true} % Pour éviter la collision babel enumitem pour les listes

\usepackage{enumitem} % Customize lists
\setlist{nolistsep} % Reduce spacing between bullet points and numbered lists

\usepackage{booktabs} % Required for nicer horizontal rules in tables

\usepackage{xcolor} % Required for specifying colors by name
%\definecolor{ocre}{RGB}{243,102,25} % Define the orange color used for highlighting throughout the book
 \definecolor{ocre}{RGB}{49,133,156} % Couleur ''bleue''
\definecolor{violetf}{RGB}{112,48,160} % Couleur ''violet''
\usepackage{enumitem}
\usepackage{pifont} % Pour les dinglist
\usepackage{multicol}
\usepackage{array} % Centrage vertical dans les tableaux

%----------------------------------------------------------------------------------------
%	FONTS
%----------------------------------------------------------------------------------------

\usepackage{avant} % Use the Avantgarde font for headings
%\usepackage{times} % Use the Times font for headings
%\usepackage{mathptmx} % Use the Adobe Times Roman as the default text font together with math symbols from the Sym­bol, Chancery and Com­puter Modern fonts
\usepackage[adobe-utopia]{mathdesign}
\usepackage{microtype} % Slightly tweak font spacing for aesthetics
\usepackage[utf8]{inputenc} % Required for including letters with accents
\usepackage[T1]{fontenc} % Use 8-bit encoding that has 256 glyphs

%----------------------------------------------------------------------------------------
%	BIBLIOGRAPHY AND INDEX
%----------------------------------------------------------------------------------------

\usepackage[style=alphabetic,citestyle=numeric,sorting=nyt,sortcites=true,autopunct=true,babel=hyphen,hyperref=true,abbreviate=false,backref=true,backend=biber]{biblatex}
\addbibresource{bibliography.bib} % BibTeX bibliography file
\defbibheading{bibempty}{}

\usepackage{calc} % For simpler calculation - used for spacing the index letter headings correctly
\usepackage{makeidx} % Required to make an index
\makeindex % Tells LaTeX to create the files required for indexing

%----------------------------------------------------------------------------------------
%	MAIN TABLE OF CONTENTS
%----------------------------------------------------------------------------------------

\usepackage{titletoc} % Required for manipulating the table of contents

\setcounter{tocdepth}{2}     % Dans la table des matieres
\setcounter{secnumdepth}{2}

\contentsmargin{0cm} % Removes the default margin

% Part text styling
\titlecontents{part}[0cm]
{\addvspace{20pt}\centering\large\bfseries}
{}
{}
{}

% Chapter text styling
\titlecontents{chapter}[1.25cm] % Indentation
{\addvspace{12pt}\large\sffamily\bfseries} % Spacing and font options for chapters
{\color{ocre!60}\contentslabel[\Large\thecontentslabel]{1.25cm}\color{ocre}} % Chapter number
{\color{ocre}}  
{\color{ocre!60}\normalsize\;\titlerule*[.5pc]{.}\;\thecontentspage} % Page number

% Section text styling
\titlecontents{section}[1.25cm] % Indentation
{\addvspace{3pt}\sffamily\bfseries} % Spacing and font options for sections
{\color{ocre!60}\contentslabel[\thecontentslabel]{1.25cm} \color{ocre}} % Section number
{\color{ocre}}
{\hfill\color{ocre!60}\thecontentspage} % Page number
[]

% Subsection text styling
\titlecontents{subsection}[1.25cm] % Indentation
{\addvspace{1pt}\sffamily\small} % Spacing and font options for subsections
{\contentslabel[\thecontentslabel]{1.25cm}} % Subsection number
{}
{\ \titlerule*[.5pc]{.}\;\thecontentspage} % Page number
[]


% Subsection text styling
\titlecontents{subsubsection}[1.25cm] % Indentation
{\addvspace{1pt}\sffamily\small} % Spacing and font options for subsections
{\contentslabel[\thecontentslabel]{1.25cm}} % Subsection number
{}
{\ \titlerule*[.5pc]{.}\;\thecontentspage} % Page number
[]

% List of figures
\titlecontents{figure}[0em]
{\addvspace{-5pt}\sffamily}
{\thecontentslabel\hspace*{1em}}
{}
{\ \titlerule*[.5pc]{.}\;\thecontentspage}
[]

% List of tables
\titlecontents{table}[0em]
{\addvspace{-5pt}\sffamily}
{\thecontentslabel\hspace*{1em}}
{}
{\ \titlerule*[.5pc]{.}\;\thecontentspage}
[]

%----------------------------------------------------------------------------------------
%	MINI TABLE OF CONTENTS IN PART HEADS
%----------------------------------------------------------------------------------------

% Chapter text styling
\titlecontents{lchapter}[0em] % Indenting
{\addvspace{15pt}\large\sffamily\bfseries} % Spacing and font options for chapters
{\color{ocre}\contentslabel[\Large\thecontentslabel]{1.25cm}\color{ocre}} % Chapter number
{}  
{\color{ocre}\normalsize\sffamily\bfseries\;\titlerule*[.5pc]{.}\;\thecontentspage} % Page number

% Section text styling
\titlecontents{lsection}[0em] % Indenting
{\sffamily\small} % Spacing and font options for sections
{\contentslabel[\thecontentslabel]{1.25cm}} % Section number
{}
{}

% Subsection text styling
\titlecontents{lsubsection}[.5em] % Indentation
{\normalfont\footnotesize\sffamily} % Font settings
{}
{}
{}

%----------------------------------------------------------------------------------------
%	PAGE HEADERS
%----------------------------------------------------------------------------------------

\usepackage{fancyhdr} % Required for header and footer configuration



\pagestyle{fancy}
 \renewcommand{\headrulewidth}{0pt}
 \fancyhead{}
 \fancyhead[L]{%
 \noindent\begin{minipage}[c]{2.6cm}%
 \includegraphics[width=2cm]{png/logo_lycee.png}%
 \end{minipage}}

\fancyhead[C]{\rule{8cm}{.5pt}}

 \fancyhead[R]{%
 \noindent\begin{minipage}[c]{3cm}
 \begin{flushright}
 \footnotesize{\textit{\textsf{\xxtete}}}%
 \end{flushright}
 \end{minipage}
}


\fancyfoot[C]{\rule{12cm}{.5pt}}
\renewcommand{\footrulewidth}{0.2pt}
\fancyfoot[C]{\footnotesize{\bfseries \thepage}}
\fancyfoot[L]{ 
\begin{minipage}[c]{.4\linewidth}
\noindent\footnotesize{{\xxauteur}}
\end{minipage}}


\fancyfoot[R]{\footnotesize{\xxpied}
\ifthenelse{\isodd{\value{page}}}{
\begin{tikzpicture}[overlay]
\node[shape=rectangle, 
      rounded corners = .25 cm,
	  draw= ocre,
	  line width=2pt, 
	  fill = ocre!10,
	  minimum width  = 2.5cm,
	  minimum height = 3cm,] at (\xxposongletx,\xxposonglety) {};
\node at (\xxposonglettext,\xxposonglety) {\rotatebox{90}{\textbf{\large\color{ocre}{\xxonglet}}}};
%{};
\end{tikzpicture}}{}
}
%
%
%
% Removes the header from odd empty pages at the end of chapters
\makeatletter
\renewcommand{\cleardoublepage}{
\clearpage\ifodd\c@page\else
\hbox{}
\vspace*{\fill}
\thispagestyle{empty}
\newpage
\fi}

\fancypagestyle{plain}{%
\fancyhf{} % vide l’en-tête et le pied~de~page.
%\fancyfoot[C]{\bfseries \thepage} % numéro de la page en cours en gras
% et centré en pied~de~page.
\fancyfoot[R]{\footnotesize{\xxpied}}
\fancyfoot[C]{\rule{12cm}{.5pt}}
\renewcommand{\footrulewidth}{0.2pt}
\fancyfoot[C]{\footnotesize{\bfseries \thepage}}
\fancyfoot[L]{ 
\begin{minipage}[c]{.4\linewidth}
\noindent\footnotesize{{\xxauteur}}
\end{minipage}}}



%----------------------------------------------------------------------------------------
%	THEOREM STYLES
%----------------------------------------------------------------------------------------

% Conflit avec la police adobe
%\usepackage{amsmath,amsfonts,amssymb,amsthm} % For math equations, theorems, symbols, etc
\usepackage{amsmath,amsthm}

\newcommand{\intoo}[2]{\mathopen{]}#1\,;#2\mathclose{[}}
\newcommand{\ud}{\mathop{\mathrm{{}d}}\mathopen{}}
\newcommand{\intff}[2]{\mathopen{[}#1\,;#2\mathclose{]}}
%\newtheorem{notation}{Notation}[chapter]
\newtheorem{notation}{Notation}[section]

% Boxed/framed environments
\newtheoremstyle{ocrenumbox}% % Theorem style name
{0pt}% Space above
{0pt}% Space below
{\normalfont}% % Body font
{}% Indent amount
{\small\bf\sffamily\color{ocre}}% % Theorem head font
{\;}% Punctuation after theorem head
{0.25em}% Space after theorem head
{\small\sffamily\color{ocre}\thmname{#1}\nobreakspace\thmnumber%{\@ifnotempty{#1}{}\@upn{#2}}% Theorem text (e.g. Theorem 2.1)
\thmnote{\nobreakspace\the\thm@notefont\sffamily\bfseries\color{black}---\nobreakspace#3.}} % Optional theorem note
\renewcommand{\qedsymbol}{$\blacksquare$}% Optional qed square


% Boite pour les corriges
\newtheoremstyle{correctionbox}% % Theorem style name
{0pt}% Space above
{0pt}% Space below
{\normalfont}% % Body font
{}% Indent amount
{\small\bf\sffamily\color{violet}}% % Theorem head font
{\;}% Punctuation after theorem head
{0.25em}% Space after theorem head
{\small\sffamily\color{ocre}\thmname{#1}\nobreakspace\thmnumber%{\@ifnotempty{#1}{}\@upn{#2}}% Theorem text (e.g. Theorem 2.1)
\thmnote{\nobreakspace\the\thm@notefont\sffamily\bfseries\color{black}---\nobreakspace#3.}} % Optional theorem note
\renewcommand{\qedsymbol}{$\blacksquare$}% Optional qed square



\newtheoremstyle{blacknumex}% Theorem style name
{5pt}% Space above
{5pt}% Space below
{\normalfont}% Body font
{} % Indent amount
{\small\bf\sffamily}% Theorem head font
{\;}% Punctuation after theorem head
{0.25em}% Space after theorem head
{\small\sffamily{\tiny\ensuremath{\blacksquare}}\nobreakspace\thmname{#1}\nobreakspace\thmnumber%{\@ifnotempty{#1}{}\@upn{#2}}% Theorem text (e.g. Theorem 2.1)
\thmnote{\nobreakspace\the\thm@notefont\sffamily\bfseries---\nobreakspace#3.}}% Optional theorem note

\newtheoremstyle{blacknumbox} % Theorem style name
{0pt}% Space above
{0pt}% Space below
{\normalfont}% Body font
{}% Indent amount
{\small\bf\sffamily}% Theorem head font
{\;}% Punctuation after theorem head
{0.25em}% Space after theorem head
{\small\sffamily\thmname{#1}\nobreakspace 
\thmnote{\nobreakspace\the\thm@notefont\sffamily\bfseries---\nobreakspace#3.}}% Optional theorem note

% Non-boxed/non-framed environments
\newtheoremstyle{ocrenum}% % Theorem style name
{5pt}% Space above
{5pt}% Space below
{\normalfont}% % Body font
{}% Indent amount
{\small\bf\sffamily\color{ocre}}% % Theorem head font
{\;}% Punctuation after theorem head
{0.25em}% Space after theorem head
{\small\sffamily\color{ocre}\thmname{#1}\nobreakspace%\thmnumber{\@ifnotempty{#1}{}\@upn{#2}}% Theorem text (e.g. Theorem 2.1)
\thmnote{\nobreakspace\the\thm@notefont\sffamily\bfseries\color{black}---\nobreakspace#3.}} % Optional theorem note
\renewcommand{\qedsymbol}{$\blacksquare$}% Optional qed square
\makeatother

% Environnement pour les titres de parties
\newtheoremstyle{partiebox} 
{0pt}% Space above
{0pt}% Space below
{\normalfont}% Body font
{}% Indent amount
{\small\bf\sffamily}% Theorem head font
{\;}% Punctuation after theorem head
{0.25em}% Space after theorem head




% Defines the theorem text style for each type of theorem to one of the three styles above
\newcounter{dummy} 
\numberwithin{dummy}{section}
\theoremstyle{ocrenumbox}
%\newtheorem{theoremeT}[dummy]{Théorème}
\newtheorem{theoremeT}[dummy]{Théorème}
\newtheorem{resultatT}[dummy]{Résultat}
\newtheorem{savoirT}[dummy]{Savoir}
\newtheorem{methodeT}[dummy]{Méthode}
\newtheorem{objectifT}[dummy]{Objectif}
%\newtheorem{problem}{Problem}[chapter]
\newtheorem{problem}{Problem}[section]
%\newtheorem{exerciseT}{Exercise}[chapter]
\newtheorem{exerciseT}{Exercice}[section]

\theoremstyle{blacknumex}
%\newtheorem{exampleT}{Example}[chapter]
\newtheorem{exempleT}{Exemple}[section]
\newtheorem{termT}{Terminal\\}[section]
\newtheorem{pyT}{Python\\}[section]
\newtheorem{sciT}{Scilab\\}[section]
\newtheorem{pseudoT}{Pseudo Code\\}[section]
\newtheorem{sqlT}{SQL\\}[section]

\theoremstyle{blacknumbox}
%\newtheorem{vocabulary}{Vocabulary}[chapter]
\newtheorem{vocabulary}{Vocabulaire}[section]
%\newtheorem{definitionT}{Definition}[section]
\newtheorem{definitionT}{Définition}[section]
\newtheorem{rappelT}{Rappel}[section]
\newtheorem{demoT}{Démonstration}[section]
\newtheorem{corollaryT}[dummy]{Corollaire}
\newtheorem{hypoT}{Hypothèse(s)}

\theoremstyle{ocrenum}
\newtheorem{proposition}[dummy]{Proposition}

\theoremstyle{partiebox}
\newtheorem{titrepartieT}[]{}
\newtheorem{titrechapitreT}[]{}

\theoremstyle{correctionbox}
\newtheorem{correctionT}[dummy]{\color{violet}{Correction}}

%----------------------------------------------------------------------------------------
%	DEFINITION OF COLORED BOXES
%----------------------------------------------------------------------------------------

\RequirePackage[framemethod=tikz]{mdframed} % Required for creating the theorem, definition, exercise and corollary boxes

% Theorem box
\newmdenv[skipabove=7pt,
skipbelow=7pt,
backgroundcolor=ocre!10,
linecolor=ocre,
innerleftmargin=5pt,
innerrightmargin=5pt,
innertopmargin=5pt,
leftmargin=0cm,
rightmargin=0cm,
innerbottommargin=5pt]{tBox}


% Correction
\newmdenv[skipabove=7pt,
skipbelow=7pt,
backgroundcolor=violet!10,
linecolor=violet,
innerleftmargin=5pt,
innerrightmargin=5pt,
innertopmargin=5pt,
leftmargin=0cm,
rightmargin=0cm,
innerbottommargin=5pt]{coBox}


% Exercise box	  
\newmdenv[skipabove=7pt,
skipbelow=7pt,
rightline=false,
leftline=true,
topline=false,
bottomline=false,
backgroundcolor=ocre!10,
linecolor=ocre,
innerleftmargin=5pt,
innerrightmargin=5pt,
innertopmargin=5pt,
innerbottommargin=5pt,
leftmargin=0cm,
rightmargin=0cm,
linewidth=4pt]{eBox}	

% Definition box
\newmdenv[skipabove=7pt,
skipbelow=7pt,
rightline=false,
leftline=true,
topline=false,
bottomline=false,
backgroundcolor=ocre!10,
linecolor=ocre,
innerleftmargin=5pt,
innerrightmargin=5pt,
innertopmargin=0pt,
leftmargin=0cm,
rightmargin=0cm,
linewidth=4pt,
innerbottommargin=0pt]{dBox}	

% Demonstration box
\newmdenv[skipabove=7pt,
skipbelow=7pt,
rightline=false,
leftline=true,
topline=false,
bottomline=false,
%backgroundcolor=ocre!10,
linecolor=ocre,
innerleftmargin=5pt,
innerrightmargin=5pt,
innertopmargin=0pt,
leftmargin=0cm,
rightmargin=0cm,
linewidth=4pt,
innerbottommargin=0pt]{demoBox}	

% Corollary box
\newmdenv[skipabove=7pt,
skipbelow=7pt,
rightline=false,
leftline=true,
topline=false,
bottomline=false,
linecolor=gray,
backgroundcolor=black!5,
innerleftmargin=5pt,
innerrightmargin=5pt,
innertopmargin=5pt,
leftmargin=0cm,
rightmargin=0cm,
linewidth=4pt,
innerbottommargin=5pt]{cBox}


% Hypothèses
\newmdenv[skipabove=7pt,
skipbelow=7pt,
rightline=false,
leftline=true,
topline=false,
bottomline=false,
linecolor=gray,
backgroundcolor=black!5,
innerleftmargin=5pt,
innerrightmargin=5pt,
innertopmargin=5pt,
leftmargin=0cm,
rightmargin=0cm,
linewidth=4pt,
innerbottommargin=5pt]{hyBox}


% Boite pour le titre de la partie (pBox)
\newmdenv[skipabove=7pt,
skipbelow=7pt,
rightline=true,
leftline=false,
topline=false,
bottomline=false,
linecolor=ocre,
backgroundcolor=none,
innerleftmargin=5pt,
innerrightmargin=5pt,
innertopmargin=5pt,
leftmargin=0cm,
rightmargin=0cm,
linewidth=4pt,
innerbottommargin=5pt]{pBox}

% Boite pour le titre du chapitre (chBox)
\newmdenv[skipabove=7pt,
skipbelow=7pt,
rightline=false,
leftline=true,
topline=false,
bottomline=false,
linecolor=ocre,
%backgroundcolor=black!5,
innerleftmargin=5pt,
innerrightmargin=5pt,
innertopmargin=5pt,
leftmargin=0cm,
rightmargin=0cm,
linewidth=4pt,
innerbottommargin=5pt]{chBox}


% Boite pour les exemples
\newmdenv[skipabove=7pt,
skipbelow=7pt,
rightline=false,
leftline=true,
topline=false,
bottomline=false,
linecolor=gray,
backgroundcolor=white,
innerleftmargin=5pt,
innerrightmargin=5pt,
innertopmargin=5pt,
leftmargin=0cm,
rightmargin=0cm,
linewidth=4pt,
innerbottommargin=5pt]{exBox}

% Boite pour le terminal
\newmdenv[skipabove=7pt,
skipbelow=7pt,
rightline=false,
leftline=true,
topline=false,
bottomline=false,
linecolor=gray,
backgroundcolor=white,
innerleftmargin=5pt,
innerrightmargin=5pt,
innertopmargin=5pt,
leftmargin=0cm,
rightmargin=0cm,
linewidth=4pt,
innerbottommargin=5pt]{termBox}


% Boite pour Python
\newmdenv[skipabove=7pt,
skipbelow=7pt,
rightline=false,
leftline=true,
topline=false,
bottomline=false,
linecolor=gray,
backgroundcolor=white,
innerleftmargin=5pt,
innerrightmargin=5pt,
innertopmargin=0pt,
leftmargin=0cm,
rightmargin=0cm,
linewidth=4pt,
innerbottommargin=5pt]{pyBox}

% Boite pour scilab
\newmdenv[skipabove=7pt,
skipbelow=7pt,
rightline=false,
leftline=true,
topline=false,
bottomline=false,
linecolor=gray,
backgroundcolor=white,
innerleftmargin=5pt,
innerrightmargin=5pt,
innertopmargin=5pt,
leftmargin=0cm,
rightmargin=0cm,
linewidth=4pt,
innerbottommargin=5pt]{sciBox}


% Boite pour pseudo
\newmdenv[skipabove=7pt,
skipbelow=7pt,
rightline=false,
leftline=true,
topline=false,
bottomline=false,
linecolor=gray,
backgroundcolor=white,
innerleftmargin=5pt,
innerrightmargin=5pt,
innertopmargin=5pt,
leftmargin=0cm,
rightmargin=0cm,
linewidth=4pt,
innerbottommargin=5pt]{pseudoBox}

% Boite pour pseudo
\newmdenv[skipabove=7pt,
skipbelow=7pt,
rightline=false,
leftline=true,
topline=false,
bottomline=false,
linecolor=gray,
backgroundcolor=white,
innerleftmargin=5pt,
innerrightmargin=5pt,
innertopmargin=5pt,
leftmargin=0cm,
rightmargin=0cm,
linewidth=4pt,
innerbottommargin=5pt]{sqlBox}


% Creates an environment for each type of theorem and assigns it a theorem text style from the "Theorem Styles" section above and a colored box from above
\newenvironment{theorem}{\begin{tBox}\begin{theoremeT}}{\end{theoremeT}\end{tBox}}
\newenvironment{resultat}{\begin{tBox}\begin{resultatT}}{\end{resultatT}\end{tBox}}
\newenvironment{methode}{\begin{tBox}\begin{methodeT}}{\end{methodeT}\end{tBox}}
\newenvironment{savoir}{\begin{tBox}\begin{savoirT}}{\end{savoirT}\end{tBox}}
\newenvironment{obj}{\begin{tBox}\begin{objectifT}}{\end{objectifT}\end{tBox}}
\newenvironment{corrige}{\begin{coBox}\begin{correctionT}}{\end{correctionT}\end{coBox}}
\newenvironment{exercise}{\begin{eBox}\begin{exerciseT}}{\hfill{\color{ocre}\tiny\ensuremath{\blacksquare}}\end{exerciseT}\end{eBox}}				  
\newenvironment{exercice}{\begin{eBox}\begin{exerciseT}}{\hfill{\color{ocre}\tiny\ensuremath{\blacksquare}}\end{exerciseT}\end{eBox}}				  

\newenvironment{definition}{\begin{dBox}\begin{definitionT}}{\end{definitionT}\end{dBox}}	
\newenvironment{rappel}{\begin{dBox}\begin{rappelT}}{\end{rappelT}\end{dBox}}	
\newenvironment{defi}{\begin{dBox}\begin{definitionT}}{\end{definitionT}\end{dBox}}	
\newenvironment{demo}{\begin{demoBox}\begin{demoT}}{\end{demoT}\end{demoBox}}	
%\newenvironment{exemple}{\begin{exempleT}}{\hfill{\tiny\ensuremath{\blacksquare}}\end{exempleT}}		
\newenvironment{corollary}{\begin{cBox}\begin{corollaryT}}{\end{corollaryT}\end{cBox}}
\newenvironment{hypo}{\begin{hyBox}\begin{hypoT}}{\end{hypoT}\end{hyBox}}	\newenvironment{exemple}{\begin{exBox}\begin{exempleT}}{\hfill{\tiny\ensuremath{\blacksquare}}\end{exempleT}\end{exBox}}	
\newenvironment{titrepartie}{\begin{pBox}\begin{titrepartieT}}{\end{titrepartieT}\end{pBox}}	
\newenvironment{titrechapitre}{\begin{chBox}\begin{titrechapitreT}}{\end{titrechapitreT}\end{chBox}}	

\newenvironment{term}{ \begin{termBox}\begin{termT}}{\end{termT}\end{termBox}}
\newenvironment{py}{ \begin{pyBox}\begin{pyT}}{\end{pyT}\end{pyBox}}
\newenvironment{sci}{ \begin{sciBox}\begin{sciT}}{\end{sciT}\end{sciBox}}
\newenvironment{pseudo}{ \begin{pseudoBox}\begin{pseudoT}}{\end{pseudoT}\end{pseudoBox}}
\newenvironment{envsql}{ \begin{sqlBox}\begin{sqlT}}{\end{sqlT}\end{sqlBox}}


%----------------------------------------------------------------------------------------
%	REMARK ENVIRONMENT
%----------------------------------------------------------------------------------------

\newenvironment{remark}{\par\vspace{10pt}\small % Vertical white space above the remark and smaller font size
\begin{list}{}{
\leftmargin=35pt % Indentation on the left
\rightmargin=25pt}\item\ignorespaces % Indentation on the right
\makebox[-2.5pt]{\begin{tikzpicture}[overlay]
\node[draw=ocre!60,line width=1pt,circle,fill=ocre!25,font=\sffamily\bfseries,inner sep=2pt,outer sep=0pt] at (-15pt,0pt){\textcolor{ocre}{R}};\end{tikzpicture}} % Orange R in a circle
\advance\baselineskip -1pt}{\end{list}\vskip5pt} % Tighter line spacing and white space after remark

\newenvironment{rem}{\par\vspace{10pt}\small % Vertical white space above the remark and smaller font size
\begin{list}{}{
\leftmargin=35pt % Indentation on the left
\rightmargin=25pt}\item\ignorespaces % Indentation on the right
\makebox[-2.5pt]{\begin{tikzpicture}[overlay]
\node[draw=ocre!60,line width=1pt,circle,fill=ocre!25,font=\sffamily\bfseries,inner sep=2pt,outer sep=0pt] at (-15pt,0pt){\textcolor{ocre}{R}};\end{tikzpicture}} % Orange R in a circle
\advance\baselineskip -1pt}{\end{list}\vskip5pt} % Tighter line spacing and white space after remark


\newenvironment{warn}{\par\vspace{10pt}\small % Vertical white space above the remark and smaller font size
\begin{list}{}{
\leftmargin=35pt % Indentation on the left
\rightmargin=25pt}\item\ignorespaces % Indentation on the right
\makebox[-2.5pt]{\begin{tikzpicture}[overlay]
\node[draw=red!60,line width=1pt,circle,fill=red!25,font=\sffamily\bfseries,inner sep=2pt,outer sep=0pt] at (-15pt,0pt){\textcolor{black}{!}};\end{tikzpicture}} % Point d'exclamation dans un cercle
\advance\baselineskip -1pt}{\end{list}\vskip5pt} % Tighter line spacing and white space after remark


%----------------------------------------------------------------------------------------
%	SECTION NUMBERING IN THE MARGIN
%----------------------------------------------------------------------------------------
\setcounter{secnumdepth}{3}
\setcounter{tocdepth}{2}



\makeatletter
\renewcommand{\@seccntformat}[1]{\llap{\textcolor{ocre}{\csname the#1\endcsname}\hspace{1em}}}                    
\renewcommand{\section}{\@startsection{section}{1}{\z@}
{-4ex \@plus -1ex \@minus -.4ex}
{1ex \@plus.2ex }
{\normalfont\large\sffamily\bfseries}}
\renewcommand{\subsection}{\@startsection {subsection}{2}{\z@}
{-3ex \@plus -0.1ex \@minus -.4ex}
{0.5ex \@plus.2ex }
{\normalfont\sffamily\bfseries}}
\renewcommand{\subsubsection}{\@startsection {subsubsection}{3}{\z@}
{-2ex \@plus -0.1ex \@minus -.2ex}
{.2ex \@plus.2ex }
{\normalfont\small\sffamily\bfseries}}                        
\renewcommand\paragraph{\@startsection{paragraph}{4}{\z@}
{-2ex \@plus-.2ex \@minus .2ex}
{.1ex}
{\normalfont\small\sffamily\bfseries}}

%----------------------------------------------------------------------------------------
%	PART HEADINGS
%----------------------------------------------------------------------------------------


%----------------------------------------------------------------------------------------
%	CHAPTER HEADINGS
%----------------------------------------------------------------------------------------

% \newcommand{\thechapterimage}{}%
% \newcommand{\chapterimage}[1]{\renewcommand{\thechapterimage}{#1}}%
% \def\@makechapterhead#1{%
% {\parindent \z@ \raggedright \normalfont
% \ifnum \c@secnumdepth >\m@ne
% \if@mainmatter
% \begin{tikzpicture}[remember picture,overlay]
% \node at (current page.north west)
% {\begin{tikzpicture}[remember picture,overlay]
% \node[anchor=north west,inner sep=0pt] at (0,0) {\includegraphics[width=\paperwidth]{\thechapterimage}};
% \draw[anchor=west] (\Gm@lmargin,-9cm) node [line width=2pt,rounded corners=15pt,draw=ocre,fill=white,fill opacity=0.5,inner sep=15pt]{\strut\makebox[22cm]{}};
% \draw[anchor=west] (\Gm@lmargin+.3cm,-9cm) node {\huge\sffamily\bfseries\color{black}\thechapter. #1\strut};
% \end{tikzpicture}};
% \end{tikzpicture}
% \else
% \begin{tikzpicture}[remember picture,overlay]
% \node at (current page.north west)
% {\begin{tikzpicture}[remember picture,overlay]
% \node[anchor=north west,inner sep=0pt] at (0,0) {\includegraphics[width=\paperwidth]{\thechapterimage}};
% \draw[anchor=west] (\Gm@lmargin,-9cm) node [line width=2pt,rounded corners=15pt,draw=ocre,fill=white,fill opacity=0.5,inner sep=15pt]{\strut\makebox[22cm]{}};
% \draw[anchor=west] (\Gm@lmargin+.3cm,-9cm) node {\huge\sffamily\bfseries\color{black}#1\strut};
% \end{tikzpicture}};
% \end{tikzpicture}
% \fi\fi\par\vspace*{270\p@}}}

%-------------------------------------------

\def\@makeschapterhead#1{%
\begin{tikzpicture}[remember picture,overlay]
\node at (current page.north west)
{\begin{tikzpicture}[remember picture,overlay]
\node[anchor=north west,inner sep=0pt] at (0,0) {\includegraphics[width=\paperwidth]{\thechapterimage}};
\draw[anchor=west] (\Gm@lmargin,-9cm) node [line width=2pt,rounded corners=15pt,draw=ocre,fill=white,fill opacity=0.5,inner sep=15pt]{\strut\makebox[22cm]{}};
\draw[anchor=west] (\Gm@lmargin+.3cm,-9cm) node {\huge\sffamily\bfseries\color{black}#1\strut};
\end{tikzpicture}};
\end{tikzpicture}
\par\vspace*{270\p@}}
\makeatother

%----------------------------------------------------------------------------------------
%	HYPERLINKS IN THE DOCUMENTS
%----------------------------------------------------------------------------------------


\hypersetup{hidelinks,backref=true,pagebackref=true,hyperindex=true,colorlinks=false,breaklinks=true,urlcolor= ocre,bookmarks=true,bookmarksopen=false,pdftitle={Title},pdfauthor={Author}}
\usepackage{bookmark}
\bookmarksetup{
open,
numbered,
addtohook={%
\ifnum\bookmarkget{level}=0 % chapter
\bookmarksetup{bold}%
\fi
\ifnum\bookmarkget{level}=-1 % part
\bookmarksetup{color=ocre,bold}%
\fi
}
}

%----------------------------------------------------------------------------------------
%	
%----------------------------------------------------------------------------------------

\newcommand{\thechapterimage}{}%
\newcommand{\chapterimage}[1]{\renewcommand{\thechapterimage}{#1}}%
\def\@makechapterhead#1{%
{\parindent \z@ \raggedright \normalfont
\begin{tikzpicture}[remember picture,overlay]
\node at (current page.north west)
{\begin{tikzpicture}[remember picture,overlay]
\node[anchor=north west,inner sep=0pt] at (0,0) {\includegraphics[width=\paperwidth]{\thechapterimage}};
%\draw[anchor=west] (\Gm@lmargin,-9cm) node [line width=2pt,rounded corners=15pt,draw=ocre,fill=white,fill opacity=0.5,inner sep=15pt]{\strut\makebox[22cm]{}};
%\draw[anchor=west] (\Gm@lmargin+.3cm,-9cm) node {\huge\sffamily\bfseries\color{black}\thechapter. #1\strut};
\end{tikzpicture}};
\end{tikzpicture}
\par\vspace*{270\p@}
}}

 \newcounter{exo}


\makeatletter             
\renewcommand{\subparagraph}{\@startsection{exo}{5}{\z@}%
                                    {-2ex \@plus-.2ex \@minus .2ex}%
                                    {0ex}%               
{\normalfont\bfseries Question \hspace{.7cm} }}
\makeatother
\renewcommand{\thesubparagraph}{\arabic{subparagraph}} 
\makeatletter


%%%% Environnement pour inclure du code
\usepackage{textcomp}
\usepackage[french]{algorithm2e}
\usepackage{listings}
\lstloadlanguages{R}   % pour regler les pb d accent utf8 dans les codes
\lstset{language=R} % pour regler les pb d accent utf8 dans les codes
\renewcommand{\lstlistlistingname}{Listings}
\renewcommand{\lstlistingname}{Listing}

\SetKwBlock{Fonction}{Début Fonction}{Fin Fonction}
\SetKwComment{Comment}{start}{end}

\definecolor{Bleu}{rgb}{0.1,0.1,1.0}
\definecolor{Noir}{rgb}{0,0,0}
\definecolor{Grau}{rgb}{0.5,0.5,0.5}
\definecolor{DunkelGrau}{rgb}{0.15,0.15,0.15}
\definecolor{Hellbraun}{rgb}{0.5,0.25,0.0}
\definecolor{Magenta}{rgb}{1.0,0.0,1.0}
\definecolor{Gris}{gray}{0.5}
\definecolor{Vert}{rgb}{0,0.5,0}
\definecolor{SourceHintergrund}{rgb}{1,1.0,0.95}


\lstnewenvironment{python}[1][]{
\lstset{
%escapeinside={\%*}{*)},
inputencoding=utf8,   % pour regler les pb d accent utf8 dans les codes
extendedchars=true,   % pour regler les pb d accent utf8 dans les codes
language=python,
basicstyle=\ttfamily\footnotesize, 	
stringstyle=\color{red}, 
showstringspaces=false, 
alsoletter={1234567890},
otherkeywords={\ , \}, \{},
keywordstyle=\color{blue},
emph={access,and,break,class,continue,def,del,elif ,else,
except,exec,finally,for,from,global,if,import,in,i s,
lambda,not,or,pass,print,raise,return,try,while},
emphstyle=\color{black}\bfseries,
emph={[2]True, False, None, self},
emphstyle=[2]\color{black},
emph={[3]from, import, as},
emphstyle=[3]\color{blue},
upquote=true,
columns=flexible, % pour empecher d'avoir un espacement mono
morecomment=[s]{"""}{"""},
commentstyle=\color{Hellbraun}\slshape, 
%emph={[4]1, 2, 3, 4, 5, 6, 7, 8, 9, 0},
emphstyle=[4]\color{blue},
literate=*{:}{{\textcolor{blue}:}}{1}
{=}{{\textcolor{blue}=}}{1}
{-}{{\textcolor{blue}-}}{1}
{+}{{\textcolor{blue}+}}{1}
{*}{{\textcolor{blue}*}}{1}
{!}{{\textcolor{blue}!}}{1}
{(}{{\textcolor{blue}(}}{1}
{)}{{\textcolor{blue})}}{1}
{[}{{\textcolor{blue}[}}{1}
{]}{{\textcolor{blue}]}}{1}
{<}{{\textcolor{blue}<}}{1}
{>}{{\textcolor{blue}>}}{1}
{COMPLETER}{{\textcolor{red}COMPLETER}}{1},
literate=%
            {é}{{\'{e}}}1
            {è}{{\`{e}}}1
            {ê}{{\^{e}}}1
            {ë}{{\¨{e}}}1
            {û}{{\^{u}}}1
            {ù}{{\`{u}}}1
            {â}{{\^{a}}}1
            {à}{{\`{a}}}1
            {î}{{\^{i}}}1
            {ç}{{\c{c}}}1
            {Ç}{{\c{C}}}1
            {É}{{\'{E}}}1
            {Ê}{{\^{E}}}1
            {À}{{\`{A}}}1
            {Â}{{\^{A}}}1
            {Î}{{\^{I}}}1, % pour regler les pb d accent utf8 dans les codes
%framexleftmargin=1mm, framextopmargin=1mm, frame=shadowbox, rulesepcolor=\color{blue},#1
%backgroundcolor=\color{SourceHintergrund}, 
%framexleftmargin=1mm, framexrightmargin=1mm, framextopmargin=1mm, frame=single, framerule=1pt, rulecolor=\color{black},#1
}}{}



\lstnewenvironment{scilab}[1][]{
\lstset{
language=scilab,
basicstyle=\sffamily\footnotesize, 	
stringstyle=\color{red}, 
showstringspaces=false, 
alsoletter={1234567890},
otherkeywords={\ , \}, \{},
keywordstyle=\color{blue},
emph={access,and,break,class,continue,def,del,elif ,else,
except,exec,finally,for,from,global,if,import,in,i s,
lambda,not,or,pass,print,raise,return,try,while,Debut},
emphstyle=\color{black}\bfseries,
emph={[2]True, False, None, self},
emphstyle=[2]\color{black},
emph={[3]from, import, as},
emphstyle=[3]\color{blue},
upquote=true,
columns=flexible, % pour empecher d'avoir un espacement mono
morecomment=[s]{"""}{"""},
commentstyle=\color{Hellbraun}\slshape, 
%emph={[4]1, 2, 3, 4, 5, 6, 7, 8, 9, 0},
emphstyle=[4]\color{blue},
literate=*{:}{{\textcolor{blue}:}}{1}
{=}{{\textcolor{blue}=}}{1}
{-}{{\textcolor{blue}-}}{1}
{+}{{\textcolor{blue}+}}{1}
{*}{{\textcolor{blue}*}}{1}
{!}{{\textcolor{blue}!}}{1}
{(}{{\textcolor{blue}(}}{1}
{)}{{\textcolor{blue})}}{1}
{[}{{\textcolor{blue}[}}{1}
{]}{{\textcolor{blue}]}}{1}
{<}{{\textcolor{blue}<}}{1}
{>}{{\textcolor{blue}>}}{1},
%framexleftmargin=1mm, framextopmargin=1mm, frame=shadowbox, rulesepcolor=\color{blue},#1
%backgroundcolor=\color{SourceHintergrund}, 
%framexleftmargin=1mm, framexrightmargin=1mm, framextopmargin=1mm, frame=single, framerule=1pt, rulecolor=\color{black},#1
}}{}


\lstdefinestyle{stylepython}{%
escapeinside={\%*}{*)},
inputencoding=utf8,   % pour regler les pb d accent utf8 dans les codes
extendedchars=true,   % pour regler les pb d accent utf8 dans les codes
language=python,
basicstyle=\sffamily\footnotesize, 	
stringstyle=\color{red}, 
showstringspaces=false, 
alsoletter={1234567890},
otherkeywords={\ , \}, \{},
keywordstyle=\color{blue},
emph={access,and,break,class,continue,def,del,elif ,else,
except,exec,finally,for,from,global,if,import,in,i s,
lambda,not,or,pass,print,raise,return,try,while},
emphstyle=\color{black}\bfseries,
emph={[2]True, False, None, self},
emphstyle=[2]\color{green},
emph={[3]from, import, as},
emphstyle=[3]\color{blue},
upquote=true,
columns=flexible, % pour empecher d'avoir un espacement mono
morecomment=[s]{"""}{"""},
commentstyle=\color{Hellbraun}\slshape, 
%emph={[4]1, 2, 3, 4, 5, 6, 7, 8, 9, 0},
emphstyle=[4]\color{blue},
literate=*{:}{{\textcolor{blue}:}}{1}
{=}{{\textcolor{blue}=}}{1}
{-}{{\textcolor{blue}-}}{1}
{+}{{\textcolor{blue}+}}{1}
{*}{{\textcolor{blue}*}}{1}
{!}{{\textcolor{blue}!}}{1}
{(}{{\textcolor{blue}(}}{1}
{)}{{\textcolor{blue})}}{1}
{[}{{\textcolor{blue}[}}{1}
{]}{{\textcolor{blue}]}}{1}
{<}{{\textcolor{blue}<}}{1}
{>}{{\textcolor{blue}>}}{1}
{COMPLETER}{{\textcolor{red}COMPLETER}}{1},
literate=%
            {é}{{\'{e}}}1
            {è}{{\`{e}}}1
            {ê}{{\^{e}}}1
            {ë}{{\¨{e}}}1
            {û}{{\^{u}}}1
            {ù}{{\`{u}}}1
            {â}{{\^{a}}}1
            {à}{{\`{a}}}1
            {î}{{\^{i}}}1
            {ç}{{\c{c}}}1
            {Ç}{{\c{C}}}1
            {É}{{\'{E}}}1
            {Ê}{{\^{E}}}1
            {À}{{\`{A}}}1
            {Â}{{\^{A}}}1
            {Î}{{\^{I}}}1,
%numbers=left,                    % where to put the line-numbers; possible values are (none, left, right)
%numbersep=5pt,                   % how far the line-numbers are from the code
%numberstyle=\tiny\color{mygray}, % the style that is used for the line-numbers
}



\lstnewenvironment{termi}[1][]{
\lstset{
language=scilab,
basicstyle=\sffamily\footnotesize, 	
stringstyle=\color{red}, 
showstringspaces=false, 
alsoletter={1234567890},
otherkeywords={\ , \}, \{},
keywordstyle=\color{blue},
emph={access,and,break,class,continue,def,del,elif ,else,
except,exec,finally,for,from,global,if,import,in,i s,
lambda,not,or,pass,print,raise,return,try,while,Debut},
emphstyle=\color{black}\bfseries,
emph={[2]True, False, None, self},
emphstyle=[2]\color{green},
emph={[3]from, import, as},
emphstyle=[3]\color{blue},
upquote=true,
columns=flexible, % pour empecher d'avoir un espacement mono
morecomment=[s]{"""}{"""},
commentstyle=\color{Hellbraun}\slshape, 
%emph={[4]1, 2, 3, 4, 5, 6, 7, 8, 9, 0},
emphstyle=[4]\color{blue},
literate=*{:}{{\textcolor{blue}:}}{1}
{=}{{\textcolor{blue}=}}{1}
{-}{{\textcolor{blue}-}}{1}
{+}{{\textcolor{blue}+}}{1}
{*}{{\textcolor{blue}*}}{1}
{!}{{\textcolor{blue}!}}{1}
{(}{{\textcolor{blue}(}}{1}
{)}{{\textcolor{blue})}}{1}
{[}{{\textcolor{blue}[}}{1}
{]}{{\textcolor{blue}]}}{1}
{<}{{\textcolor{blue}<}}{1}
{>}{{\textcolor{blue}>}}{1},
%framexleftmargin=1mm, framextopmargin=1mm, frame=shadowbox, rulesepcolor=\color{blue},#1
%backgroundcolor=\color{SourceHintergrund}, 
%framexleftmargin=1mm, framexrightmargin=1mm, framextopmargin=1mm, frame=single, framerule=1pt, rulecolor=\color{black},#1
}}{}


\lstnewenvironment{sql}[1][]{
\lstset{
%escapeinside={\%*}{*)},
%inputencoding=utf8,   % pour regler les pb d accent utf8 dans les codes
%extendedchars=true,   % pour regler les pb d accent utf8 dans les codes
language=sql,
basicstyle=\sffamily\footnotesize, 	
stringstyle=\color{red}, 
showstringspaces=false, 
alsoletter={1234567890},
otherkeywords={\ , \}, \{},
keywordstyle=\color{blue},
emph={access,and,break,class,continue,def,del,elif ,else,
except,exec,finally,for,from,global,if,import,in,i s,
lambda,not,or,pass,print,raise,return,try,while},
emphstyle=\color{black}\bfseries,
emph={[2]True, False, None, self},
emphstyle=[2]\color{black},
emph={[3]from, import, as},
emphstyle=[3]\color{blue},
upquote=true,
columns=flexible, % pour empecher d'avoir un espacement mono
morecomment=[s]{"""}{"""},
commentstyle=\color{Hellbraun}\slshape, 
%emph={[4]1, 2, 3, 4, 5, 6, 7, 8, 9, 0},
emphstyle=[4]\color{blue},
literate=*{:}{{\textcolor{blue}:}}{1}
{=}{{\textcolor{blue}=}}{1}
{-}{{\textcolor{blue}-}}{1}
{+}{{\textcolor{blue}+}}{1}
{*}{{\textcolor{blue}*}}{1}
{!}{{\textcolor{blue}!}}{1}
{(}{{\textcolor{blue}(}}{1}
{)}{{\textcolor{blue})}}{1}
{[}{{\textcolor{blue}[}}{1}
{]}{{\textcolor{blue}]}}{1}
{<}{{\textcolor{blue}<}}{1}
{>}{{\textcolor{blue}>}}{1}
{COMPLETER}{{\textcolor{red}COMPLETER}}{1},
literate=%
            {é}{{\'{e}}}1
            {è}{{\`{e}}}1
            {ê}{{\^{e}}}1
            {ë}{{\¨{e}}}1
            {û}{{\^{u}}}1
            {ù}{{\`{u}}}1
            {â}{{\^{a}}}1
            {à}{{\`{a}}}1
            {î}{{\^{i}}}1
            {ç}{{\c{c}}}1
            {Ç}{{\c{C}}}1
            {É}{{\'{E}}}1
            {Ê}{{\^{E}}}1
            {À}{{\`{A}}}1
            {Â}{{\^{A}}}1
            {Î}{{\^{I}}}1, % pour regler les pb d accent utf8 dans les codes
%framexleftmargin=1mm, framextopmargin=1mm, frame=shadowbox, rulesepcolor=\color{blue},#1
%backgroundcolor=\color{SourceHintergrund}, 
%framexleftmargin=1mm, framexrightmargin=1mm, framextopmargin=1mm, frame=single, framerule=1pt, rulecolor=\color{black},#1
}}{}


% Définition des booleéns
\newif\iffiche
\newif\ifprof
\newif\iftd
\newif\ifcours

%%%%%%%%%%%%
% Définition des vecteurs 
%%%%%%%%%%%%
 \newcommand{\vect}[1]{\overrightarrow{#1}}
\newcommand{\axe}[2]{\left(#1,\vect{#2}\right)}

\newcommand{\rep}[1]{\mathcal{R}_{#1}}
\newcommand{\vx}[1]{\vect{x_{#1}}}
\newcommand{\vy}[1]{\vect{y_{#1}}}
\newcommand{\vz}[1]{\vect{z_{#1}}}

%%%%%%%%%%%%
% Définition des torseurs 
%%%%%%%%%%%%

 \newcommand{\torseur}[1]{%
\left\{{#1}\right\}
}

\newcommand{\torseurcin}[3]{%
\left\{\mathcal{#1} \left(#2/#3 \right) \right\}
}

\newcommand{\torseurstat}[3]{%
\left\{\mathcal{#1} \left(#2\rightarrow #3 \right) \right\}
}

 \newcommand{\torseurc}[8]{%
%\left\{#1 \right\}=
\left\{
{#1}
\right\}
 = 
\left\{%
\begin{array}{cc}%
{#2} & {#5}\\%
{#3} & {#6}\\%
{#4} & {#7}\\%
\end{array}%
\right\}_{#8}%
}

 \newcommand{\torseurcol}[7]{
\left\{%
\begin{array}{cc}%
{#1} & {#4}\\%
{#2} & {#5}\\%
{#3} & {#6}\\%
\end{array}%
\right\}_{#7}%
}

 \newcommand{\torseurl}[3]{%
%\left\{\mathcal{#1}\right\}_{#2}=%
\left\{%
\begin{array}{l}%
{#1} \\%
{#2} %
\end{array}%
\right\}_{#3}%
}

 \newcommand{\vectv}[3]{%
\vect{V\left( {#1} \in {#2}/{#3}\right)}
}


\newcommand{\vectf}[2]{%
\vect{R\left( {#1} \rightarrow {#2}\right)}
}

\newcommand{\vectm}[3]{%
\vect{\mathcal{M}\left( {#1}, {#2} \rightarrow {#3}\right)}
}


 \newcommand{\vectg}[3]{%
\vect{\Gamma \left( {#1} \in {#2}/{#3}\right)}
}

 \newcommand{\vecto}[2]{%
\vect{\Omega\left( {#1}/{#2}\right)}
}
% }$$\left\{\mathcal{#1} \right\}_{#2} =%
% \left\{%
% \begin{array}{c}%
%  #3 \\%
%  #4 %
% \end{array}%
% \right\}_{#5}}

\usepackage{multicol}
\fichetrue
%\fichefalse

%\proftrue
\proffalse

\tdtrue
%\tdfalse

\courstrue
\coursfalse

\def\discipline{Sciences \\Industrielles de \\ l'Ingénieur}
\def\xxtete{Sciences Industrielles de l'Ingénieur}

\def\classe{PTSI}
\def\xxnumpartie{Cycle 6}
\def\xxpartie{Vérification des performances mécaniques  \\
des systèmes mécaniques}

\def\xxnumchapitre{Chapitre 1}
\def\xxchapitre{Détermination des grandeurs cinématiques : vitesse et accélération}

\def\xxtitreexo{Exercices d'application du cours}
\def\xxsourceexo{\hspace{.2cm} D'après ressources de S. Genouël.}


\def\xxposongletx{2}
\def\xxposonglettext{1.45}
\def\xxposonglety{20}
\def\xxonglet{Cy. 6 -- Ch. 1}

\def\xxactivite{Colle 1}
\def\xxauteur{\textsl{S. Genouël}}

\def\xxcompetences{%
\textsl{%
\textbf{Savoirs et compétences :}\\
%\noindent \textbf{Résoudre :} à partir des modèles retenus :
\begin{itemize}[label=\ding{112},font=\color{ocre}] 
%\item %\textit{Rés -- C1.1 :} Loi entrée sortie géométrique et cinématique -- Fermeture géométrique.
\item Écrire le vecteur position, vitesse d’un point d’un solide.
\item Écrire le vecteur accélération d’un point d’un solide.
\end{itemize}
%
%\noindent \textit{Mod2 -- C4.1 :} Représentation par schéma bloc.
}}

\def\xxfigures{
\includegraphics[width=.8\textwidth]{images/prot_01}
}%figues de la page de garde

\def\xxpied{%
Partie 6 -- Étude cinématique des systèmes  \\
Ch. 1 : Détermination des grandeurs cinématiques : vitesse et accélération -- \xxactivite%
}


\setcounter{secnumdepth}{5}
%---------------------------------------------------------------------------


\begin{document}
%\chapterimage{images/Fond_Cin}
\pagestyle{empty}


%%%%%%%% PAGE DE GARDE COURS
\ifcours
\begin{tikzpicture}[remember picture,overlay]
\node at (current page.north west)
{\begin{tikzpicture}[remember picture,overlay]
\node[anchor=north west,inner sep=0pt] at (0,0) {\includegraphics[width=\paperwidth]{\thechapterimage}};
\draw[anchor=west] (-2cm,-8cm) node [line width=2pt,rounded corners=15pt,draw=ocre,fill=white,fill opacity=0.6,inner sep=40pt]{\strut\makebox[22cm]{}};
\draw[anchor=west] (1cm,-8cm) node {\huge\sffamily\bfseries\color{black} %
\begin{minipage}{1cm}
\rotatebox{90}{\LARGE\sffamily\textsc{\color{ocre}\textbf{\xxnumpartie}}}
\end{minipage} \hfill
\begin{minipage}[c]{14cm}
\begin{titrepartie}
\begin{flushright}
\renewcommand{\baselinestretch}{1.1} 
\Large\sffamily\textsc{\textbf{\xxpartie}}
\renewcommand{\baselinestretch}{1} 
\end{flushright}
\end{titrepartie}
\end{minipage} \hfill
\begin{minipage}[c]{3.5cm}
{\large\sffamily\textsc{\textbf{\color{ocre} \discipline}}}
\end{minipage} 
 };
\end{tikzpicture}};
\end{tikzpicture}


\begin{tikzpicture}[overlay]
\node[shape=rectangle, 
      rounded corners = .25 cm,
	  draw= ocre,
	  line width=2pt, 
	  fill = ocre!10,
	  minimum width  = 2.5cm,
	  minimum height = 3cm,] at (18cm,0) {};
\node at (17.7cm,0) {\rotatebox{90}{\textbf{\Large\color{ocre}{\classe}}}};
%{};
\end{tikzpicture}

\vspace{3.5cm}

\begin{tikzpicture}[remember picture,overlay]
\draw[anchor=west] (-2cm,-6cm) node {\huge\sffamily\bfseries\color{black} %
\begin{minipage}{2cm}
\begin{center}
\LARGE\sffamily\textsc{\color{ocre}\textbf{\xxactivite}}
\end{center}
\end{minipage} \hfill
\begin{minipage}[c]{15cm}
\begin{titrechapitre}
\renewcommand{\baselinestretch}{1.1} 
\Large\sffamily\textsc{\textbf{\xxnumchapitre}}

\Large\sffamily\textsc{\textbf{\xxchapitre}}
\vspace{.5cm}

\renewcommand{\baselinestretch}{1} 
\normalsize\normalfont
\xxcompetences
\end{titrechapitre}
\end{minipage}  };
\end{tikzpicture}
\vfill

\begin{flushright}
\begin{minipage}[c]{.3\linewidth}
\begin{center}
\xxfigures
\end{center}
\end{minipage}\hfill
\begin{minipage}[c]{.6\linewidth}
\startcontents
\printcontents{}{1}{}
\end{minipage}
\end{flushright}

\begin{tikzpicture}[remember picture,overlay]
\draw[anchor=west] (4.5cm,-.7cm) node {
\begin{minipage}[c]{.2\linewidth}
\begin{flushright}
\includegraphics[width=2cm]{png/logoCC}
\end{flushright}
\end{minipage}
\begin{minipage}[c]{.2\linewidth}
\textsl{\xxauteur} \\
\textsl{\classe}
\end{minipage}
 };
\end{tikzpicture}
\newpage
\pagestyle{fancy}

\newpage
\pagestyle{fancy}

\else
\fi


%%%%%%%% PAGE DE GARDE TD
\iftd
%\begin{tikzpicture}[remember picture,overlay]
%\node at (current page.north west)
%{\begin{tikzpicture}[remember picture,overlay]
%\draw[anchor=west] (-2cm,-3.25cm) node [line width=2pt,rounded corners=15pt,draw=ocre,fill=white,fill opacity=0.6,inner sep=40pt]{\strut\makebox[22cm]{}};
%\draw[anchor=west] (1cm,-3.25cm) node {\huge\sffamily\bfseries\color{black} %
%\begin{minipage}{1cm}
%\rotatebox{90}{\LARGE\sffamily\textsc{\color{ocre}\textbf{\xxnumpartie}}}
%\end{minipage} \hfill
%\begin{minipage}[c]{13.5cm}
%\begin{titrepartie}
%\begin{flushright}
%\renewcommand{\baselinestretch}{1.1} 
%\Large\sffamily\textsc{\textbf{\xxpartie}}
%\renewcommand{\baselinestretch}{1} 
%\end{flushright}
%\end{titrepartie}
%\end{minipage} \hfill
%\begin{minipage}[c]{3.5cm}
%{\large\sffamily\textsc{\textbf{\color{ocre} \discipline}}}
%\end{minipage} 
% };
%\end{tikzpicture}};
%\end{tikzpicture}

%%%%%%%%%% PAGE DE GARDE TD %%%%%%%%%%%%%%%
%\begin{tikzpicture}[overlay]
%\node[shape=rectangle, 
%      rounded corners = .25 cm,
%	  draw= ocre,
%	  line width=2pt, 
%	  fill = ocre!10,
%	  minimum width  = 2.5cm,
%	  minimum height = 2.5cm,] at (18.5cm,0) {};
%\node at (17.7cm,0) {\rotatebox{90}{\textbf{\Large\color{ocre}{\classe}}}};
%%{};
%\end{tikzpicture}

% PARTIE ET CHAPITRE
%\begin{tikzpicture}[remember picture,overlay]
%\draw[anchor=west] (-1cm,-2.1cm) node {\large\sffamily\bfseries\color{black} %
%\begin{minipage}[c]{15cm}
%\begin{flushleft}
%\xxnumchapitre \\
%\xxchapitre
%\end{flushleft}
%\end{minipage}  };
%\end{tikzpicture}

% Bandeau titre exo
\begin{tikzpicture}[remember picture,overlay]
\draw[anchor=west] (-2cm,-6cm) node {\huge\sffamily\bfseries\color{black} %
\begin{minipage}{5cm}
\begin{center}
\LARGE\sffamily\color{ocre}\textbf{\textsc{\xxactivite}}

\begin{center}
\xxfigures
\end{center}

\end{center}
\end{minipage} \hfill
\begin{minipage}[c]{12cm}
\begin{titrechapitre}
\renewcommand{\baselinestretch}{1.1} 
\large\sffamily\textbf{\textsc{\xxtitreexo}}

\small\sffamily{\textbf{\textit{\color{black!70}\xxsourceexo}}}
\vspace{.5cm}

\renewcommand{\baselinestretch}{1} 
\normalsize\normalfont
\xxcompetences
\end{titrechapitre}
\end{minipage}  };
\end{tikzpicture}

\else
\fi


%%%%%%%% PAGE DE GARDE FICHE
\iffiche
\begin{tikzpicture}[remember picture,overlay]
\node at (current page.north west)
{\begin{tikzpicture}[remember picture,overlay]
\draw[anchor=west] (-2cm,-3.25cm) node [line width=2pt,rounded corners=15pt,draw=ocre,fill=white,fill opacity=0.6,inner sep=40pt]{\strut\makebox[22cm]{}};
\draw[anchor=west] (1cm,-3.25cm) node {\huge\sffamily\bfseries\color{black} %
\begin{minipage}{1cm}
\rotatebox{90}{\LARGE\sffamily\textsc{\color{ocre}\textbf{\xxnumpartie}}}
\end{minipage} \hfill
\begin{minipage}[c]{14cm}
\begin{titrepartie}
\begin{flushright}
\renewcommand{\baselinestretch}{1.1} 
\large\sffamily\textsc{\textbf{\xxpartie} \\} 

\vspace{.2cm}

\normalsize\sffamily\textsc{\textbf{\xxnumchapitre -- \xxchapitre}}
\renewcommand{\baselinestretch}{1} 
\end{flushright}
\end{titrepartie}
\end{minipage} \hfill
\begin{minipage}[c]{3.5cm}
{\large\sffamily\textsc{\textbf{\color{ocre} \discipline}}}
\end{minipage} 
 };
\end{tikzpicture}};
\end{tikzpicture}


\begin{tikzpicture}[overlay]
\node[shape=rectangle, 
      rounded corners = .25 cm,
	  draw= ocre,
	  line width=2pt, 
	  fill = ocre!10,
	  minimum width  = 2.5cm,
	  minimum height = 2.5cm,] at (18.5cm,0.5cm) {};
%	  minimum height = 2.5cm,] at (18.5cm,0cm) {};
\node at (17.7cm,0.5) {\rotatebox{90}{\textsf{\textbf{\large\color{ocre}{\classe}}}}};
%{};
\end{tikzpicture}



\else
\fi



\vspace{8cm}
\pagestyle{fancy}
\thispagestyle{plain}


\def\columnseprulecolor{\color{ocre}}
\setlength{\columnseprule}{0.4pt} 
\begin{multicols}{2}
\section*{Étude d'une benne de camion}


Soit $\mathcal{R}_0$ un repère lié au châssis 0 d'un camion benne. Soit $\mathcal{R}_1$ et $\mathcal{R}_2$  deux repères liés respectivement au corps 1 et à la tige 2 d'un des deux vérins hydrauliques.
On suppose que le vérin étudié (corps+tige) se déplace dans le plan $(\vect{x_0},\vect{y_0})$.
On pose $\alpha=(\vect{x_0};\vect{x_1})$ et $\vect{OB}=\lambda (t)\vect{x_1}$.


\begin{center}
\includegraphics[width=.9\linewidth]{images/benne1}
\end{center}

\begin{center}
\includegraphics[width=.7\linewidth]{images/benne2}
\end{center}


Le paramétrage du système est le suivant :
$$
\vect{OB} = \lambda(t) \vect{x_1}
$$


\begin{center}
\includegraphics[width=.45\linewidth]{images/parametrage}
\end{center}



%\subparagraph{}
%Donner le paramétrage du système.
%\ifprof
%\begin{corrige}
%\end{corrige}\else \fi

\subparagraph{}
Quelle est la trajectoire du point $B$ appartenant au solide 2 par rapport à 0.
\ifprof
\begin{corrige}
\end{corrige}\else \fi

\subparagraph{}
Calculer $\vectv{B}{2}{0}$.
\ifprof
\begin{corrige}
\end{corrige}\else \fi

\subparagraph{}
Calculer $\vect{\Gamma(B,2/0)}$.

\ifprof
\begin{corrige}
\end{corrige}\else \fi

\end{multicols}
\newpage \setcounter{exo}{0}
\begin{multicols}{2}

\section*{Étude d'une bras manipulateur}


La figure ci-dessous représente un bras manipulateur permettant de déplacer des objets.
Ce mécanisme est constitué :
\begin{itemize}
\item d'un bâti S0;
\item d'un solide S1 entraîné en rotation par un moteur M1;
\item d'un solide S2 entraîné en rotation par un moteur M2;
\item d'un solide S3 entraîné en translation par un vérin V1;
\item d'une pince située à l’extrémité du vérin permettant de saisir l’objet.
\end{itemize}

On note : 
\begin{itemize}
\item $1/0$ la rotation d'axe $(A,\vect{z_0})$.
\item $2/1$ la rotation d'axe $(B,\vect{x_1})$.
\item $3/2$ la translation rectiligne de direction $\vect{z_2}$.
\item $\vect{AB}=a\vect{y_1}$.
\end{itemize}


\begin{center}
\includegraphics[width=.7\linewidth]{images/bras1}
\end{center}

\begin{center}
\includegraphics[width=\linewidth]{images/parametrage_02}
\end{center}

$$\vect{BC}=\lambda\vect{z_2}$$




%\ifprof
%\begin{corrige}
%\end{corrige}\else \fi
%
%
%
%Donner le paramétrage du système.
%\ifprof
%\begin{corrige}
%\end{corrige}\else \fi

\subparagraph{}
\textit{Proposer un schéma cinématique du bras manipulateur.}

\subparagraph{}
\textit{Quelle est la trajectoire du point $B$ appartenant au solide 2 par rapport à 0.}
\ifprof
\begin{corrige}
\end{corrige}\else \fi

\subparagraph{}
\textit{Calculer $\vectv{C}{3}{2}$, $\vectv{C}{2}{1}$, $\vectv{C}{1}{0}$, $\vectv{B}{2}{0}$, $\vectv{C}{3}{0}$.}
\ifprof
\begin{corrige}
\end{corrige}\else \fi

\subparagraph{}
\textit{Calculer $\vect{\Gamma(A,2/0)}$, $\vect{\Gamma(B,2/0)}$, $\vect{\Gamma(C,2/0)}$.}
\ifprof
\begin{corrige}
\end{corrige}\else \fi

\end{multicols}
\newpage \setcounter{exo}{0}
\begin{multicols}{2}

\section*{Étude d'une centrifugeuse à 2 degrés de liberté [1]}

\begin{center}
\includegraphics[height=3.5cm]{images/centrifugeuse_1}

\textit{Centrifugeuse humaine développée par le CNRS / MEDES } 
\end{center}



\begin{center}
\includegraphics[width=0.9\linewidth]{images/centrifugeuse_2}

\textit{Modélisation cinématique} 
\end{center}



Le paramétrage de la centrifugeuse est donnée ci dessous : 



\begin{center}
\includegraphics[height=3cm]{images/centrifugeuse_3}
\end{center}


Les paramètres constants du système sont les suivants : 
\begin{itemize}%[$\bullet$]
\item $\vect{O_0O_1} = a \vect{i_1}$;
\item $\vect{O_1G} = b \vect{i_2} + c \vect{k_2}$.
\end{itemize}





\subparagraph{}
\textit{Donner la trajectoire du point $G$ dans le repère $\mathcal{R}_0$.}
\ifprof
\begin{corrige}
La trajectoire du point $G$ dans le repère $\mathcal{R}_0$  est donnée par le vecteur :
$$
\vect{O_0 G}(t)  = \vect{O_0O_1} + \vect{O_1G}
= a \vect{i_1} +  b \vect{i_2} + c \vect{k_2}
$$

Il faut alors projeter les vecteurs dans $\mathcal{R}_0$ : 
\begin{eqnarray*}
\vect{O_0 G}(t) &=& a \left(\cos\alpha(t) \vect{i_0} + \sin\alpha(t) \vect{j_0} \right) 
+ b \left(\cos\beta(t) \vect{i_1} - \sin\beta(t) \vect{k_1} \right) 
+ c\left(\cos\beta(t) \vect{k_1} + \sin\beta(t)\vect{i_1}\right)\\
&=& a \left(\cos\alpha(t) \vect{i_0} + \sin\alpha(t) \vect{j_0} \right) 
+ b \left(\cos\beta(t) \left(\cos\alpha(t) \vect{i_0} + \sin\alpha(t) \vect{j_0} \right) - \sin\beta(t) \vect{k_0} \right) \\
&& + c\left(\cos\beta(t) \vect{k_0} + \sin\beta(t)  \left(\cos\alpha(t) \vect{i_0} + \sin\alpha(t) \vect{j_0} \right) \right)\\
&=& \left[ \begin{array}{c} 
a \cos\alpha(t) + b \cos\beta(t) \cos\alpha(t) +c \sin\beta(t)\cos\alpha(t)\\
a \sin\alpha(t) + b \cos\beta(t) \sin\alpha(t) +c \sin\beta(t)\sin\alpha(t) \\
- b\sin\beta(t) + c\cos\beta(t)
\end{array}\right]_{\mathcal{R}_0}
\end{eqnarray*}

On a ainsi l'équation paramétrique de la position du point $G$.

\end{corrige}
\else \fi


\subparagraph{}
\textit{Calculer $\vectv{O_0}{S_1}{S_0}$.}

\subparagraph{}
\textit{Calculer $\vectv{O_1}{S_2}{S_1}$.}

\subparagraph{}
\textit{Calculer $\vectv{O_1}{S_1}{S_0}$ et $\vectg{O_1}{S_1}{S_0}$.}
\ifprof
\begin{corrige}


\textbf{Méthode 1 -- }
Par définition, 
$$
\vectv{O_1}{S_1}{S_0} 
= \left[\dfrac{d\vect{O_0O_1}(t)}{dt}\right]_{\mathcal{R}_0}
= \left[\dfrac{d \left(a \vect{i_1}\right) }{dt}\right]_{\mathcal{R}_0}
= a \left[\dfrac{d  \vect{i_1} }{dt}\right]_{\mathcal{R}_0}
$$

On a :
\begin{eqnarray*}
\left[\dfrac{d  \vect{i_1} }{dt}\right]_{\mathcal{R}_0}
&=&\left[\dfrac{d \left(\cos\alpha(t)\vect{i_0}+\sin\alpha(t)\vect{j_0} \right)}{dt}\right]_{\mathcal{R}_0}
=\left[\dfrac{d  \cos\alpha(t)\vect{i_0}}{dt}\right]_{\mathcal{R}_0}
+\left[\dfrac{d  \sin\alpha(t)\vect{j_0} }{dt}\right]_{\mathcal{R}_0}\\
& = & 
\dfrac{d \cos\alpha(t)}{dt} \vect{i_0}  
+\cos\alpha(t)\underbrace{\left[\dfrac{d  \vect{i_0}}{dt}\right]_{\mathcal{R}_0}}_{\vect{0}}
+\dfrac{d \sin\alpha(t)}{dt} \vect{i_0}  
+\sin(t)\underbrace{\left[\dfrac{d  \vect{j_0}}{dt}\right]_{\mathcal{R}_0}}_{\vect{0}}\\
& = & -\dot{\alpha}(t)\sin\alpha(t) \vect{i_0}   + \dot{\alpha}(t)\cos\alpha(t) \vect{j_0}  = 
\dot{\alpha}(t)\vect{j_1}
\end{eqnarray*}

Ainsi,
$$
\vectv{O_1}{S_1}{S_0} 
= \left[\begin{array}{c} 
-a\dot{\alpha}(t)\sin\alpha(t) \\
a \dot{\alpha}(t)\cos\alpha(t) \\
0 \end{array}\right]_{\mathcal{R}_0}
=\left[\begin{array}{c} 0 \\ a\dot{\alpha}(t) \\ 0\end{array}\right]_{\mathcal{R}_1}
$$

Dans les deux cas, $\vect{O_0O_1}(t)$ est dérivé par rapport $\mathcal{R}_0$ mais il s'exprime différemment dans $\mathcal{R}_0$ et $\mathcal{R}_1$ :
\begin{itemize}
\item $\vectv{O_1}{S_1}{S_0} = -a\dot{\alpha}(t)\sin\alpha(t) \vect{i_0}   + a\dot{\alpha}(t) \cos\alpha(t) \vect{j_0}$ : ici la base de \textbf{projection} et de \textbf{dérivation} est la base $\mathcal{B}_0$;
\item $\vectv{O_1}{S_1}{S_0} = a\dot{\alpha}(t)\vect{j_1}$ : ici la base de dérivation est la base $\mathcal{B}_0$ et la base de projection est $\mathcal{B}_1$.
\end{itemize}


\textbf{Méthode 2 -- Utilisation de la dérivation vectorielle.}

Calcul de $\vectv{O_1}{S_1}{S_0}$.

On rappelle que :
$$
\vectv{O_1}{S_1}{S_0} 
= a \left[\dfrac{d  \vect{i_1} }{dt}\right]_{\mathcal{R}_0}
$$

Le calcul de $\left[\dfrac{d  \vect{i_1} }{dt}\right]_{\mathcal{R}_0}$ peut donc être réalisé ainsi : 
$$ 
\left[\dfrac{d  \vect{i_1} }{dt}\right]_{\mathcal{R}_0} = 
\underbrace{\left[\dfrac{d  \vect{i_1} }{dt}\right]_{\mathcal{R}_1}}_{\vect{0}} + \vecto{S_1}{S_0}\wedge \vect{i_1}
=\dot{\alpha}\vect{k_0}  \wedge \vect{i_1}
=\dot{\alpha} \vect{j_1}
$$

Ainsi 
$$
\vectv{O_1}{S_1}{S_0} 
= a \dot{\alpha} \vect{j_1}
$$

\textbf{Méthode 3 -- }
Calcul de $\vectv{O_1}{S_1}{S_0}$.

$S_1$ et $S_0$ sont en liaison pivot de centre $O_0$, on a donc :  $\vectv{O_0}{S_1}{S_0}=\vect{0}$.

En conséquence, 
$$
\vectv{O_1}{S_1}{S_0} = \vectv{O_0}{S_1}{S_0} + \vect{O_1O_0}\wedge   \vecto{S_1}{S_0} = \vect{0} - a \vect{i_1} \wedge \left( \dot{\alpha}\vect{k_0} \right)
=a \dot{\alpha}\vect{j_1}
$$

\end{corrige}\else \fi


\subparagraph{}
\textit{Calculer $\vecto{S_1}{S_0}$, $\vecto{S_2}{S_1}$ et $\vecto{S_2}{S_0}$.}



\subparagraph{}
\textit{Calculer $\vectv{G}{S_2}{S_0}$.}
\ifprof
\begin{corrige}

Calcul de $\vectv{G}{S_2}{S_0}$.
%On a :
%$$\vecto{S_2}{S_0}=\vecto{S_2}{S_1}+\vecto{S_1}{S_0} = \dot{\alpha}\vect{k_0} + \dot{\beta}\vect{j_1}  $$

%Par ailleurs, 
On a : 
$$\vectv{G}{S_2}{S_0} = \vectv{G}{S_2}{S_1} + \vectv{G}{S_1}{S_0} $$
Calculons $\vectv{G}{S_1}{S_0}$ :
$$
\vectv{G}{S_1}{S_0} 
= \vectv{O_1}{S_1}{S_0} + \vect{GO_1}\wedge\vecto{S_1}{S_0}
= a\dot{\alpha}\vect{j_1} - \left(b\vect{i_2} + c\vect{k_2}\right) \wedge\left( \dot{\alpha} \vect{k_0}\right)
$$
$$
\vectv{G}{S_1}{S_0} 
= a\dot{\alpha}\vect{j_1}  + b\dot{\alpha} \sin(\beta+\pi/2) \vect{j_1} + c \dot{\alpha}\sin\beta\vect{j_1}
= \dot{\alpha} \left(a+b \cos\beta + c \sin\beta\right) \vect{j_1} 
$$

Par ailleurs calculons $\vectv{G}{S_2}{S_1}$ :
$$\vectv{G}{S_2}{S_1} = \vectv{O_1}{S_2}{S_1} + \vect{GO_1}\wedge \vecto{S_2}{S_1}
=-\left(b\vect{i_2}+c\vect{k_2}\right) \wedge \left(\dot{\beta}\vect{j_1}\right)
=-\dot{\beta}\left(b\vect{k_2}-c\vect{i_2} \right)
$$

Au final, 
$$\vectv{G}{S_2}{S_0} = \dot{\alpha} \left(a+b \cos\beta + c \sin\beta\right) \vect{j_1} 
-\dot{\beta}\left(b\vect{k_2}-c\vect{i_2} \right)
$$

Il est aussi possible de calculer $\vectv{G}{S_2}{S_0}$ ainsi : 
$$\vectv{G}{S_2}{S_0} = \left[\dfrac{d \vect{O_0G}}{dt}\right]_{\mathcal{R}_0}$$ 

\end{corrige}\else \fi




\subparagraph{}
\textit{Calculer $\vectg{G}{S_2}{S_0}$.}
\ifprof
\begin{corrige}
\end{corrige}\else \fi



\begin{thebibliography}{2}
\bibitem[1]{cite1} Centrifugeuse humaine -- CNRS Photothèque/Sébastien Godefroy et MEDES, \textit{Avio et Tiger}, \url{http://www.medes.fr/home_fr/fiche-centrifugeuse/mainColumnParagraphs/0/document/Presentation\%20centrifugeuse%2018.12.07.pdf}.
\end{thebibliography}
\end{multicols}
\newpage \setcounter{exo}{0}
\begin{multicols}{2}

\section*{Robot de peinture}

\vspace{.25cm}

On étudie un robot de peinture de voiture. Ce robot se déplace par rapport à une carrosserie de voiture, et projette dessus de la peinture. L'objectif est de déterminer les lois du mouvement du robot, pour lui permettre de vérifier le critère de vitesse de déplacement relatif (entre le robot et la carrosserie de voiture) du cahier des charges.

\begin{center}
\includegraphics[width=\linewidth]{images/fig_1}

\end{center}

\vspace{.25cm}


La modélisation cinématique du robot est donnée sur la figure suivante :


\begin{center}
\includegraphics[width=\linewidth]{images/fig_2}
\end{center}



\vspace{.25cm}

Le chariot $S_1$, auquel on associe le repère $\mathcal{R}_1\left(A,\vect{x_1},\vect{y_1},\vect{z_1}\right)$ est en mouvement de translation de direction $\vect{y_0}$ par rapport au bâti $S_0$ de repère $\mathcal{R}_0\left(A,\vect{x_0},\vect{y_0},\vect{z_0}\right)$. 

Le corps $S_2$, auquel on associe le repère $\mathcal{R}_2\left(A,\vect{x_2},\vect{y_2},\vect{z_2}\right)$ est en mouvement de rotation autour de l'axe $(B,\vect{z_0})$ avec le chariot $S_1$. 

Le bras $S_3$, auquel on associe le repère $\mathcal{R}_3\left(B,\vect{x_3},\vect{y_3},\vect{z_3}\right)$ est en mouvement de rotation autour de l'axe $(B,\vect{y_2})$ avec le corps $S_2$. 

On a $\vect{OD}=b\vect{y_0}$ avec $b=\sqrt{L^2-H^2}$.




\subparagraph{}
\textit{Construire les figures planes de repérage/paramétrage puis exprimer les vecteurs vitesse instantanée de rotation $\vecto{1}{0}$, $\vecto{2}{1}$, $\vecto{3}{2}$.}

\subparagraph{}
\textit{Déterminer $\vectv{P}{3}{0}$}

On désire que $P$ décrive la droite $(D,\vect{x_0})$ à vitesse constante $V$, conformément au cahier des charges. 

\subparagraph{}
\textit{Représenter sur une figure dans le plan $(O,\vect{x_0},\vect{y_0})$, puis sur une figure dans le plan $(O,\vect{x_0},\vect{y_0})$, les positions des points $O$, $D$, $A$, $B$ et $P$ du robot lorsque celui-ci est en position extrême ($A$ est en $D$).}


\subparagraph{}
\textit{Traduire, à l'aide de l'expression de $\vectv{P}{3}{0}$ le fait que $P$ se déplace à la vitesse $V$ selon $\vect{x_0}$. En déduire $\dot{\beta}$.}

\subparagraph{}
\textit{Exprimer alors $\dot{\lambda}$ et $\dot{\alpha}$ en fonction de $L$, $V$, $\alpha$ et $\beta_0$. }

\subparagraph{}
\textit{A l'aide de la figure précédente, exprimer $\beta_0$ en fonction de $b$ et $L$.}

\subparagraph{}
\textit{Exprimer $\dot{\lambda}$ et $\dot{\alpha}$ en fonction de $V$, $b$ et $\alpha$.}

\end{multicols}

\end{document}


