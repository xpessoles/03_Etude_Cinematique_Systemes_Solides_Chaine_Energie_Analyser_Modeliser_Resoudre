\documentclass[10pt,oneside]{article}
\usepackage[T1]{fontenc}
\usepackage[utf8]{inputenc}
%\DeclareUnicodeCharacter{00A0}{ }
\usepackage[adobe-utopia]{mathdesign}

\usepackage{amsmath}
\usepackage[francais]{babel}
\usepackage[dvips]{graphicx}
%\usepackage{here}
\usepackage{framed}
\usepackage[normalem]{ulem}
\usepackage{fancyhdr}
\usepackage{titlesec}
\usepackage{vmargin}

\usepackage{amsmath}
\usepackage{ifthen}
\usepackage{multirow}
\usepackage{multicol} % Portions de texte en colonnes

%\usepackage{xltxtra} % Logo XeLaTeX
%\usepackage{pst-solides3d}
\usepackage{color}
%\usepackage{colortbl}
\usepackage{titletoc} % Pour la mise en forme de la table des matières

%\usepackage[crop=off]{auto-pst-pdf}
%\usepackage{bclogo}


%\usepackage{longtable}
%\usepackage{flafter}%floatants après la référence
%\usepackage{pst-solides3d}
%\usepackage{pstricks}
%\usepackage{minitoc}
%\setcounter{minitocdepth}{4}
%\usepackage{draftcopy}% "Brouillon"
%\usepackage{floatflt}
%\usepackage{psfrag}
%\usepackage{listings} % Permet d'insérer du code de programmation
%\usepackage{lmodern}
%\usepackage[adobe-utopia,uppercase=upright,greeklowercase=upright]{mathdesign}
%\usepackage{minionpro}
%\usepackage{pifont}
%\usepackage{amssymb}
%\usepackage[francais]{varioref}

\setmarginsrb{1.5cm}{1cm}{1cm}{1.5cm}{1cm}{1cm}{1cm}{1cm}

\definecolor{gris25}{gray}{0.75}
\definecolor{bleu}{RGB}{18,33,98}
\definecolor{bleuf}{RGB}{42,94,171}
\definecolor{bleuc}{RGB}{231,239,247}
\definecolor{rougef}{RGB}{185,18,27}
\definecolor{rougec}{RGB}{255,230,231}
\definecolor{vertf}{RGB}{103,126,82}
\definecolor{vertc}{RGB}{220,255,191}
\definecolor{violetf}{RGB}{112,48,160}
\definecolor{violetc}{RGB}{230,224,236}
\definecolor{jaunec}{RGB}{220,255,191}

\usepackage[%
    pdftitle={CIN -- Géométrie vectorielle -- Applications},
    pdfauthor={Xavier Pessoles},
    colorlinks=true,
    linkcolor=blue,
    citecolor=magenta]{hyperref}



% \makeatletter \let\ps@plain\ps@empty \makeatother
%% DEBUT DU DOCUMENT
%% =================
\sloppy
\hyphenpenalty 10000

\newcommand{\Pointilles}[1][3]{%
\multido{}{#1}{\makebox[\linewidth]{\dotfill}\\[\parskip]
}}


\begin{document}


\newboolean{prof}
\setboolean{prof}{false}
%------------- En tetes et Pieds de Pages ------------
\pagestyle{fancy}
\renewcommand{\headrulewidth}{0pt}

\fancyhead{}
\fancyhead[L]{%
\noindent\noindent\begin{minipage}[c]{2.6cm}
%Lycée Rouvière PTSI
\includegraphics[width=2cm]{png/logo_ptsi.png}%
\end{minipage}
}

\fancyhead[C]{\rule{12cm}{.5pt}}

\fancyhead[R]{%
\begin{minipage}[c]{3cm}
\begin{flushright}
\footnotesize{\textit{\textsf{Sciences Industrielles\\ de l'Ingénieur}}}%
\end{flushright}
\end{minipage}
}

\renewcommand{\footrulewidth}{0.2pt}

\fancyfoot[C]{\footnotesize{\bfseries \thepage}}
\fancyfoot[L]{\footnotesize{2013 -- 2014} \\ X. \textsc{Pessoles}}
\ifthenelse{\boolean{prof}}{%
\fancyfoot[R]{\footnotesize{CI 3 : CIN -- Applications} \\ \footnotesize{Ch 5 : Cinématique du solide -- P}}
}{%
\fancyfoot[R]{\footnotesize{CI 3 : CIN -- Applications} \\ \footnotesize{Ch 5 : Cinématique du solide -- E}}
}


%\begin{center}
%\textit{Centre d'intérêt}
%\end{center}



\begin{center}
 \Large\textsc{CI 3 -- CIN : Étude du comportement cinématique des systèmes}
\end{center}

\begin{center}
 \large\textsc{Chapitre 5 -- Cinématique du solide indéformable}
\end{center}

\begin{center}
\textsc{Exercices d'application} 
\end{center}
\begin{flushright}
\textit{D'après ressources de Jean-Pierre Pupier.} 
\end{flushright}
\vspace{.5cm}

\subsection*{Exercice 1 -- Dérivation vectorielle}
\begin{center}
\includegraphics[width=.95\textwidth]{png/fig1}
\end{center}


\subparagraph{}
\textit{Faire les calculs suivants : 
$\left[ \dfrac{d\vect{y_1}}{dt}\right]_{\mathcal{R}_0}$, 
$\left[ \dfrac{d\vect{x_0}}{dt}\right]_{\mathcal{R}_1}$, 
$\left[ \dfrac{d\vect{y_1}}{dt}\right]_{\mathcal{R}_3}$, 
$\left[ \dfrac{d\vect{z_2}}{dt}\right]_{\mathcal{R}_0}$, 
$\left[ \dfrac{d\vect{y_3}}{dt}\right]_{\mathcal{R}_0}$, 
$\left[ \dfrac{d\vect{x_3}}{dt}\right]_{\mathcal{R}_0}$.}
\ifthenelse{\boolean{prof}}{
\begin{corrige}
\end{corrige}
}{}
\subparagraph{}
\textit{Faire les calculs suivants : 
$\left[ \dfrac{d\vect{V}}{dt}\right]_{\mathcal{R}_0}$ avec $\vect{V} = 3\cos\alpha(t)\vect{x_1}$,
$\left[ \dfrac{d\vect{U}}{dt}\right]_{\mathcal{R}_0}$ avec $\vect{U} = -7\sin\alpha(t)\vect{y_2}$,
$\left[ \dfrac{d\vect{W}}{dt}\right]_{\mathcal{R}_0}$ avec $\vect{W} = -3\alpha(t)^3\vect{y_1}+6\sin\alpha(t)\vect{y_0}$,
$\left[ \dfrac{d\vect{S}}{dt}\right]_{\mathcal{R}_0}$ avec $\vect{S} = 4t^3\alpha(t)\cos\alpha(t)\vect{y_1}$.
}


\subsection*{Exercice 2}
\setcounter{subparagraph}{0}
\begin{minipage}[c]{.55\linewidth}
\begin{center}
\includegraphics[width=.95\textwidth]{png/fig2}
\end{center}
\end{minipage}\hfill
\begin{minipage}[c]{.4\linewidth}
Soit le mécanisme plan constitué par :
\begin{itemize}
\item solide $S_0$ : fixe, repère lié $\mathcal{R}_0=\left( O,\vect{x_0},\vect{y_0},\vect{z_0}\right)$;
\item solide $S_1$ : barre $OP$ de longueur $L$, en liaison pivot d’axe $(O,\vect{z_0})$ par rapport à $S_0$, repère lié  
à $\mathcal{R}_1=\left( O,\vect{x_1},\vect{y_1},\vect{z_0}\right)$
\item solide $S_2$ : disque de centre $P$ et de rayon $R$, en liaison pivot d’axe $(P,\vect{z_0})$ par rapport à $S_1$, repère lié à
$\mathcal{R}_2=\left( P,\vect{x_2},\vect{y_2},\vect{z_0}\right)$.
\end{itemize}
On note :
\begin{itemize}
\item $\alpha = \left(\vect{x_0};\vect{x_1}\right)$;
\item $\beta = \left(\vect{x_1};\vect{x_2}\right)$.
\end{itemize}

\end{minipage}

\subparagraph{}
\textit{Déterminer la trajectoire du point $M$ dans le repère $\mathcal{R}_0$.}
\ifthenelse{\boolean{prof}}{
\begin{corrige}
\end{corrige}
}{}

\subparagraph{}
\textit{Déterminer $\vecto{S_1}{S_0}$, $\vecto{S_2}{S_1}$, $\vecto{S_2}{S_0}$.}
\ifthenelse{\boolean{prof}}{
\begin{corrige}
\end{corrige}
}{}

\subparagraph{}
\textit{Déterminer $\vectv{M}{S_2}{S_0}$.}
\ifthenelse{\boolean{prof}}{
\begin{corrige}
\end{corrige}
}{}

\subparagraph{}
\textit{Déterminer $\vectv{I}{S_2}{S_0}$.}
\ifthenelse{\boolean{prof}}{
\begin{corrige}
\end{corrige}
}{}


\subparagraph{}
\textit{Pourquoi ne faut-il absolument pas dériver le vecteur $\vect{OI}$ ? .}
\ifthenelse{\boolean{prof}}{
\begin{corrige}
\end{corrige}
}{}

\subparagraph{}
\textit{Déterminer $\vectg{M}{S_2}{S_0}$.}
\ifthenelse{\boolean{prof}}{
\begin{corrige}
\end{corrige}
}{}



\begin{minipage}[c]{.4\linewidth}
Le mécanisme précédent a été en réalité complété par un cercle de centre $O$ lié à $S_0$ et de rayon $R$ (voir la figure ci-contre).

 
Par ailleurs on adopte $L = R$. De plus à $t = 0$,  $\alpha=\beta=0$.

$S_1$ est un bras porte satellite et $S_2$ un satellite qui roule sans glisser en $I$ sur $S_0$. Cette condition se traduit par $\vectv{I}{S_2}{S_0}$.

\end{minipage}\hfill
\begin{minipage}[c]{.55\linewidth}
\begin{center}
\includegraphics[width=.95\textwidth]{png/fig3}
\end{center}
\end{minipage}
 

\subparagraph{}
\textit{Déduire des questions précédentes la relation entre $\dot{\alpha}$ et $\dot{\beta}$.}
\ifthenelse{\boolean{prof}}{
\begin{corrige}
\end{corrige}
}{}


\subparagraph{}
\textit{Donner l'expression de $\vectv{M}{S_2}{S_0}$ en projection dans $\mathcal{R}_0 = \left(O,\vect{x_0},\vect{y_0},\vect{z_0}\right)$.}
\ifthenelse{\boolean{prof}}{
\begin{corrige}
\end{corrige}
}{}

\subparagraph{}
\textit{En déduire la trajectoire du point $M$ par rapport à $\mathcal{R}_0$.}
\ifthenelse{\boolean{prof}}{
\begin{corrige}
\end{corrige}
}{}




\end{document}