\documentclass[11pt,oneside]{article}
\usepackage[T1]{fontenc}
\usepackage[utf8]{inputenc}
%\DeclareUnicodeCharacter{00A0}{ }
\usepackage[adobe-utopia]{mathdesign}

\usepackage{amsmath}
\usepackage[francais]{babel}
\usepackage[dvips]{graphicx}
%\usepackage{here}
\usepackage{framed}
\usepackage[normalem]{ulem}
\usepackage{fancyhdr}
\usepackage{titlesec}
\usepackage{vmargin}

\usepackage{amsmath}
\usepackage{ifthen}
\usepackage{multirow}
\usepackage{multicol} % Portions de texte en colonnes

%\usepackage{xltxtra} % Logo XeLaTeX
%\usepackage{pst-solides3d}
\usepackage{color}
%\usepackage{colortbl}
\usepackage{titletoc} % Pour la mise en forme de la table des matières

%\usepackage[crop=off]{auto-pst-pdf}
%\usepackage{bclogo}


%\usepackage{longtable}
%\usepackage{flafter}%floatants après la référence
%\usepackage{pst-solides3d}
%\usepackage{pstricks}
%\usepackage{minitoc}
%\setcounter{minitocdepth}{4}
%\usepackage{draftcopy}% "Brouillon"
%\usepackage{floatflt}
%\usepackage{psfrag}
%\usepackage{listings} % Permet d'insérer du code de programmation
%\usepackage{lmodern}
%\usepackage[adobe-utopia,uppercase=upright,greeklowercase=upright]{mathdesign}
%\usepackage{minionpro}
%\usepackage{pifont}
%\usepackage{amssymb}
%\usepackage[francais]{varioref}

\setmarginsrb{1.5cm}{1cm}{1cm}{1.5cm}{1cm}{1cm}{1cm}{1cm}

\definecolor{gris25}{gray}{0.75}
\definecolor{bleu}{RGB}{18,33,98}
\definecolor{bleuf}{RGB}{42,94,171}
\definecolor{bleuc}{RGB}{231,239,247}
\definecolor{rougef}{RGB}{185,18,27}
\definecolor{rougec}{RGB}{255,230,231}
\definecolor{vertf}{RGB}{103,126,82}
\definecolor{vertc}{RGB}{220,255,191}
\definecolor{violetf}{RGB}{112,48,160}
\definecolor{violetc}{RGB}{230,224,236}
\definecolor{jaunec}{RGB}{220,255,191}

\usepackage[%
    pdftitle={CIN -- Géométrie vectorielle -- Applications},
    pdfauthor={Xavier Pessoles},
    colorlinks=true,
    linkcolor=blue,
    citecolor=magenta]{hyperref}



% \makeatletter \let\ps@plain\ps@empty \makeatother
%% DEBUT DU DOCUMENT
%% =================
\sloppy
\hyphenpenalty 10000

\newcommand{\Pointilles}[1][3]{%
\multido{}{#1}{\makebox[\linewidth]{\dotfill}\\[\parskip]
}}


\begin{document}


\newboolean{prof}
\setboolean{prof}{false}
%------------- En tetes et Pieds de Pages ------------
\pagestyle{fancy}
\renewcommand{\headrulewidth}{0pt}

\fancyhead{}
\fancyhead[L]{%
\noindent\noindent\begin{minipage}[c]{2.6cm}
%Lycée Rouvière PTSI
\includegraphics[width=2cm]{png/logo_ptsi.png}%
\end{minipage}
}

\fancyhead[C]{\rule{12cm}{.5pt}}

\fancyhead[R]{%
\begin{minipage}[c]{3cm}
\begin{flushright}
\footnotesize{\textit{\textsf{Sciences Industrielles\\ de l'Ingénieur}}}%
\end{flushright}
\end{minipage}
}

\renewcommand{\footrulewidth}{0.2pt}

\fancyfoot[C]{\footnotesize{\bfseries \thepage}}
\fancyfoot[L]{\footnotesize{2013 -- 2014} \\ X. \textsc{Pessoles}}
\ifthenelse{\boolean{prof}}{%
\fancyfoot[R]{\footnotesize{CI 3 : CIN -- Applications} \\ \footnotesize{Ch. 2 : Géométrie -- P}}
}{%
\fancyfoot[R]{\footnotesize{CI 3 : CIN -- Applications} \\ \footnotesize{Ch. 2 : Géométrie -- E}}
}


%\begin{center}
%\textit{Centre d'intérêt}
%\end{center}



\begin{center}
 \Large\textsc{CI 3 -- CIN : Étude du comportement cinématique des systèmes}
\end{center}

\begin{center}
 \large\textsc{Chapitre 2 -- Géométrie dans l'espace}
\end{center}

\begin{center}
\textsc{Exercices d'application} 
\end{center}
\begin{flushright}
\textit{D'après ressources de Jean-Pierre Pupier.} 
\end{flushright}
\vspace{.5cm}

\subsection*{Exercice 1}
Soit un repère $\mathcal{R}=\left(O,\vect{x},\vect{y},\vect{z} \right)$. On donne les coordonnées dans $\mathcal{R}$ des points suivants correspondants respectivement à l'origine et à l'extrémité des vecteurs :
\begin{itemize}
\item $\vect{V_1}$ : point $A_1$ : $(2,1,0)$, point $B_1$ : $(3,1,0)$;
\item $\vect{V_2}$ : point $A_2$ : $(1,-3,0)$, point $B_2$ : $(-2,-1,0)$;
\item $\vect{V_3}$ : point $A_3$ : $(1,1,0)$, point $B_3$ : $(3,2,0)$;
\item $\vect{V_4}$ : point $A_4$ : $(-1,2,0)$, point $B_4$ : $(1,1,0)$.
\end{itemize}

\subparagraph{}
\textit{Calculer les composantes de chaque vecteur dans la base $\mathcal{B}$ associée au repère $\mathcal{R}$.}

\ifthenelse{\boolean{prof}}{
\begin{corrige}
\end{corrige}
}{}



\subparagraph{}
\textit{Calculer la norme de chaque vecteur.}

\ifthenelse{\boolean{prof}}{
\begin{corrige}
\end{corrige}
}{}

\subparagraph{}
\textit{Calculer la somme de ces quatre vecteurs dans la base $\mathcal{B}$.}

\ifthenelse{\boolean{prof}}{
\begin{corrige}
\end{corrige}
}{}

\subparagraph{}
\textit{Écrire les composantes du vecteur unitaire colinéaire à $\vect{V_2}$ et de même sens dans la base $\mathcal{B}$. }

\ifthenelse{\boolean{prof}}{
\begin{corrige}
\end{corrige}
}{}

\subparagraph{}
\textit{Calculer les produits scalaires $\vect{V_1}\cdot \vect{V_2}$ et  $\vect{V_3}\cdot \vect{V_4}$.}

\ifthenelse{\boolean{prof}}{
\begin{corrige}
\end{corrige}
}{}

\subparagraph{}
\textit{Calculer les produits vectoriels $\vect{V_1}\wedge \vect{V_2}$ et  $\vect{V_3}\wedge \vect{V_4}$.}
\ifthenelse{\boolean{prof}}{
\begin{corrige}
\end{corrige}
}{}



\subsection*{Exercice 2}
\setcounter{subparagraph}{0}
\subparagraph{}
\textit{Dessiner le troisième vecteur de la base orthonormée directe $\mathcal{B}=\left(\vect{x},\vect{y},\vect{z}\right)$.}

\ifthenelse{\boolean{prof}}{
\begin{corrige}
\end{corrige}
}{}

\begin{center}
\includegraphics[width=.9\textwidth]{png/exo2}
\end{center}
\subparagraph{}
\textit{Exprimer les produits des vecteurs de base d'une base orthonormée directe.}
$$
\vect{x}\cdot\vect{y}
\quad
\vect{x}\wedge\vect{y}
\quad
\vect{y}\cdot\vect{z}
\quad
\vect{y}\wedge\vect{z}
\quad
\vect{x}\cdot\vect{z}
\quad
\vect{x}\wedge\vect{z}
$$

\ifthenelse{\boolean{prof}}{
\begin{corrige}
\end{corrige}
}{}

\subparagraph{}
\textit{Calculer le cosinus puis l'angle $\alpha$ formé par les vecteurs $\vect{V_1}=\left[\begin{array}{c}-1\\2\\2\end{array}\right]_{\mathcal{B}}$ et 
$\vect{V_2}=\left[\begin{array}{c}3\\-2\\3\end{array}\right]_{\mathcal{B}}$.}

\ifthenelse{\boolean{prof}}{
\begin{corrige}
\end{corrige}
}{}

\subparagraph{}
\textit{Calculer le sinus puis l'angle $\gamma$ formé par les vecteurs $\vect{V_1}=\left[\begin{array}{c}1\\4\\8\end{array}\right]_{\mathcal{B}}$ et 
$\vect{V_2}=\left[\begin{array}{c}-2\\5\\3\end{array}\right]_{\mathcal{B}}$.}

\ifthenelse{\boolean{prof}}{
\begin{corrige}
\end{corrige}
}{}

\subparagraph{}
\textit{Calculer l'angle entre $\vect{V}=10\vect{x}+8\vect{y}+6\vect{z}$ et le vecteur de base $\vect{x}$.}

\ifthenelse{\boolean{prof}}{
\begin{corrige}
\end{corrige}
}{}

\subsection*{Exercice 3}
\setcounter{subparagraph}{0}

\subparagraph{}
\textit{Représentez un repère orthonormé $\mathcal{R}=\left(O,\vect{x},\vect{y},\vect{z} \right)$ en vue orthogonale ($\vect{y}$ vertical, $\vect{x}$ horizontal, $\vect{z}$ vers <<nous>>), puis un repère orthonormé $\mathcal{R}_1=\left(O,\vect{x_1},\vect{y_1},\vect{z} \right)$ tel que $\alpha=\left(\vect{x},\vect{x_1}\right)$. Mettez un point $M$ tel que $\vect{OM}=a\vect{x_1}$ avec $a>0$.}

\ifthenelse{\boolean{prof}}{
\begin{corrige}
\end{corrige}
}{}

\subparagraph{}
\textit{Exprimer les composantes de $\vect{OM}$ en projection sur la base $\mathcal{B}$ liée au repère $\mathcal{R}$.}


\ifthenelse{\boolean{prof}}{
\begin{corrige}
\end{corrige}
}{}

\subparagraph{}
\textit{Exprimer $\vect{z}\wedge\vect{OM}$. Vous l'exprimerez en projection sur la base $\mathcal{B}$ puis dans $\mathcal{B}_1$ (utiliser plusieurs méthodes).}

\ifthenelse{\boolean{prof}}{
\begin{corrige}
\end{corrige}
}{}



\subsection*{Exercice 4}
\setcounter{subparagraph}{0}

On donne les coordonnées de trois points dans le repère orthonormé $\mathcal{R}  = \left(D,\vect{x},\vect{y},\vect{z} \right)$:
$$A : (3, 2, 0) \quad B : (0, 3, 2) \quad C : (2, 3, 0)$$
\subparagraph{}
\textit{Calculez les composantes du vecteur $\vect{V}$ de norme 1000 colinéaire à $\vect{AB}$ et de même sens.}

\ifthenelse{\boolean{prof}}{
\begin{corrige}
\end{corrige}
}{}

\subparagraph{}
\textit{Calculez le moment au point $A$ du pointeur $(D,\vect{D})$ où  $\vect{D}=(200,300,-100)_{\mathcal{B}}$.}

\ifthenelse{\boolean{prof}}{
\begin{corrige}
\end{corrige}
}{}

\subparagraph{}
\textit{Calculez le moment au point $E$ milieu de $AB$, du pointeur$(D,\vect{D})$.}

\ifthenelse{\boolean{prof}}{
\begin{corrige}
\end{corrige}
}{}

\subparagraph{}
\textit{Calculez le moment par rapport à l'axe $\delta$ (orienté de $A$ vers $B$) du pointeur $(D,\vect{D})$. }

\ifthenelse{\boolean{prof}}{
\begin{corrige}
\end{corrige}
}{}



\subsection*{Exercice 5}
\setcounter{subparagraph}{0}

\begin{minipage}[c]{.7\linewidth}
\subparagraph{}
\textit{On note $\mathcal{B}=(\vect{x},\vect{y},\vect{z})$, $\mathcal{B}_1=(\vect{x_1},\vect{y_1},\vect{z})$,  $\alpha=\left(\vect{x},\vect{x_1}\right)$, $\beta=\left(\vect{x_1},\vect{V} \right)$.}

\textit{Exprimer les composantes scalaires sous formes de colonnes du vecteur $\vect{V}$ en projection sur la base $\mathcal{B}_1$ puis sur la base $\mathcal{B}$ et ceci en fonction de la norme de $\vect{V}$ notée simplement $V$ et des angles orientés $\alpha$ et $\beta$. }
\end{minipage} \hfill
\begin{minipage}[c]{.29\linewidth}
\begin{center}
\includegraphics[height=4cm]{png/exo4_1}
\end{center}
\end{minipage}

\ifthenelse{\boolean{prof}}{
\begin{corrige}
\end{corrige}
}{}


\begin{minipage}[c]{.7\linewidth}
\subparagraph{}
\textit{Même question avec $\mathcal{B}=(\vect{x},\vect{y},\vect{z})$, $\mathcal{B}_1=(\vect{x_1},\vect{y_1},\vect{z})$,  $\alpha=\left(\vect{x},\vect{x_1}\right)$, $\beta=\left(\vect{y_1},\vect{V} \right)$.}
\end{minipage} \hfill
\begin{minipage}[c]{.29\linewidth}
\begin{center}
\includegraphics[height=4cm]{png/exo4_2}
\end{center}
\end{minipage}

\ifthenelse{\boolean{prof}}{
\begin{corrige}
\end{corrige}
}{}

\begin{minipage}[c]{.7\linewidth}
\subparagraph{}
\textit{Même question avec $\mathcal{B}=(\vect{x},\vect{y},\vect{z})$, $\mathcal{B}_1=(\vect{x_1},\vect{y_1},\vect{z})$,  $\alpha=\left(\vect{z},\vect{V}\right)$, $\beta=\left(\vect{x},\vect{x_1} \right)$.}
\end{minipage} \hfill
\begin{minipage}[c]{.29\linewidth}
\begin{center}
\includegraphics[height=4cm]{png/exo4_3}
\end{center}
\end{minipage}

\ifthenelse{\boolean{prof}}{
\begin{corrige}
\end{corrige}
}{}


\begin{minipage}[c]{.7\linewidth}
\subparagraph{}
\textit{Même question avec $\mathcal{B}=(\vect{x},\vect{y},\vect{z})$, $\mathcal{B}_1=(\vect{x_1},\vect{y},\vect{z_1})$,  $\alpha=\left(\vect{z_1},\vect{V}\right)$, $\beta=\left(\vect{z},\vect{z_1} \right)$.}
\end{minipage} \hfill
\begin{minipage}[c]{.29\linewidth}
\begin{center}
\includegraphics[height=4cm]{png/exo4_4}
\end{center}
\end{minipage}

\ifthenelse{\boolean{prof}}{
\begin{corrige}
\end{corrige}
}{}


\begin{minipage}[c]{.7\linewidth}
\subparagraph{}
\textit{Même question avec $\mathcal{B}=(\vect{x},\vect{y},\vect{z})$, $\mathcal{B}_1=(\vect{x},\vect{y_1},\vect{z_1})$,  $\alpha=\left(\vect{x},\vect{V}\right)$, $\beta=\left(\vect{z},\vect{z_1} \right)$.}
\end{minipage} \hfill
\begin{minipage}[c]{.29\linewidth}
\begin{center}
\includegraphics[height=4cm]{png/exo4_5}
\end{center}
\end{minipage}

\ifthenelse{\boolean{prof}}{
\begin{corrige}
\end{corrige}
}{}


\begin{minipage}[c]{.7\linewidth}
\subparagraph{}
\textit{Même question avec $\mathcal{B}=(\vect{x},\vect{y},\vect{z})$, $\mathcal{B}_1=(\vect{x_1},\vect{y_1},\vect{z})$,  $\alpha=\left(\vect{y_1},\vect{V}\right)$, $\beta=\left(\vect{y},\vect{y_1} \right)$.}
\end{minipage} \hfill
\begin{minipage}[c]{.29\linewidth}
\begin{center}
\includegraphics[height=4cm]{png/exo4_6}
\end{center}
\end{minipage}

\ifthenelse{\boolean{prof}}{
\begin{corrige}
\end{corrige}
}{}


\end{document}