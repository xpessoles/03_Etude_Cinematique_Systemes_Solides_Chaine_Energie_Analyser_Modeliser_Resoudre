\documentclass[10pt,oneside]{article}
\usepackage[T1]{fontenc}
\usepackage[utf8]{inputenc}
%\DeclareUnicodeCharacter{00A0}{ }
\usepackage[adobe-utopia]{mathdesign}

\usepackage{amsmath}
\usepackage[francais]{babel}
\usepackage[dvips]{graphicx}
%\usepackage{here}
\usepackage{framed}
\usepackage[normalem]{ulem}
\usepackage{fancyhdr}
\usepackage{titlesec}
\usepackage{vmargin}

\usepackage{amsmath}
\usepackage{ifthen}
\usepackage{multirow}
\usepackage{multicol} % Portions de texte en colonnes

%\usepackage{xltxtra} % Logo XeLaTeX
%\usepackage{pst-solides3d}
\usepackage{color}
%\usepackage{colortbl}
\usepackage{titletoc} % Pour la mise en forme de la table des matières

%\usepackage[crop=off]{auto-pst-pdf}
%\usepackage{bclogo}


%\usepackage{longtable}
%\usepackage{flafter}%floatants après la référence
%\usepackage{pst-solides3d}
%\usepackage{pstricks}
%\usepackage{minitoc}
%\setcounter{minitocdepth}{4}
%\usepackage{draftcopy}% "Brouillon"
%\usepackage{floatflt}
%\usepackage{psfrag}
%\usepackage{listings} % Permet d'insérer du code de programmation
%\usepackage{lmodern}
%\usepackage[adobe-utopia,uppercase=upright,greeklowercase=upright]{mathdesign}
%\usepackage{minionpro}
%\usepackage{pifont}
%\usepackage{amssymb}
%\usepackage[francais]{varioref}

\setmarginsrb{1.5cm}{1cm}{1cm}{1.5cm}{1cm}{1cm}{1cm}{1cm}

\definecolor{gris25}{gray}{0.75}
\definecolor{bleu}{RGB}{18,33,98}
\definecolor{bleuf}{RGB}{42,94,171}
\definecolor{bleuc}{RGB}{231,239,247}
\definecolor{rougef}{RGB}{185,18,27}
\definecolor{rougec}{RGB}{255,230,231}
\definecolor{vertf}{RGB}{103,126,82}
\definecolor{vertc}{RGB}{220,255,191}
\definecolor{violetf}{RGB}{112,48,160}
\definecolor{violetc}{RGB}{230,224,236}
\definecolor{jaunec}{RGB}{220,255,191}


%Si le boolen xp est vrai : compilation pour xabi
%Sinon compilation Damien
\newboolean{xp}
\setboolean{xp}{true}

\newboolean{prof}
\setboolean{prof}{false}

\def\xxtitre{\ifthenelse{\boolean{xp}}{
CI 3 -- CIN : Étude du comportement cinématique des systèmes}{
}}

\def\xxsoustitre{\ifthenelse{\boolean{xp}}{
%Chapitre 6 -- Cinématique du point immatériel dans un solide en mouvement
}{}}


\def\xxauteur{\ifthenelse{\boolean{xp}}{
\noindent 2013 -- 2014 \\
Xavier \textsc{Pessoles}}{
}}


\def\xxpied{\ifthenelse{\boolean{xp}}{
CI 3 : CIN -- Cours \\
Ch 7 : Torseurs -- TD -- \ifthenelse{\boolean{prof}}{P}{E}%
}{}}

\usepackage[%
    pdftitle={CIN : Torseurs},
    pdfauthor={Xavier Pessoles},
    colorlinks=true,
    linkcolor=blue,
    citecolor=magenta]{hyperref}


\usepackage{pifont}
\sloppy
\hyphenpenalty 10000


\begin{document}





% \makeatletter \let\ps@plain\ps@empty \makeatother
%% DEBUT DU DOCUMENT
%% =================




%------------- En tetes et Pieds de Pages ------------


\pagestyle{fancy}
\ifthenelse{\boolean{xp}}{%
\renewcommand{\headrulewidth}{0pt}}{%
\renewcommand{\headrulewidth}{0.2pt}} %pour mettre le trait en haut
%\renewcommand{\headrulewidth}{0.2pt}

\fancyhead{}
\fancyhead[L]{%
\ifthenelse{\boolean{xp}}{%
\noindent\begin{minipage}[c]{2.6cm}%
\includegraphics[width=2cm]{png/logo_ptsi.png}%
\end{minipage}%
}{%
\footnotesize{\textit{\textsf{Lycée François Premier}}}
}}

\ifthenelse{\boolean{xp}}{%
\fancyhead[C]{\rule{12cm}{.5pt}}}{
}


\fancyhead[R]{%
\noindent\begin{minipage}[c]{3cm}
\begin{flushright}
\footnotesize{\textit{\textsf{Sciences Industrielles \\ de l'ingénieur}}}%
\end{flushright}
\end{minipage}
}


\ifthenelse{\boolean{xp}}{%
\fancyhead[C]{\rule{12cm}{.5pt}}}{
}

\renewcommand{\footrulewidth}{0.2pt}

\fancyfoot[C]{\footnotesize{\bfseries \thepage}}
\fancyfoot[L]{%
\begin{minipage}[c]{.2\linewidth}
\noindent\footnotesize{{\xxauteur}}
\end{minipage}
\ifthenelse{\boolean{xp}}{}{%
\begin{minipage}[c]{.15\linewidth}
\includegraphics[width=2cm]{png/logoCC.png}
\end{minipage}}
}


\fancyfoot[R]{\footnotesize{\xxpied}}



\begin{center}
 \Large\textsc{\xxtitre}

\end{center}

\begin{center}
 \large\textsc{\xxsoustitre}
\end{center}

\begin{center}
 \large\textsc{Travail Dirigé : Robot à deux moteurs}
\end{center}

\begin{flushright}
 \textit{Ressources de Jean-Pierre Pupier.}
\end{flushright}


\subsection*{Mise en situation}
\vspace{.25cm}

Le système étudié est un robot industriel à deux moteurs destiné à la manutention de pièces lourdes. Ce robot a une structure en parallélogramme déformable qui lui permet de déplacer son poignet dans l’aire de travail.

\vspace{.25cm}

\noindent\begin{minipage}[c]{.47\linewidth}
\begin{center}
\includegraphics[height=6cm]{png/fig1}
\end{center}
\end{minipage}\hfill
\begin{minipage}[c]{.47\linewidth}
\begin{center}
\includegraphics[height=6cm]{png/fig2}
\end{center}
\end{minipage}

\vspace{.25cm}

Le schéma ci-dessus représente la cinématique simplifiée du robot à deux moteurs. L'outil faisant le travail est fixé au point $J$ et le socle est le solide \textbf{0}. La pièce \textbf{3} est rectiligne et a une longueur $CJ$. 

Au point $A$ il y a :
\begin{itemize}
\item une liaison pivot entre \textbf{0} et \textbf{1} motorisée par un moteur \textbf{M1},
\item une liaison pivot entre \textbf{0} et \textbf{2} motorisée par un moteur \textbf{M2}.
\end{itemize}

Ces deux motorisations sont, bien sûr, indépendantes.

La géométrie est telle que : $AB=EC=L$; $EA=CB=D$; $BJ=H$.

\subparagraph{}
\textit{Quel est le nom du quadrilatère $AECB$ ? Donner la propriété de ce quadrilatère.}
\ifthenelse{\boolean{prof}}{%
\begin{corrige}
\end{corrige}
}{%
}

On donne :
\begin{itemize}
\item $\mathcal{R}_0=\left(A;\vect{x_0};\vect{y_0};\vect{z_0}\right)$ repère lié au solide \textbf{0};
\item $\mathcal{R}_A=\left(A;\vect{x_A};\vect{y_A};\vect{z_0}\right)$ repère lié au solide \textbf{1};
\item $\mathcal{R}_2=\left(A;\vect{x_2};\vect{y_2};\vect{z_0}\right)$ repère lié au solide \textbf{2};
\item $\mathcal{R}_3=\left(C;\vect{x_2};\vect{y_2};\vect{z_0}\right)$ repère lié au solide \textbf{3};
\item $\mathcal{R}_4=\left(E;\vect{x_1};\vect{y_1};\vect{z_0}\right)$ repère lié au solide \textbf{4}.
\end{itemize}

\subparagraph{}
\textit{A partir de la définition des repères associés aux quatre autres solides indiquer les noms des axes manquants.}

\subsection*{Première partie : $\beta=0$ et moteur M2 arrêté}
\begin{rem}
Il est nécessaire de refaire sur votre copie le schéma du robot  dans la configuration imposée par la question.
\end{rem}

\subparagraph{}
\textit{Écrire, dans l'ordre demandé, les torseurs cinématiques des mouvements suivants aux points indiqués : il faudra les exprimer en utilisant les paramètres géométriques ou leur dérivée définis précédemment. Utiliser la base de projection donnant l’expression la plus simple :
\begin{itemize}
\item mouvement de \textbf{2} par rapport à \textbf{0}, au point $A$;
\item mouvement de \textbf{1} par rapport à \textbf{0}, au point $A$;
\item mouvement de \textbf{4} par rapport à \textbf{0}, au point $E$;
\item mouvement de \textbf{3} par rapport à \textbf{0}, au point $B$;
\item mouvement de \textbf{3} par rapport à \textbf{1}, au point $B$;
\item mouvement de \textbf{3} par rapport à \textbf{4}, au point $C$.
\end{itemize}}
\ifthenelse{\boolean{prof}}{%
\begin{corrige}
\end{corrige}
}{%
}

\subparagraph{}
\textit{Donner le nom des mouvements 1/0 et 3/0.}
\ifthenelse{\boolean{prof}}{%
\begin{corrige}
\end{corrige}
}{%
}


\subparagraph{}
\textit{Déterminer $\vectv{J}{3}{0}$.}
\ifthenelse{\boolean{prof}}{%
\begin{corrige}
\end{corrige}
}{%
}


\subparagraph{}
\textit{Définir et tracer la trajectoire $T_{J\in3/0}$.}
% Reproduire pour cela les éléments principaux du schéma sur votre copie.}
\ifthenelse{\boolean{prof}}{%
\begin{corrige}
\end{corrige}
}{%
}


\subsection*{Deuxième partie : $\alpha=60^o$ et moteur M1 arrêté}

\begin{rem}
Il est nécessaire de refaire sur votre copie le schéma du robot  dans la configuration imposée par la question.
\end{rem}


\subparagraph{}
\textit{Écrire, dans l'ordre demandé, les torseurs cinématiques des mouvements suivants aux points indiqués :
\begin{itemize}
\item mouvement de \textbf{1} par rapport à \textbf{0}, au point $A$;
\item mouvement de \textbf{2} par rapport à \textbf{0}, au point $A$;
\item mouvement de \textbf{4} par rapport à \textbf{0}, au point $E$;
\item mouvement de \textbf{3} par rapport à \textbf{0}, au point $B$;
\item mouvement de \textbf{3} par rapport à \textbf{1}, au point $B$;
\item mouvement de \textbf{4} par rapport à \textbf{2}, au point $E$;.
\end{itemize}}
\ifthenelse{\boolean{prof}}{%
\begin{corrige}
\end{corrige}
}{%
}

\subparagraph{}
\textit{Déterminer $\vectv{J}{3}{0}$.}
\ifthenelse{\boolean{prof}}{%
\begin{corrige}
\end{corrige}
}{%
}

\subparagraph{}
\textit{Définir et tracer la trajectoire $T_{J\in3/0}$.}
% Reproduire pour cela les éléments principaux du schéma sur votre copie.}
\ifthenelse{\boolean{prof}}{%
\begin{corrige}
\end{corrige}
}{%
}

\subsection*{Troisième partie : les deux moteurs fonctionnent}
\subparagraph{}
\textit{Déterminer $\vectv{J}{3}{0}$.}
\ifthenelse{\boolean{prof}}{%
\begin{corrige}
\end{corrige}
}{%
}

On donne : $ L=70\;mm$; $D=32\;mm$ ; $H=59\;mm$.

\subparagraph{}
\textit{Tracer sur une figure à l'échelle 1 la surface liée à 0 dans laquelle se déplace le point $J$ lorsque $\alpha$ varie de $60^o$ à $60^o$ et $\beta$ de $-45^o$ à $45^o$. Cette surface sera la surface de travail du robot.}

\ifthenelse{\boolean{prof}}{%
\begin{corrige}
\end{corrige}
}{%
}





\end{document}
