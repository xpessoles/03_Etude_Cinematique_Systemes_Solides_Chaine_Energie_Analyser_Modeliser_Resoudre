\documentclass[11pt,oneside]{article}
\usepackage[T1]{fontenc}
\usepackage[utf8]{inputenc}
%\DeclareUnicodeCharacter{00A0}{ }
\usepackage[adobe-utopia]{mathdesign}

\usepackage{amsmath}
\usepackage[francais]{babel}
\usepackage[dvips]{graphicx}
%\usepackage{here}
\usepackage{framed}
\usepackage[normalem]{ulem}
\usepackage{fancyhdr}
\usepackage{titlesec}
\usepackage{vmargin}

\usepackage{amsmath}
\usepackage{ifthen}
\usepackage{multirow}
\usepackage{multicol} % Portions de texte en colonnes

%\usepackage{xltxtra} % Logo XeLaTeX
%\usepackage{pst-solides3d}
\usepackage{color}
%\usepackage{colortbl}
\usepackage{titletoc} % Pour la mise en forme de la table des matières

%\usepackage[crop=off]{auto-pst-pdf}
%\usepackage{bclogo}


%\usepackage{longtable}
%\usepackage{flafter}%floatants après la référence
%\usepackage{pst-solides3d}
%\usepackage{pstricks}
%\usepackage{minitoc}
%\setcounter{minitocdepth}{4}
%\usepackage{draftcopy}% "Brouillon"
%\usepackage{floatflt}
%\usepackage{psfrag}
%\usepackage{listings} % Permet d'insérer du code de programmation
%\usepackage{lmodern}
%\usepackage[adobe-utopia,uppercase=upright,greeklowercase=upright]{mathdesign}
%\usepackage{minionpro}
%\usepackage{pifont}
%\usepackage{amssymb}
%\usepackage[francais]{varioref}

\setmarginsrb{1.5cm}{1cm}{1cm}{1.5cm}{1cm}{1cm}{1cm}{1cm}

\definecolor{gris25}{gray}{0.75}
\definecolor{bleu}{RGB}{18,33,98}
\definecolor{bleuf}{RGB}{42,94,171}
\definecolor{bleuc}{RGB}{231,239,247}
\definecolor{rougef}{RGB}{185,18,27}
\definecolor{rougec}{RGB}{255,230,231}
\definecolor{vertf}{RGB}{103,126,82}
\definecolor{vertc}{RGB}{220,255,191}
\definecolor{violetf}{RGB}{112,48,160}
\definecolor{violetc}{RGB}{230,224,236}
\definecolor{jaunec}{RGB}{220,255,191}

\usepackage[%
    pdftitle={CIN -- Géométrie vectorielle -- Applications},
    pdfauthor={Xavier Pessoles},
    colorlinks=true,
    linkcolor=blue,
    citecolor=magenta]{hyperref}



% \makeatletter \let\ps@plain\ps@empty \makeatother
%% DEBUT DU DOCUMENT
%% =================
\sloppy
\hyphenpenalty 10000

\newcommand{\Pointilles}[1][3]{%
\multido{}{#1}{\makebox[\linewidth]{\dotfill}\\[\parskip]
}}


\begin{document}


\newboolean{prof}
\setboolean{prof}{false}
%------------- En tetes et Pieds de Pages ------------
\pagestyle{fancy}
\renewcommand{\headrulewidth}{0pt}

\fancyhead{}
\fancyhead[L]{%
\noindent\noindent\begin{minipage}[c]{2.6cm}
%Lycée Rouvière PTSI
\includegraphics[width=2cm]{png/logo_ptsi.png}%
\end{minipage}
}

\fancyhead[C]{\rule{12cm}{.5pt}}

\fancyhead[R]{%
\begin{minipage}[c]{3cm}
\begin{flushright}
\footnotesize{\textit{\textsf{Sciences Industrielles\\ de l'Ingénieur}}}%
\end{flushright}
\end{minipage}
}

\renewcommand{\footrulewidth}{0.2pt}

\fancyfoot[C]{\footnotesize{\bfseries \thepage}}
\fancyfoot[L]{\footnotesize{2013 -- 2014} \\ X. \textsc{Pessoles} -- TD de S. Genouël}
\ifthenelse{\boolean{prof}}{%
\fancyfoot[R]{\footnotesize{CI 3 : CIN -- Applications} \\ \footnotesize{Ch. 4 : Entrée -- Sortie -- P}}
}{%
\fancyfoot[R]{\footnotesize{CI 3 : CIN -- Applications} \\ \footnotesize{Ch. 4 : Lois Entrée -- Sortie -- E}}
}


%\begin{center}
%\textit{Centre d'intérêt}
%\end{center}



\begin{center}
 \Large\textsc{CI 3 -- CIN : Étude du comportement cinématique des systèmes}
\end{center}

\begin{center}
 \large\textsc{Chapitre 4 -- Étude des chaînes fermées :
Détremination des lois Entrées -- Sorties}
\end{center}

\begin{center}
\textsc{Exercices d'application supplémentaires} 
\end{center}
\begin{flushright}
\textit{D'après ressources de Stéphane Genouël \cite{SG}.} 
\end{flushright}
\vspace{.5cm}

\subsection*{Exercice 1 -- Pompe hydraulique à pistons radiaux}
 
On s’intéresse au comportement cinématique du dispositif de transformation de mouvement par excentrique 
qui permet de transformer le mouvement de rotation continu de l’arbre d’entrée, sur lequel est fixé 
l’excentrique 1, en mouvement de translation alternative du piston 2. 

\begin{center}
\includegraphics[height=5cm]{png/fig1} 
\includegraphics[height=5cm]{png/fig2} 
\end{center}




\subparagraph{}
\textit{Donner le graphe de liaison de ce système.}

\ifthenelse{\boolean{prof}}{
\begin{corrige}
\end{corrige}
}{}

\subparagraph{}
\textit{Donner les caractéristiques, le paramètre d’entrée et le paramètre de sortie du système. }

\ifthenelse{\boolean{prof}}{
\begin{corrige}
\end{corrige}
}{}


\subparagraph{}
\textit{Déterminer la loi E/S en position du système à l’aide d’une fermeture géométrique. }

\ifthenelse{\boolean{prof}}{
\begin{corrige}
\end{corrige}
}{}


\subparagraph{}
\textit{En déduire la vitesse du piston par rapport au cylindre (c'est-à-dire la loi E/S en vitesse). }

\ifthenelse{\boolean{prof}}{
\begin{corrige}
\end{corrige}
}{}


\subsection*{Exercice 2 -- Système d'orientation d'antenne}
\setcounter{subparagraph}{0}

\begin{minipage}[c]{.65\linewidth}
Le système d’orientation d’antenne ci-contre permet, grâce à une télécommande, de régler à distance l’orientation de sa parabole afin d’optimiser la réception des chaines de télévision. 
 
Pour cela, le vérin électrique est alimenté en énergie électrique par le préactionneur, de façon à faire rentrer ou sortir la tige et obtenir ainsi la position de l’antenne désirée. 
\end{minipage} \hfill
\begin{minipage}[c]{.3\linewidth}
\begin{center}
\includegraphics[width=.95\textwidth]{png/fig3} 
\end{center}
\end{minipage} 


\begin{center}
\includegraphics[height=5cm]{png/fig4} 
\includegraphics[height=5cm]{png/fig5} 
\end{center}

\begin{obj}
Déterminer la durée d’alimentation en énergie électrique du système d’orientation pour un 
changement de position de l’antenne donné. 

\end{obj}



\begin{minipage}[c]{.65\linewidth}
Une représentation 2D du système d’orientation d’antenne est donnée ci-dessous :
\begin{itemize}
\item $\vect{AC}=L_0\vect{x_0}$;
\item $\vect{AB}=L_1\vect{x_1}$;
\item $\alpha_1(t)$ : paramètre de mouvement de l’antenne 1 par rapport au support 0; 
\item $\alpha_2(t)$ : paramètre de mouvement du corps 2 par rapport au support 0;
\item $d(t)$ : paramètre de mouvement de la tige 3 par rapport au corps 2. 
\end{itemize}
\end{minipage} \hfill
\begin{minipage}[c]{.3\linewidth}
\begin{center}
\includegraphics[width=.95\textwidth]{png/fig6} 
\end{center}
\end{minipage} 


\subparagraph{}
\textit{Réaliser, en s’inspirant de la figure ci-dessus, le schéma cinématique du système d’orientation d’antenne dans le plan 
$\left(O,\vect{x_0},\vect{y_0}\right)$. Paramétrer ce schéma cinématique. }

\ifthenelse{\boolean{prof}}{
\begin{corrige}
\end{corrige}
}{}

\subparagraph{}
\textit{Donner le graphe de liaison de ce système.}

\ifthenelse{\boolean{prof}}{
\begin{corrige}
\end{corrige}
}{}

\subparagraph{}
\textit{Donner les caractéristiques, le paramètre d’entrée et le paramètre de sortie du système.}

\ifthenelse{\boolean{prof}}{
\begin{corrige}
\end{corrige}
}{}

\subparagraph{}
\textit{Déterminer la loi E/S en position du système à l’aide d’une fermeture géométrique. }

\ifthenelse{\boolean{prof}}{
\begin{corrige}
\end{corrige}
}{}


\vspace{.5cm}

\begin{minipage}[c]{.45\linewidth}
Le vérin électrique utilisé est constitué : 
\begin{itemize}
\item d’un moteur électrique ;
\item d’un réducteur à engrenage (rapport de réduction $k=1/5$);
\item d’un dispositif de transformation de mouvement de type vis-
écrou (pas $p=2$). 
\end{itemize}
\end{minipage} \hfill
\begin{minipage}[c]{.5\linewidth}
\begin{center}
\includegraphics[width=.95\textwidth]{png/fig7} 
\end{center}
\end{minipage} 

\begin{center}
\includegraphics[width=.8\textwidth]{png/fig8} 
\end{center}

On suppose que le moteur électrique tourne à la vitesse constante de $6000\; tr/min$. 


\subparagraph{}
\textit{Déterminer la vitesse de sortie de la tige par rapport au corps.}

\ifthenelse{\boolean{prof}}{
\begin{corrige}
\end{corrige}
}{}


On souhaite faire passer l’antenne 1 d’une position initiale ($\alpha_1 = 58\textdegree$) à une position finale ($\alpha_2 = 82\textdegree$) .

\subparagraph{}
\textit{Déterminer, à l’aide de la courbe de la loi entrée-sortie donnée ci-dessous, la durée d’alimentation du vérin électrique permettant ce changement de position. }

\ifthenelse{\boolean{prof}}{
\begin{corrige}
\end{corrige}
}{}

\begin{center}
\includegraphics[width=.8\textwidth]{png/fig9} 
\end{center}


\begin{thebibliography}{2}
\bibitem[1]{SG} Stéphane Genouël, Cours TP et TD en MPSI et PCSI -- Lycée Chateaubriand Rennes \url{http://stephane.genouel.free.fr/}.
\end{thebibliography}

\end{document}