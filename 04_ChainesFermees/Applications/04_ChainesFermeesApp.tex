\documentclass[11pt,oneside]{article}
\usepackage[T1]{fontenc}
\usepackage[utf8]{inputenc}
%\DeclareUnicodeCharacter{00A0}{ }
\usepackage[adobe-utopia]{mathdesign}

\usepackage{amsmath}
\usepackage[francais]{babel}
\usepackage[dvips]{graphicx}
%\usepackage{here}
\usepackage{framed}
\usepackage[normalem]{ulem}
\usepackage{fancyhdr}
\usepackage{titlesec}
\usepackage{vmargin}

\usepackage{amsmath}
\usepackage{ifthen}
\usepackage{multirow}
\usepackage{multicol} % Portions de texte en colonnes

%\usepackage{xltxtra} % Logo XeLaTeX
%\usepackage{pst-solides3d}
\usepackage{color}
%\usepackage{colortbl}
\usepackage{titletoc} % Pour la mise en forme de la table des matières

%\usepackage[crop=off]{auto-pst-pdf}
%\usepackage{bclogo}


%\usepackage{longtable}
%\usepackage{flafter}%floatants après la référence
%\usepackage{pst-solides3d}
%\usepackage{pstricks}
%\usepackage{minitoc}
%\setcounter{minitocdepth}{4}
%\usepackage{draftcopy}% "Brouillon"
%\usepackage{floatflt}
%\usepackage{psfrag}
%\usepackage{listings} % Permet d'insérer du code de programmation
%\usepackage{lmodern}
%\usepackage[adobe-utopia,uppercase=upright,greeklowercase=upright]{mathdesign}
%\usepackage{minionpro}
%\usepackage{pifont}
%\usepackage{amssymb}
%\usepackage[francais]{varioref}

\setmarginsrb{1.5cm}{1cm}{1cm}{1.5cm}{1cm}{1cm}{1cm}{1cm}

\definecolor{gris25}{gray}{0.75}
\definecolor{bleu}{RGB}{18,33,98}
\definecolor{bleuf}{RGB}{42,94,171}
\definecolor{bleuc}{RGB}{231,239,247}
\definecolor{rougef}{RGB}{185,18,27}
\definecolor{rougec}{RGB}{255,230,231}
\definecolor{vertf}{RGB}{103,126,82}
\definecolor{vertc}{RGB}{220,255,191}
\definecolor{violetf}{RGB}{112,48,160}
\definecolor{violetc}{RGB}{230,224,236}
\definecolor{jaunec}{RGB}{220,255,191}

\usepackage[%
    pdftitle={CIN -- Géométrie vectorielle -- Applications},
    pdfauthor={Xavier Pessoles},
    colorlinks=true,
    linkcolor=blue,
    citecolor=magenta]{hyperref}



% \makeatletter \let\ps@plain\ps@empty \makeatother
%% DEBUT DU DOCUMENT
%% =================
\sloppy
\hyphenpenalty 10000

\newcommand{\Pointilles}[1][3]{%
\multido{}{#1}{\makebox[\linewidth]{\dotfill}\\[\parskip]
}}


\begin{document}


\newboolean{prof}
\setboolean{prof}{false}
%------------- En tetes et Pieds de Pages ------------
\pagestyle{fancy}
\renewcommand{\headrulewidth}{0pt}

\fancyhead{}
\fancyhead[L]{%
\noindent\noindent\begin{minipage}[c]{2.6cm}
%Lycée Rouvière PTSI
\includegraphics[width=2cm]{png/logo_ptsi.png}%
\end{minipage}
}

\fancyhead[C]{\rule{12cm}{.5pt}}

\fancyhead[R]{%
\begin{minipage}[c]{3cm}
\begin{flushright}
\footnotesize{\textit{\textsf{Sciences Industrielles\\ de l'Ingénieur}}}%
\end{flushright}
\end{minipage}
}

\renewcommand{\footrulewidth}{0.2pt}

\fancyfoot[C]{\footnotesize{\bfseries \thepage}}
\fancyfoot[L]{\footnotesize{2013 -- 2014} \\ X. \textsc{Pessoles}}
\ifthenelse{\boolean{prof}}{%
\fancyfoot[R]{\footnotesize{CI 3 : CIN -- Applications} \\ \footnotesize{Ch. 4 : Entrée -- Sortie -- P}}
}{%
\fancyfoot[R]{\footnotesize{CI 3 : CIN -- Applications} \\ \footnotesize{Ch. 4 : Lois Entrée -- Sortie -- E}}
}


%\begin{center}
%\textit{Centre d'intérêt}
%\end{center}



\begin{center}
 \Large\textsc{CI 3 -- CIN : Étude du comportement cinématique des systèmes}
\end{center}

\begin{center}
 \large\textsc{Chapitre 4 -- Étude des chaînes fermées :
Détermination des lois Entrées -- Sorties}
\end{center}

\begin{center}
\textsc{Exercices d'application} 
\end{center}
\begin{flushright}
\textit{D'après ressources de Jean-Pierre Pupier.} 
\end{flushright}
\vspace{.5cm}

\subsection*{Exercice 1 -- Poussoir et roulette}
 
\begin{center}
\includegraphics[width=.6\textwidth]{png/fig1_1b} 
\end{center}
Fonctionnement : La rotation de 1 (entrée) fait mouvoir 3 (sortie).

\subparagraph{}
\textit{Définir les différents repères liés aux solides. Dessiner les axes sur le schéma.}

\ifthenelse{\boolean{prof}}{
\begin{corrige}
\begin{center}
\includegraphics[width=.6\textwidth]{png/fig1_1c} 
\end{center}
On a les repères suivants :
\begin{itemize}
\item $\mathcal{R}_0=\left(B,\vect{x_0},\vect{y_0},\vect{z_0} \right)$ lié à 0;
\item $\mathcal{R}_1=\left(B,\vect{x_1},\vect{y_1},\vect{z_0} \right)$ lié à 1;
\item $\mathcal{R}_2=\left(A,\vect{x_2},\vect{y_2},\vect{z_0}\right)$ lié à 2;
\item $\mathcal{R}_3=\left(E,\vect{x_0},\vect{y_0},\vect{z_0} \right)$ lié à 3.
\end{itemize}
\end{corrige}
}{}

\subparagraph{}
\textit{Réaliser le paramétrage géométrique de ce mécanisme (tous les paramètres). Préciser si les paramètres sont variables ou constants. Indiquez les paramètres sur le schéma.}

\ifthenelse{\boolean{prof}}{
\begin{corrige}
Il est possible de définir les paramètres géométriques constants suivants : 
\begin{itemize}
\item $\vect{BA} = R\vect{x_1}$;
\item $\vect{AC} = r\vect{x_2}$ et $\vect{DA} = r\vect{y_0}$;
\item $\vect{BF} = a\vect{x_0}-b\vect{y_0}$.
\end{itemize}

Les paramètres angulaires permettant le changements de repères sont donnés par les figures suivantes: 
\begin{center}
\includegraphics[width=.6\textwidth]{png/fig1_2c} 
\end{center}

On peut par ailleurs définir $\lambda$ tel que $\vect{FE} = \lambda \vect{y_0}$ ou encore $\mu_1(t)$ et $\mu_2(t)$ tels que $\vect{BD} = \mu_1(t) \vect{x_0}+\mu_2(t) \vect{y_0}$
\end{corrige}
}{}

\subparagraph{}
\textit{Trouver la loi entrée sortie.}

\ifthenelse{\boolean{prof}}{
\begin{corrige}
On peut par exemple écrire la fermeture de chaîne vectorielle suivante :
$$
\vect{BA}+\vect{AD}+\vect{DB}=\vect{0} 
$$

En projetant cette loi sur $\vect{y_0}$ on obtient : 
$$
\vect{BA}\cdot\vect{y_0}+\vect{AD}\cdot\vect{y_0}+\vect{DB}\cdot\vect{y_0}=0
\Longleftrightarrow
R\vect{x_1}\cdot\vect{y_0} - r\vect{y_0}\cdot\vect{y_0}- \mu_1(t) \vect{x_0}\cdot\vect{y_0}-\mu_2(t) \vect{y_0}\cdot\vect{y_0} = 0
$$
$$
\Longleftrightarrow
R\cos\left( \dfrac{\pi}{2}-\alpha(t)\right) - r-\mu_2(t)=0
$$
On a donc $R\sin\alpha(t) - r-\mu_2(t)=0$ soit :
$$\mu_2(t)=R\sin\alpha(t) - r$$
\end{corrige}
}{}


\subparagraph{}
\textit{Calculer l'expression de la vitesse de 3 dans 0 en fonction de la vitesse angulaire de 1 dans 0 et de certains paramètres constants.}

\ifthenelse{\boolean{prof}}{
\begin{corrige}
En dérivant l'expression précédente, on a donc 
$$\dot{\mu_2(t)}{dt}=\dot{\alpha(t)}R\cos\alpha(t) $$
\end{corrige}
}{}


\subsection*{Exercice 2 -- Mécanisme pour mouvement alternatif}
\setcounter{subparagraph}{0}
 
\begin{center}
\includegraphics[width=.7\textwidth]{png/fig2_1} 
\end{center}

Le mécanisme ci-dessus est cinématiquement plan. La rotation de l'arbre d'entrée 1 permet d'imprimer un mouvement de translation alternatif à l'arbre de sortie 4.

\begin{rem}
Soignez l'écriture (les indices doivent être parfaitement lisibles).
\end{rem}

\subparagraph{}
\textit{Définir les différents repères liés aux sous-ensembles cinématiques. Indiquer les autres origines possibles pour chaque repère.}


\ifthenelse{\boolean{prof}}{
\begin{corrige}
On a les repères suivants :
\begin{itemize}
\item $\mathcal{R}_0=\left(A,\vect{x_0},\vect{y_0},\vect{z_0} \right)$ lié à 0;
\item $\mathcal{R}_1=\left(A,\vect{x_1},\vect{y_1},\vect{z_0} \right)$ lié à 1 (autre origine possible : $B$);
\item $\mathcal{R}_2=\left(C,\vect{x_2},\vect{y_2},\vect{z_0}\right)$ lié à 2;
\item $\mathcal{R}_3=\left(E,\vect{x_2},\vect{y_2},\vect{z_0}\right)$ lié à 3 (autre origine possible : $D$);
\item $\mathcal{R}_4=\left(E,\vect{x_0},\vect{y_0},\vect{z_0} \right)$ lié à 4.
\end{itemize}
\end{corrige}
}{}


\subparagraph{}
\textit{Compléter le schéma ci-dessus en indiquant les divers axes utiles des repères.}


\ifthenelse{\boolean{prof}}{
\begin{corrige}
\end{corrige}
}{}


\subparagraph{}
\textit{Effectuer le paramétrage de ce mécanisme (les paramètres intermédiaires non utiles pour trouver la loi entrée-sortie ne doivent pas apparaître). On donne : $AB = r$ , $AC = a$, $\vect{CF}\cdot\vect{x_0}=b$. La position du point D sur CE n'a aucune importance ; il ne faut pas la faire intervenir dans les calculs. Il en est de même de l'altitude du point $F$.}


\ifthenelse{\boolean{prof}}{
\begin{corrige}
\end{corrige}
}{}


\subparagraph{}
\textit{Indiquer les paramètres variables et les paramètres constants sur le schéma.}


\ifthenelse{\boolean{prof}}{
\begin{corrige}
\end{corrige}
}{}


\subparagraph{}
\textit{Trouver la loi entrée-sortie. }


\ifthenelse{\boolean{prof}}{
\begin{corrige}
\end{corrige}
}{}


\subparagraph{}
\textit{Trouver l'expression de la valeur du paramètre d'entrée pour laquelle le point $E$ est au maximum en bas en utilisant une méthode mathématique puis en utilisant une méthode géométrique (plus intuitive). Faites un dessin pour la deuxième réponse.}


\ifthenelse{\boolean{prof}}{
\begin{corrige}
\end{corrige}
}{}


\subsection*{Exercice 3 -- Joint de Cardan}
\setcounter{subparagraph}{0}


Un joint de Cardan est un accouplement qui permet de transmettre un mouvement de rotation entre deux arbres concourants mais non alignés. L'angle maximum pratiquement utilisé entre les arbres est de $45\textdegree$. Une application courante est la transmission entre boite de vitesses  et roues-avant d’une voiture. 

Les vues ci-dessous donnent des images d’un joint de cardan.

\begin{center}
\begin{tabular}{ccc}
\includegraphics[width=.3\textwidth]{png/fig3_1} & 
\includegraphics[width=.3\textwidth]{png/fig3_2} & 
\includegraphics[width=.3\textwidth]{png/fig3_3} 
\end{tabular}
\end{center}

La modélisation suivante est proposée.
\begin{center}
\includegraphics[width=.7\textwidth]{png/fig3_4} 
\end{center}

On appelle : 
\begin{itemize}
\item $\mathcal{R}$ le repère lié au solide $R$ considéré comme fixe. $\mathcal{R}=\left(O,\vect{x},\vect{y},\vect{z} \right)$;
\item $\mathcal{R}'$ le repère lié au solide R considéré comme fixe. $\mathcal{R}'=\left(O,\vect{u},\vect{v},\vect{z} \right)$. On pose $\alpha = \left(\vect{y},\vect{v} \right)$ (constant);
\item $\alpha$ l'"angle de brisure";
\item $\mathcal{R}_1$ le repère lié au solide 1. $\mathcal{R}_1 = \left(O,\vect{x_1},\vect{y},\vect{z_1} \right)$. On pose  $\theta_1 = \left(\vect{x},\vect{x_1} \right)$;
\item $\mathcal{R}_3$ le repère lié au solide 3. $\mathcal{R}_3 = \left(O,\vect{x_3},\vect{v},\vect{z_3} \right)$. On pose $\theta_3 = \left(\vect{u},\vect{x_3} \right)$.
\end{itemize}

\subparagraph{}
\textit{Tracer en vue orthogonale, les trois dessins (figures de changement de base) permettant le passage de $\mathcal{R}$ à $\mathcal{R}_1$ , de $\mathcal{R}$ à $\mathcal{R}'$ et de $\mathcal{R}'$ à $\mathcal{R}_3$.}
\ifthenelse{\boolean{prof}}{
\begin{corrige}
\end{corrige}
}{}

\subparagraph{}
\textit{Exprimer la condition géométrique sur 2 permettant de lier $\mathcal{R}_1$ à $\mathcal{R}_3$.}
\ifthenelse{\boolean{prof}}{
\begin{corrige}
\end{corrige}
}{}

\subparagraph{}
\textit{Développer cette relation et trouver la loi entrée sortie : $\theta_3 = f(\theta_1 , \alpha)$. Tracer, pour $\alpha=45\textdegree$, la courbe représentant l’évolution de la sortie $\theta_3$ en fonction de l’entrée $\theta_1$ avec $\theta_1$ variant de $-\pi$ à $+\pi$.}
\ifthenelse{\boolean{prof}}{
\begin{corrige}
\end{corrige}
}{}

\subparagraph{}
\textit{Dériver cette relation par rapport au temps pour trouver la vitesse de sortie $\dot{\theta_3}$ en fonction de la vitesse d’entrée $\dot{\theta_1}$, de $\theta_1$ et de $\alpha$.}

\ifthenelse{\boolean{prof}}{
\begin{corrige}
\end{corrige}
}{}

\subparagraph{}
\textit{Tracer l’évolution de la vitesse de sortie $\dot{\theta_3}$ en fonction notamment de l’évolution de l’angle d’entrée $\theta_1$. On prendra un angle de brisure de $45\textdegree$ et une vitesse d’entée constante $\dot{\theta_1}$ de 1 rad/s.}
\ifthenelse{\boolean{prof}}{
\begin{corrige}
\end{corrige}
}{}

\subparagraph{}
\textit{Conclure sur une des propriétés de ce mécanisme.}
\ifthenelse{\boolean{prof}}{
\begin{corrige}
\end{corrige}
}{}



\end{document}
