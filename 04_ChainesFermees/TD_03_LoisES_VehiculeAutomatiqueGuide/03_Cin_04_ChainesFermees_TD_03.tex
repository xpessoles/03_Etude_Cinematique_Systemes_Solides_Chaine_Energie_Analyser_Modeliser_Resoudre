\documentclass[10pt]{article}
\usepackage[T1]{fontenc}
\usepackage[utf8]{inputenc}
%\DeclareUnicodeCharacter{00A0}{ }
\usepackage[adobe-utopia]{mathdesign}

\usepackage{amsmath}
\usepackage[francais]{babel}
\usepackage[dvips]{graphicx}
%\usepackage{here}
\usepackage{framed}
\usepackage[normalem]{ulem}
\usepackage{fancyhdr}
\usepackage{titlesec}
\usepackage{vmargin}

\usepackage{amsmath}
\usepackage{ifthen}
\usepackage{multirow}
\usepackage{multicol} % Portions de texte en colonnes

%\usepackage{xltxtra} % Logo XeLaTeX
%\usepackage{pst-solides3d}
\usepackage{color}
%\usepackage{colortbl}
\usepackage{titletoc} % Pour la mise en forme de la table des matières

%\usepackage[crop=off]{auto-pst-pdf}
%\usepackage{bclogo}


%\usepackage{longtable}
%\usepackage{flafter}%floatants après la référence
%\usepackage{pst-solides3d}
%\usepackage{pstricks}
%\usepackage{minitoc}
%\setcounter{minitocdepth}{4}
%\usepackage{draftcopy}% "Brouillon"
%\usepackage{floatflt}
%\usepackage{psfrag}
%\usepackage{listings} % Permet d'insérer du code de programmation
%\usepackage{lmodern}
%\usepackage[adobe-utopia,uppercase=upright,greeklowercase=upright]{mathdesign}
%\usepackage{minionpro}
%\usepackage{pifont}
%\usepackage{amssymb}
%\usepackage[francais]{varioref}

\setmarginsrb{1.5cm}{1cm}{1cm}{1.5cm}{1cm}{1cm}{1cm}{1cm}

\definecolor{gris25}{gray}{0.75}
\definecolor{bleu}{RGB}{18,33,98}
\definecolor{bleuf}{RGB}{42,94,171}
\definecolor{bleuc}{RGB}{231,239,247}
\definecolor{rougef}{RGB}{185,18,27}
\definecolor{rougec}{RGB}{255,230,231}
\definecolor{vertf}{RGB}{103,126,82}
\definecolor{vertc}{RGB}{220,255,191}
\definecolor{violetf}{RGB}{112,48,160}
\definecolor{violetc}{RGB}{230,224,236}
\definecolor{jaunec}{RGB}{220,255,191}
%\usepackage{algorithm}
%\usepackage{algorithmic}
\usepackage[french]{algorithm2e}

\SetKwBlock{Fonction}{Début Fonction}{Fin Fonction}
\SetKwComment{Comment}{start}{end}
% Python sources

\usepackage{listings}
\lstloadlanguages{R}   % pour regler les pb d accent utf8 dans les codes
\lstset{language=R} % pour regler les pb d accent utf8 dans les codes

\usepackage{textcomp}
\usepackage{setspace}
%\usepackage{palatino}

%\usepackage{color}
\definecolor{Bleu}{rgb}{0.1,0.1,1.0}
\definecolor{Noir}{rgb}{0,0,0}
\definecolor{Grau}{rgb}{0.5,0.5,0.5}
\definecolor{DunkelGrau}{rgb}{0.15,0.15,0.15}
\definecolor{Hellbraun}{rgb}{0.5,0.25,0.0}
\definecolor{Magenta}{rgb}{1.0,0.0,1.0}
\definecolor{Gris}{gray}{0.5}
\definecolor{Vert}{rgb}{0,0.5,0}
\definecolor{SourceHintergrund}{rgb}{1,1.0,0.95}


%
\renewcommand{\lstlistlistingname}{Listings}
\renewcommand{\lstlistingname}{Listing}

\lstnewenvironment{python}[1][]{
\lstset{
%escapeinside={\%*}{*)},
%inputencoding=utf8,   % pour regler les pb d accent utf8 dans les codes
%extendedchars=true,   % pour regler les pb d accent utf8 dans les codes
language=python,
basicstyle=\sffamily\footnotesize, 	
stringstyle=\color{red}, 
showstringspaces=false, 
alsoletter={1234567890},
otherkeywords={\ , \}, \{},
keywordstyle=\color{blue},
emph={access,and,break,class,continue,def,del,elif ,else,
except,exec,finally,for,from,global,if,import,in,i s,
lambda,not,or,pass,print,raise,return,try,while},
emphstyle=\color{black}\bfseries,
emph={[2]True, False, None, self},
emphstyle=[2]\color{olive},
emph={[3]from, import, as},
emphstyle=[3]\color{blue},
upquote=true,
columns=flexible, % pour empecher d'avoir un espacement mono
morecomment=[s]{"""}{"""},
commentstyle=\color{Hellbraun}\slshape, 
%emph={[4]1, 2, 3, 4, 5, 6, 7, 8, 9, 0},
emphstyle=[4]\color{blue},
literate=*{:}{{\textcolor{blue}:}}{1}
{=}{{\textcolor{blue}=}}{1}
{-}{{\textcolor{blue}-}}{1}
{+}{{\textcolor{blue}+}}{1}
{*}{{\textcolor{blue}*}}{1}
{!}{{\textcolor{blue}!}}{1}
{(}{{\textcolor{blue}(}}{1}
{)}{{\textcolor{blue})}}{1}
{[}{{\textcolor{blue}[}}{1}
{]}{{\textcolor{blue}]}}{1}
{<}{{\textcolor{blue}<}}{1}
{>}{{\textcolor{blue}>}}{1}
{COMPLETER}{{\textcolor{red}COMPLETER}}{1},
literate=%
            {é}{{\'{e}}}1
            {è}{{\`{e}}}1
            {ê}{{\^{e}}}1
            {ë}{{\¨{e}}}1
            {û}{{\^{u}}}1
            {ù}{{\`{u}}}1
            {â}{{\^{a}}}1
            {à}{{\`{a}}}1
            {î}{{\^{i}}}1
            {ç}{{\c{c}}}1
            {Ç}{{\c{C}}}1
            {É}{{\'{E}}}1
            {Ê}{{\^{E}}}1
            {À}{{\`{A}}}1
            {Â}{{\^{A}}}1
            {Î}{{\^{I}}}1, % pour regler les pb d accent utf8 dans les codes
%framexleftmargin=1mm, framextopmargin=1mm, frame=shadowbox, rulesepcolor=\color{blue},#1
%backgroundcolor=\color{SourceHintergrund}, 
%framexleftmargin=1mm, framexrightmargin=1mm, framextopmargin=1mm, frame=single, framerule=1pt, rulecolor=\color{black},#1
}}{}



\lstnewenvironment{scilab}[1][]{
\lstset{
language=scilab,
basicstyle=\sffamily\footnotesize, 	
stringstyle=\color{red}, 
showstringspaces=false, 
alsoletter={1234567890},
otherkeywords={\ , \}, \{},
keywordstyle=\color{blue},
emph={access,and,break,class,continue,def,del,elif ,else,
except,exec,finally,for,from,global,if,import,in,i s,
lambda,not,or,pass,print,raise,return,try,while,Debut},
emphstyle=\color{black}\bfseries,
emph={[2]True, False, None, self},
emphstyle=[2]\color{olive},
emph={[3]from, import, as},
emphstyle=[3]\color{blue},
upquote=true,
columns=flexible, % pour empecher d'avoir un espacement mono
morecomment=[s]{"""}{"""},
commentstyle=\color{Hellbraun}\slshape, 
%emph={[4]1, 2, 3, 4, 5, 6, 7, 8, 9, 0},
emphstyle=[4]\color{blue},
literate=*{:}{{\textcolor{blue}:}}{1}
{=}{{\textcolor{blue}=}}{1}
{-}{{\textcolor{blue}-}}{1}
{+}{{\textcolor{blue}+}}{1}
{*}{{\textcolor{blue}*}}{1}
{!}{{\textcolor{blue}!}}{1}
{(}{{\textcolor{blue}(}}{1}
{)}{{\textcolor{blue})}}{1}
{[}{{\textcolor{blue}[}}{1}
{]}{{\textcolor{blue}]}}{1}
{<}{{\textcolor{blue}<}}{1}
{>}{{\textcolor{blue}>}}{1},
%framexleftmargin=1mm, framextopmargin=1mm, frame=shadowbox, rulesepcolor=\color{blue},#1
%backgroundcolor=\color{SourceHintergrund}, 
%framexleftmargin=1mm, framexrightmargin=1mm, framextopmargin=1mm, frame=single, framerule=1pt, rulecolor=\color{black},#1
}}{}


\lstdefinestyle{stylepython}{%
escapeinside={\%*}{*)},
inputencoding=utf8,   % pour regler les pb d accent utf8 dans les codes
extendedchars=true,   % pour regler les pb d accent utf8 dans les codes
language=python,
basicstyle=\sffamily\footnotesize, 	
stringstyle=\color{red}, 
showstringspaces=false, 
alsoletter={1234567890},
otherkeywords={\ , \}, \{},
keywordstyle=\color{blue},
emph={access,and,break,class,continue,def,del,elif ,else,
except,exec,finally,for,from,global,if,import,in,i s,
lambda,not,or,pass,print,raise,return,try,while},
emphstyle=\color{black}\bfseries,
emph={[2]True, False, None, self},
emphstyle=[2]\color{green},
emph={[3]from, import, as},
emphstyle=[3]\color{blue},
upquote=true,
columns=flexible, % pour empecher d'avoir un espacement mono
morecomment=[s]{"""}{"""},
commentstyle=\color{Hellbraun}\slshape, 
%emph={[4]1, 2, 3, 4, 5, 6, 7, 8, 9, 0},
emphstyle=[4]\color{blue},
literate=*{:}{{\textcolor{blue}:}}{1}
{=}{{\textcolor{blue}=}}{1}
{-}{{\textcolor{blue}-}}{1}
{+}{{\textcolor{blue}+}}{1}
{*}{{\textcolor{blue}*}}{1}
{!}{{\textcolor{blue}!}}{1}
{(}{{\textcolor{blue}(}}{1}
{)}{{\textcolor{blue})}}{1}
{[}{{\textcolor{blue}[}}{1}
{]}{{\textcolor{blue}]}}{1}
{<}{{\textcolor{blue}<}}{1}
{>}{{\textcolor{blue}>}}{1}
{COMPLETER}{{\textcolor{red}COMPLETER}}{1},
literate=%
            {é}{{\'{e}}}1
            {è}{{\`{e}}}1
            {ê}{{\^{e}}}1
            {ë}{{\¨{e}}}1
            {û}{{\^{u}}}1
            {ù}{{\`{u}}}1
            {â}{{\^{a}}}1
            {à}{{\`{a}}}1
            {î}{{\^{i}}}1
            {ç}{{\c{c}}}1
            {Ç}{{\c{C}}}1
            {É}{{\'{E}}}1
            {Ê}{{\^{E}}}1
            {À}{{\`{A}}}1
            {Â}{{\^{A}}}1
            {Î}{{\^{I}}}1,
%numbers=left,                    % where to put the line-numbers; possible values are (none, left, right)
%numbersep=5pt,                   % how far the line-numbers are from the code
%numberstyle=\tiny\color{mygray}, % the style that is used for the line-numbers
}

%
%\renewcommand{\algorithmicrequire} {\textbf{\textsc{Entrées:}}}
%\renewcommand{\algorithmicensure}  {\textbf{\textsc{Sorties:}}}
%\renewcommand{\algorithmicwhile}   {\textbf{tantque}}
%\renewcommand{\algorithmicdo}      {\textbf{faire}}
%\renewcommand{\algorithmicendwhile}{\textbf{fin tantque}}
%\renewcommand{\algorithmicend}     {\textbf{fin}}
%\renewcommand{\algorithmicif}      {\textbf{si}}
%\renewcommand{\algorithmicendif}   {\textbf{finsi}}
%\renewcommand{\algorithmicelse}    {\textbf{sinon}}
%\renewcommand{\algorithmicthen}    {\textbf{alors}}
%\renewcommand{\algorithmicfor}     {\textbf{pour}}
%\renewcommand{\algorithmicforall}  {\textbf{pour tout}}
%\renewcommand{\algorithmicdo}      {\textbf{faire}}
%\renewcommand{\algorithmicendfor}  {\textbf{fin pour}}
%\renewcommand{\algorithmicloop}    {\textbf{boucler}}
%\renewcommand{\algorithmicendloop} {\textbf{fin boucle}}
%\renewcommand{\algorithmicrepeat}  {\textbf{répéter}}
%\renewcommand{\algorithmicuntil}   {\textbf{jusqu'à}}

\lstnewenvironment{termi}[1][]{
\lstset{
language=scilab,
basicstyle=\sffamily\footnotesize, 	
stringstyle=\color{red}, 
showstringspaces=false, 
alsoletter={1234567890},
otherkeywords={\ , \}, \{},
keywordstyle=\color{blue},
emph={access,and,break,class,continue,def,del,elif ,else,
except,exec,finally,for,from,global,if,import,in,i s,
lambda,not,or,pass,print,raise,return,try,while,Debut},
emphstyle=\color{black}\bfseries,
emph={[2]True, False, None, self},
emphstyle=[2]\color{green},
emph={[3]from, import, as},
emphstyle=[3]\color{blue},
upquote=true,
columns=flexible, % pour empecher d'avoir un espacement mono
morecomment=[s]{"""}{"""},
commentstyle=\color{Hellbraun}\slshape, 
%emph={[4]1, 2, 3, 4, 5, 6, 7, 8, 9, 0},
emphstyle=[4]\color{blue},
literate=*{:}{{\textcolor{blue}:}}{1}
{=}{{\textcolor{blue}=}}{1}
{-}{{\textcolor{blue}-}}{1}
{+}{{\textcolor{blue}+}}{1}
{*}{{\textcolor{blue}*}}{1}
{!}{{\textcolor{blue}!}}{1}
{(}{{\textcolor{blue}(}}{1}
{)}{{\textcolor{blue})}}{1}
{[}{{\textcolor{blue}[}}{1}
{]}{{\textcolor{blue}]}}{1}
{<}{{\textcolor{blue}<}}{1}
{>}{{\textcolor{blue}>}}{1},
%framexleftmargin=1mm, framextopmargin=1mm, frame=shadowbox, rulesepcolor=\color{blue},#1
%backgroundcolor=\color{SourceHintergrund}, 
%framexleftmargin=1mm, framexrightmargin=1mm, framextopmargin=1mm, frame=single, framerule=1pt, rulecolor=\color{black},#1
}}{}


%
%\renewcommand{\algorithmicrequire} {\textbf{\textsc{Entrées:}}}
%\renewcommand{\algorithmicensure}  {\textbf{\textsc{Sorties:}}}
%\renewcommand{\algorithmicwhile}   {\textbf{tantque}}
%\renewcommand{\algorithmicdo}      {\textbf{faire}}
%\renewcommand{\algorithmicendwhile}{\textbf{fin tantque}}
%\renewcommand{\algorithmicend}     {\textbf{fin}}
%\renewcommand{\algorithmicif}      {\textbf{si}}
%\renewcommand{\algorithmicendif}   {\textbf{finsi}}
%\renewcommand{\algorithmicelse}    {\textbf{sinon}}
%\renewcommand{\algorithmicthen}    {\textbf{alors}}
%\renewcommand{\algorithmicfor}     {\textbf{pour}}
%\renewcommand{\algorithmicforall}  {\textbf{pour tout}}
%\renewcommand{\algorithmicdo}      {\textbf{faire}}
%\renewcommand{\algorithmicendfor}  {\textbf{fin pour}}
%\renewcommand{\algorithmicloop}    {\textbf{boucler}}
%\renewcommand{\algorithmicendloop} {\textbf{fin boucle}}
%\renewcommand{\algorithmicrepeat}  {\textbf{répéter}}
%\renewcommand{\algorithmicuntil}   {\textbf{jusqu'à}}
%%%%%%%%%%%%
% Définition des vecteurs 
%%%%%%%%%%%%
 \newcommand{\vect}[1]{\overrightarrow{#1}}
\newcommand{\axe}[2]{\left(#1,\vect{#2}\right)}

\newcommand{\rep}[1]{\mathcal{R}_{#1}}
\newcommand{\vx}[1]{\vect{x_{#1}}}
\newcommand{\vy}[1]{\vect{y_{#1}}}
\newcommand{\vz}[1]{\vect{z_{#1}}}

%%%%%%%%%%%%
% Définition des torseurs 
%%%%%%%%%%%%

 \newcommand{\torseur}[1]{%
\left\{{#1}\right\}
}

\newcommand{\torseurcin}[3]{%
\left\{\mathcal{#1} \left(#2/#3 \right) \right\}
}

\newcommand{\torseurstat}[3]{%
\left\{\mathcal{#1} \left(#2\rightarrow #3 \right) \right\}
}

 \newcommand{\torseurc}[8]{%
%\left\{#1 \right\}=
\left\{
{#1}
\right\}
 = 
\left\{%
\begin{array}{cc}%
{#2} & {#5}\\%
{#3} & {#6}\\%
{#4} & {#7}\\%
\end{array}%
\right\}_{#8}%
}

 \newcommand{\torseurcol}[7]{
\left\{%
\begin{array}{cc}%
{#1} & {#4}\\%
{#2} & {#5}\\%
{#3} & {#6}\\%
\end{array}%
\right\}_{#7}%
}

 \newcommand{\torseurl}[3]{%
%\left\{\mathcal{#1}\right\}_{#2}=%
\left\{%
\begin{array}{l}%
{#1} \\%
{#2} %
\end{array}%
\right\}_{#3}%
}

 \newcommand{\vectv}[3]{%
\vect{V\left( {#1} \in {#2}/{#3}\right)}
}


\newcommand{\vectf}[2]{%
\vect{R\left( {#1} \rightarrow {#2}\right)}
}

\newcommand{\vectm}[3]{%
\vect{\mathcal{M}\left( {#1}, {#2} \rightarrow {#3}\right)}
}


 \newcommand{\vectg}[3]{%
\vect{\Gamma \left( {#1} \in {#2}/{#3}\right)}
}

 \newcommand{\vecto}[2]{%
\vect{\Omega\left( {#1}/{#2}\right)}
}
% }$$\left\{\mathcal{#1} \right\}_{#2} =%
% \left\{%
% \begin{array}{c}%
%  #3 \\%
%  #4 %
% \end{array}%
% \right\}_{#5}}
\setcounter{tocdepth}{2}
% \mtcselectlanguage{french} 


%  ------------------------------------------
% | Modification du formatage des sections : | 
%  ------------------------------------------

% Grands titres :
% ---------------

\newcommand{\titre}[1]{%
\begin{center}
      \bigskip
      \rule{\textwidth}{1pt}
      \par\vspace{0.1cm}
      
      \textbf{\large #1}
      \par\rule{\textwidth}{1pt}
    \end{center}
    \bigskip
  }

% Supprime le numéro du chapitre dans la numérotation des sections:
% -----------------------------------------------------------------
\makeatletter
\renewcommand{\thesection}{\@arabic\c@section}
\makeatother


% \titleformat{\chapter}[display]
% {\normalfont\Large\filcenter}
% {}
% {1pc}
% {\titlerule[1pt]
%   \vspace{1pc}%
%   \Huge}[\vspace{1ex}%
% \titlerule]


%%%% Chapitres Comme PY Pechard %%%%%%%%%
% numéro du chapitre
\DeclareFixedFont{\chapnumfont}{OT1}{phv}{b}{n}{80pt}
% pour le mot « Chapitre »
\DeclareFixedFont{\chapchapfont}{OT1}{phv}{m}{it}{40pt}
% pour le titre
\DeclareFixedFont{\chaptitfont}{T1}{phv}{b}{n}{25pt}

\definecolor{gris}{gray}{0.75}
\titleformat{\chapter}[display]%
	{\sffamily}%
	{\filleft\chapchapfont\color{gris}\chaptertitlename\
	\\
	\vspace{12pt}
	\chapnumfont\thechapter}%
	{16pt}%
	{\filleft\chaptitfont}%
	[\vspace{6pt}\titlerule\titlerule\titlerule]

%%%%  Fin Chapitres Comme PY Pechard %%%%%%%%%


% Section, subsection, subsubsection sans serifs :
% % ----------------------------------------------

% \makeatletter
% \renewcommand{\section}{\@startsection{section}{0}{0mm}%
% {\baselineskip}{.3\baselineskip}%
% {\normalfont\sffamily\Large\textbf}}%
% \makeatother

\makeatletter
\renewcommand{\@seccntformat}[1]{{\textcolor{bleu}{\csname
the#1\endcsname}\hspace{0.5em}}}
\makeatother

\makeatletter
\renewcommand{\section}{\@startsection{section}{1}{\z@}%
                       {-4ex \@plus -1ex \@minus -.4ex}%
                       {1ex \@plus.2ex }%
                       {\normalfont\Large\sffamily\bfseries}}%
\makeatother
 
\makeatletter
\renewcommand{\subsection}{\@startsection {subsection}{2}{\z@}
                          {-3ex \@plus -0.1ex \@minus -.4ex}%
                          {0.5ex \@plus.2ex }%
                          {\normalfont\large\sffamily\bfseries}}
\makeatother
 
\makeatletter
\renewcommand{\subsubsection}{\@startsection {subsubsection}{3}{\z@}
                          {-2ex \@plus -0.1ex \@minus -.2ex}%
                          {0.2ex \@plus.2ex }%
                          {\normalfont\large\sffamily\bfseries}}
\makeatother
 
\makeatletter             
\renewcommand{\paragraph}{\@startsection{paragraph}{4}{\z@}%
                                    {-2ex \@plus-.2ex \@minus .2ex}%
                                    {0.1ex}%               
{\normalfont\sffamily\bfseries}}
\makeatother
 
\makeatletter
\renewcommand{\subparagraph}{\@startsection{subparagraph}{5}{\z@}%
                                       {-2ex \@plus-.1ex \@minus .2ex}%
                                       {0.1ex}%
				    {\normalfont\normalsize\sffamily\bfseries}}
\makeatletter
% \makeatletter
% \renewcommand{\subsection}{\@startsection{subsection}{1}{2mm}%
% {\baselineskip}{.3\baselineskip}%
% {\normalfont\sffamily\large\textbf}}%
% \makeatother
% 
% \makeatletter
% \renewcommand{\subsubsection}{\@startsection{subsubsection}{2}{4mm}%
% {\baselineskip}{.15\baselineskip}%
% {\normalfont\sffamily\large\textbf}}%
% \makeatother
% 
% \makeatletter
% \renewcommand{\paragraph}{\@startsection{paragraph}{3}{6mm}%
% {\baselineskip}{.15\baselineskip}%
% {\normalfont\sffamily\large\textbf}}%
% \makeatother
 
\setcounter{secnumdepth}{4}





% Formatage de la table des matières 
% Paquets nécessaires : titletoc ?

% Chapitre spéciaux écrits dans un nombre cerclé dans la table des matières.
\titlecontents{chapter}[+3pc]
  {\addvspace{10pt}\sffamily\bfseries}
{\contentslabel[{\pscirclebox[fillstyle=solid,fillcolor=gray!25,
linecolor=gray!25,framesep=4pt]{\textcolor{white}{\thecontentslabel}}}]{2.5pc}}
  {}
  {\dotfill \normalfont\thecontentspage\ }

\titlecontents{section}[3pc]
  {\addvspace{2pt}\sffamily}
  {\contentslabel[\thecontentslabel]{1.8pc}}
  {}
  {\dotfill \normalfont\thecontentspage\ }

\titlecontents{subsection}[5pc]
  {\addvspace{2pt}\sffamily}
  {\contentslabel[\thecontentslabel]{1.8pc}}
  {}
  {\dotfill \normalfont\thecontentspage\ }

\titlecontents{subsubsection}[8pc]
  {\addvspace{2pt}\sffamily}
  {\contentslabel[\thecontentslabel]{3pc}}
  {}
  {\dotfill \normalfont\thecontentspage\ }
%{\;\titlerule\;\normalfont\thecontentspage\ }

\titlecontents{paragraph}[9pc]
  {\addvspace{2pt}\sffamily}
  {\contentslabel[\thecontentslabel]{3.5pc}}
  {}
  {\dotfill \normalfont\thecontentspage\ }

%pour avoir l indentation dans minipage
\newdimen\oldparindent\oldparindent=\parindent

\makeatletter
\def\@iiiminipage#1#2[#3]#4{%
  \noindent
  \leavevmode
  \@pboxswfalse
  \setlength\@tempdima{#4}%
  \def\@mpargs{{#1}{#2}[#3]{#4}}%
  \setbox\@tempboxa\vbox\bgroup
    \color@begingroup
      \hsize\@tempdima
      \textwidth\hsize \columnwidth\hsize
      \@parboxrestore
      \parindent=\oldparindent
      \def\@mpfn{mpfootnote}\def\thempfn{\thempfootnote}\c@mpfootnote\z@
      \let\@footnotetext\@mpfootnotetext
      \let\@listdepth\@mplistdepth \@mplistdepth\z@
      \@minipagerestore
      \@setminipage}
\makeatother

%Definition de la commande question
\newcounter{Qu}
\newcommand{\Question}[2][0]{
\ifthenelse{\equal{#1}{0}}                      %demande-t-on une minipage ?
{\medskip\noindent {\refstepcounter{Qu}\textbf{Q\theQu .\hspace{0,7mm}}#2}\ifshowanswers \else \smallskip \fi}  %non donc on balance le texte
{\ifshowanswers                                 %oui minipage en mode problem
\noindent {\refstepcounter{Qu}\textbf{Q\theQu .\hspace{0,7mm}}#2}    %mode solution
\else                                           %mode problem
\noindent\begin{minipage}{#1}\noindent {\refstepcounter{Qu}\textbf{Q\theQu .\hspace{0,7mm}}#2}\end{minipage}\smallskip
\fi }
}

\newcommand{\Questionpb}[2][0]{%le premier argument entre [] est par défaut à 0
\begin{onlyproblem}\Question[#1]{#2}\end{onlyproblem}
}

\newcommand{\Onlyproblem}[2][0]{%le premier argument entre [] est par défaut à 0
%si le 2e arguement est 0
\ifthenelse{\equal{#1}{0}}
%on demande un environnement pb classique
{\begin{onlyproblem}#2\end{onlyproblem}}
%sinon on demande à faire une minipage
{\begin{onlyproblem}\noindent\begin{minipage}{#1}\parskip2ex #2\end{minipage}\smallskip \end{onlyproblem} }
}

\newcounter{Sl}
\addtocounter{Sl}{+1}
\newcommand{\Solutioncnt}[1]{\bigskip\noindent \textbf{R\theSl .\hspace{0,7mm}}\addtocounter{Sl}{+1} #1}
\newcommand{\Solutionnorm}[1]{#1}

\newif\ifmixte
\let\mixte\mixtetrue
\let\nomix\mixtefalse
\nomix

\newcommand{\Solution}[1]{
\noindent
\ifmixte
\noindent\rule[0.1cm]{17cm}{0.8pt}\\
  \begin{solution}
    \ifnum\theQu>0
    \Solutionnorm{#1}
    \else
    \Solutioncnt{#1}
    \fi
    \smallskip
  \end{solution}

\noindent\rule[0.1cm]{17cm}{0.8pt}
\else
  \begin{onlysolution}
\fbox{\parbox{\linewidth-2\fboxrule-2\fboxsep}{
    \ifnum\theQu>0
    \Solutionnorm{#1}
    \else
    \Solutioncnt{#1}
    \fi
    \smallskip
}}
  \end{onlysolution}
\fi
}
% Paquets requis : 

\definecolor{gris25}{gray}{0.75}
\definecolor{bleu}{RGB}{18,33,98}
\definecolor{bleuf}{RGB}{42,94,171}
\definecolor{bleuc}{RGB}{231,239,247}
\definecolor{rougef}{RGB}{185,18,27}
\definecolor{rougec}{RGB}{255,230,231}
\definecolor{vertf}{RGB}{103,126,82}
\definecolor{vertc}{RGB}{220,255,191}
\definecolor{violetf}{RGB}{112,48,160}
\definecolor{violetc}{RGB}{230,224,236}
\definecolor{jaunec}{RGB}{220,255,191}



\newenvironment{corrige}[1][\hsize]%
{%
    \def\FrameCommand%
    {%
\rotatebox{90}{\textit{\textsf{Corrigé}}} 
        {\color{violetf}\vrule width 3pt}%
        \hspace{0pt}%must no space.
        \fboxsep=\FrameSep\colorbox{violetc}%
    }%
    \MakeFramed{\hsize #1 \advance\hsize-\width\FrameRestore}%
}%
{\endMakeFramed}%

\newenvironment{sci}[1][\hsize]%
{%
    \def\FrameCommand%
    {%
%\rotatebox{90}{\textit{\textsf{Scilab}}\includegraphics[height=.8cm]{png/logo_scilab}} 
\rotatebox{90}{\includegraphics[height=.6cm]{png/logo_scilab}} 
        {\color{violetf}\vrule width 3pt}%
        \hspace{0pt}%must no space.
        \fboxsep=\FrameSep\colorbox{violetc}%
    }%
    \MakeFramed{\hsize #1 \advance\hsize-\width\FrameRestore}%
}%
{\endMakeFramed}%

\newenvironment{pseudo}[1][\hsize]%
{%
    \def\FrameCommand%
    {%
\rotatebox{90}{\textit{\textsf{Pseudo Code}}} 
        {\color{violetf}\vrule width 3pt}%
        \hspace{0pt}%must no space.
        \fboxsep=\FrameSep\colorbox{violetc}%
    }%
    \MakeFramed{\hsize #1 \advance\hsize-\width\FrameRestore}%
}%
{\endMakeFramed}%

\newenvironment{py}[1][\hsize]%
{%
    \def\FrameCommand%
    {%
%\rotatebox{90}{\textit{\textsf{Python}}} 
\rotatebox{90}{\includegraphics[height=.6cm]{png/logo_python}} 
        {\color{violetf}\vrule width 3pt}%
        \hspace{0pt}%must no space.
        \fboxsep=\FrameSep\colorbox{violetc}%
    }%
    \MakeFramed{\hsize #1 \advance\hsize-\width\FrameRestore}%
}%
{\endMakeFramed}%


\newenvironment{term}[1][\hsize]%
{%
    \def\FrameCommand%
    {%
\rotatebox{90}{\textit{\textsf{Terminal}}} 
        {\color{violetf}\vrule width 3pt}%
        \hspace{0pt}%must no space.
        \fboxsep=\FrameSep\colorbox{violetc}%
    }%
    \MakeFramed{\hsize #1 \advance\hsize-\width\FrameRestore}%
}%
{\endMakeFramed}%


\newenvironment{rem}[1][\hsize]%
{%
    \def\FrameCommand
    {%
\rotatebox{90}{\textit{\textsf{Remarque}}} 
        {\color{bleuf}\vrule width 3pt}%
        \hspace{0pt}%must no space.
        \fboxsep=\FrameSep\colorbox{bleuc}%
    }%
    \MakeFramed{\hsize#1\advance\hsize-\width\FrameRestore}%
}%
{\endMakeFramed}%


\newenvironment{savoir}[1][\hsize]%
{%
    \def\FrameCommand
    {%
\rotatebox{90}{\textit{\textsf{Savoir}}} 
        {\color{bleuf}\vrule width 3pt}%
        \hspace{0pt}%must no space.
        \fboxsep=\FrameSep\colorbox{bleuc}%
    }%
    \MakeFramed{\hsize#1\advance\hsize-\width\FrameRestore}%
}%
{\endMakeFramed}%

\newenvironment{Objectif}[1][\hsize]%
{%
    \def\FrameCommand
    {%
\rotatebox{90}{\textit{\textsf{Objectif}}} 
        {\color{bleuf}\vrule width 3pt}%
        \hspace{0pt}%must no space.
        \fboxsep=\FrameSep\colorbox{bleuc}%
    }%
    \MakeFramed{\hsize#1\advance\hsize-\width\FrameRestore}%
}%
{\endMakeFramed}%

\newenvironment{prob}[1][\hsize]%
{%
    \def\FrameCommand%
    {%
\rotatebox{90}{\textit{\textsf{ Problématique}}} 
        {\color{rougef}\vrule width 3pt}%
        \hspace{0pt}%must no space.
        \fboxsep=\FrameSep\colorbox{rougec}%
    }%
    \MakeFramed{\hsize#1\advance\hsize-\width\FrameRestore}%
}%
{\endMakeFramed}%

\newenvironment{obj}[1][\hsize]%
{%
    \def\FrameCommand%
    {%
\rotatebox{90}{\textit{\textsf{ $\;$}}} 
        {\color{rougef}\vrule width 3pt}%
        \hspace{0pt}%must no space.
        \fboxsep=\FrameSep\colorbox{rougec}%
    }%
    \MakeFramed{\hsize#1\advance\hsize-\width\FrameRestore}%
}%
{\endMakeFramed}%

\newenvironment{defi}[1][\hsize]%
{%
    \def\FrameCommand%
    {%
\rotatebox{90}{\textit{\textsf{Définition\\}}} 
        {\color{bleuf}\vrule width 3pt}%
        \hspace{0pt}%must no space.
        \fboxsep=\FrameSep\colorbox{bleuc}%
    }%
    \MakeFramed{\hsize#1\advance\hsize-\width\FrameRestore}%
}%
{\endMakeFramed}%


\newenvironment{demo}[1][\hsize]%
{%
    \def\FrameCommand%
    {%
\rotatebox{90}{\textit{\textsf{Démonstration\\}}} 
        {\color{bleuf}\vrule width 3pt}%
        \hspace{0pt}%must no space.
        \fboxsep=\FrameSep\colorbox{bleuc}%
    }%
    \MakeFramed{\hsize#1\advance\hsize-\width\FrameRestore}%
}%
{\endMakeFramed}%


\newenvironment{hypo}[1][\hsize]%
{%
    \def\FrameCommand%
    {%
\rotatebox{90}{\textit{\textsf{Hypothèse\\}}} 
        {\color{bleuf}\vrule width 3pt}%
        \hspace{0pt}%must no space.
        \fboxsep=\FrameSep\colorbox{bleuc}%
    }%
    \MakeFramed{\hsize#1\advance\hsize-\width\FrameRestore}%
}%
{\endMakeFramed}%


\newenvironment{prop}[1][\hsize]%
{%
    \def\FrameCommand%
    {%
\rotatebox{90}{\textit{\textsf{Propriété\\}}} 
        {\color{bleuf}\vrule width 3pt}%
        \hspace{0pt}%must no space.
        \fboxsep=\FrameSep\colorbox{bleuc}%
    }%
    \MakeFramed{\hsize#1\advance\hsize-\width\FrameRestore}%
}%
{\endMakeFramed}%

\newenvironment{props}[1][\hsize]%
{%
    \def\FrameCommand%
    {%
\rotatebox{90}{\textit{\textsf{Propriétés\\}}} 
        {\color{bleuf}\vrule width 3pt}%
        \hspace{0pt}%must no space.
        \fboxsep=\FrameSep\colorbox{bleuc}%
    }%
    \MakeFramed{\hsize#1\advance\hsize-\width\FrameRestore}%
}%
{\endMakeFramed}%

\newenvironment{exemple}[1][\hsize]%
{%
    \def\FrameCommand%
    {%
\rotatebox{90}{\textit{\textsf{Exemple\\}}} 
        {\color{vertf}\vrule width 3pt}%
        \hspace{0pt}%must no space.
        \fboxsep=\FrameSep\colorbox{vertc}%
    }%
    \MakeFramed{\hsize#1\advance\hsize-\width\FrameRestore}%
}%
{\endMakeFramed}%

\newenvironment{exercice}[1][\hsize]%
{%
    \def\FrameCommand%
    {%
\rotatebox{90}{\textit{\textsf{Exercice\\}}} 
        {\color{vertf}\vrule width 3pt}%
        \hspace{0pt}%must no space.
        \fboxsep=\FrameSep\colorbox{vertc}%
    }%
    \MakeFramed{\hsize#1\advance\hsize-\width\FrameRestore}%
}%
{\endMakeFramed}%

\newenvironment{Support}[1][\hsize]%
{%
    \def\FrameCommand%
    {%
\rotatebox{90}{\textit{\textsf{Support de cours\\}}} 
        {\color{vertf}\vrule width 3pt}%
        \hspace{0pt}%must no space.
        \fboxsep=\FrameSep\colorbox{jaunec}%
    }%
    \MakeFramed{\hsize#1\advance\hsize-\width\FrameRestore}%
}%
{\endMakeFramed}%

\newenvironment{resultat}[1][\hsize]%
{%
    \def\FrameCommand%
    {%
\rotatebox{90}{\textit{\textsf{Résultat\\}}} 
        {\color{rougef}\vrule width 3pt}%
        \hspace{0pt}%must no space.
        \fboxsep=\FrameSep\colorbox{rougec}%
    }%
    \MakeFramed{\hsize#1\advance\hsize-\width\FrameRestore}%
}%
{\endMakeFramed}%

\newenvironment{methode}[1][\hsize]%
{%
    \def\FrameCommand%
    {%
\rotatebox{90}{\textit{\textsf{Méthode\\}}} 
        {\color{rougef}\vrule width 3pt}%
        \hspace{0pt}%must no space.
        \fboxsep=\FrameSep\colorbox{rougec}%
    }%
    \MakeFramed{\hsize#1\advance\hsize-\width\FrameRestore}%
}%
{\endMakeFramed}%

\newenvironment{theo}[1][\hsize]%
{%
    \def\FrameCommand%
    {%
\rotatebox{90}{\textit{\textsf{Théorème\\}}} 
        {\color{rougef}\vrule width 3pt}%
        \hspace{0pt}%must no space.
        \fboxsep=\FrameSep\colorbox{rougec}%
    }%
    \MakeFramed{\hsize#1\advance\hsize-\width\FrameRestore}%
}%
{\endMakeFramed}%

\newenvironment{warn}[1][\hsize]%
{%
    \def\FrameCommand%
    {%
\rotatebox{90}{\textit{\textsf{Attention\\}}} 
        {\color{rougef}\vrule width 3pt}%
        \hspace{0pt}%must no space.
        \fboxsep=\FrameSep\colorbox{rougec}%
    }%
    \MakeFramed{\hsize#1\advance\hsize-\width\FrameRestore}%
}%
{\endMakeFramed}%

%Si le boolen xp est vrai : compilation pour xabi
%Sinon compilation Damien
\newboolean{xp}
\setboolean{xp}{true}

\newboolean{prof}
\setboolean{prof}{false}

\newif\ifprof
%\proftrue
\proffalse

\newboolean{td}
\setboolean{td}{true}

\usepackage[%
    pdftitle={CI3 - CIN - },
    pdfauthor={Xavier Pessoles},
    colorlinks=true,
    linkcolor=blue,
    citecolor=magenta]{hyperref}

\def\discipline{Sciences Industrielles de l'Ingénieur}

\def\xxtitre{\ifthenelse{\boolean{xp}}{CI 3 -- CIN : Étude du comportement cinématique des systèmes}{}}

\def\xxsoustitre{\ifthenelse{\boolean{xp}}{
Chapitre 4 -- Étude des chaînes fermées : Détermination des lois Entrée -- Sortie}{
}}


\def\xxauteur{\ifthenelse{\boolean{xp}}{
\noindent Xavier \textsc{Pessoles}}{
}}


\def\xxpied{\ifthenelse{\boolean{xp}}{
CI 4 : Cinématique \\
Ch 4 : Chaînes fermées -- Lois entrée -- sortie -- TD -- \ifthenelse{\boolean{prof}}{P}{E}%
}{
}}

\def\xxcathegorie{\ifthenelse{\boolean{xp}}{
2013 -- 2014 \\
Xavier \textsc{Pessoles}}{
Informatique - Cours}}

%---------------------------------------------------------------------------


\begin{document}

\ifthenelse{\boolean{xp}}{
\sloppy
\hyphenpenalty 10000


%------------- En tetes et Pieds de Pages ------------

\pagestyle{fancy}
\renewcommand{\headrulewidth}{0pt}
\fancyhead{}
\fancyhead[L]{%
\noindent\begin{minipage}[c]{2.6cm}%
\includegraphics[width=2cm]{png/logo_ptsi.png}%
\end{minipage}}


\fancyhead[C]{\rule{12cm}{.5pt}}


\fancyhead[R]{%
\noindent\begin{minipage}[c]{3cm}
\begin{flushright}
\footnotesize{\textit{\textsf{\discipline}}}%
\end{flushright}
\end{minipage}
}



\fancyhead[C]{\rule{12cm}{.5pt}}

\renewcommand{\footrulewidth}{0.2pt}

\fancyfoot[C]{\footnotesize{\bfseries \thepage}}
\fancyfoot[L]{%
\begin{minipage}[c]{.2\linewidth}
\noindent\footnotesize{{\xxauteur}}
\end{minipage}
}

\ifthenelse{\boolean{prof}}{%
\fancyfoot[R]{\footnotesize{\xxpied}}}

\begin{center}
 \huge\textsc{\xxtitre}
\end{center}

\begin{center}
 \LARGE\textsc{\xxsoustitre}
\end{center}

\vspace{.5cm}
}{\ifthenelse{\boolean{xp}}{
\usepackage[%
    pdftitle={OS et Environnement de développement},
    pdfauthor={Xavier Pessoles},
    colorlinks=true,
    linkcolor=blue,
    citecolor=magenta]{hyperref}}{
\usepackage[%
    pdftitle={OS et Environnement de développement},
    pdfauthor={Damien Iceta},
    colorlinks=true,
    linkcolor=blue,
    citecolor=magenta]{hyperref}}

\usepackage{pifont}
\usepackage{lastpage}

% \makeatletter \let\ps@plain\ps@empty \makeatother
%% DEBUT DU DOCUMENT
%% =================
\sloppy
\hyphenpenalty 10000

\newcommand{\Pointilles}[1][3]{%
\multido{}{#1}{\makebox[\linewidth]{\dotfill}\\[\parskip]
}}


\colorlet{shadecolor}{orange!15}

\newtheorem{theorem}{Theorem}


\begin{document}


\newboolean{prof}
\setboolean{prof}{true}
%------------- En tetes et Pieds de Pages ------------


\pagestyle{fancy}
%\renewcommand{\headrulewidth}{0}
\renewcommand{\headrulewidth}{0.2pt} %pour mettre le trait en haut

\fancyhead{}
\fancyhead[L]{
\footnotesize{{{\xxtitre}}}%
%\noindent\noindent\begin{minipage}[c]{2.6cm}
%\includegraphics[width=2.5cm]{png/logo.png}%
%\end{minipage}
}

%\fancyhead[C]{\rule{12cm}{.5pt}}  %pour mettre le petit trait en haut


\fancyhead[R]{%
\noindent\begin{minipage}[c]{3cm}
\begin{flushright}
\footnotesize{{{\xxcathegorie}}}%
\end{flushright}
\end{minipage}
}

\renewcommand{\footrulewidth}{0.2pt}

\fancyfoot[C]{\footnotesize{}}
\fancyfoot[L]{%
\begin{minipage}[l]{.2\linewidth}
\noindent\footnotesize{{\xxauteur}}
\end{minipage}
\begin{minipage}[c]{.15\linewidth}
%\includegraphics[width=2cm]{png/logoCC.png}
\end{minipage}}

\ifthenelse{\boolean{prof}}{%
\fancyfoot[R]{\footnotesize{Page \thepage\   sur  \pageref{LastPage}}}}

\begin{center}
 \huge\textsc{\xxtitre}
\end{center}

\begin{center}
 \LARGE\textsc{\xxsoustitre}
\end{center}

\vspace{.5cm}}



%\renewcommand{\baselinestretch}{1.2}
%\setlength{\parskip}{2ex plus 0.5ex minus 0.2ex}



\begin{comp}
\noindent \textbf{Résoudre :} à partir des modèles retenus :
\begin{itemize}
\item choisir une méthode de résolution analytique, graphique, numérique;
\item mettre en \oe{}uvre une méthode de résolution.
\end{itemize}

\noindent \textit{Rés -- C1.1 :} Loi entrée sortie géométrique et cinématique -- Fermeture géométrique.

%\noindent \textit{Mod2 -- C4.1 :} Représentation par schéma bloc.
\end{comp}

\section*{Véhicule auto guidé}

\begin{flushright}
\textit{D'après ATS -- 2014.}
\end{flushright}




\begin{minipage}[c]{.7\linewidth}
***
\begin{obj} 
***
\end{obj}


 
\end{minipage} \hfill
\begin{minipage}[c]{.25\linewidth}
\begin{center}
%\includegraphics[width=\textwidth]{images/Simulateur1}
\end{center}
\end{minipage}

\ifprof
\else
 On donne un extrait du cahier des charges.

\begin{minipage}[c]{.3\linewidth}
\begin{center}
%\includegraphics[width=\textwidth]{images/uc}

\textit{Diagramme des cas d'utilisation}
\end{center}
\end{minipage} \hfill
\begin{minipage}[c]{.3\linewidth}
\begin{center}
%\includegraphics[width=\textwidth]{images/ct}

\textit{Diagramme de contexte}
\end{center}
\end{minipage} \hfill
\begin{minipage}[c]{.38\linewidth}
\begin{center}
%\includegraphics[width=\textwidth]{images/req}

\textit{Diagramme partiel des exigences}
\end{center}
\end{minipage}

Le respect des conditions de sécurité pour la circulation des chariots à proximité du personnel de l'hôpital nécessite de maîtriser les trajectoires de déplacements du VAG.

\begin{obj}
Les objectifs de cette partie sont les suivants :
\begin{itemize}
\item étudier le comportement en virage du chariot par une approche géométrique et valider la solution retenue;
\item vérifier et ajuster les performances de la boucle de régulation de direction.
\end{itemize}
\end{obj}

Le schéma cinématique incomplet suivant (Figure 14) montre l’architecture du système d’orientation des roues arrière et avant :

Un moteur électrique (non représenté) permet la rotation simultanée des roue arrière \textbf{(1)} et avant \textbf{(4)} autour des axes $\left(A,\vect{y} \right)$ et $\left(C,\vect{y} \right)$ respectivement. L’arbre moteur est \textbf{(14)}. Par l’intermédiaire d’un réducteur de rapport de réduction $k_1$ (non représenté ci-dessus), l’arbre moteur entraîne un satellite, noté \textbf{(11)}. Ce dernier, engrène avec la roue solidaire du carter \textbf{(5)}. Pour la suite la roue sera aussi notée \textbf{(5)}. Le stator du moteur est lié à \textbf{(6)}, qui joue le rôle de porte satellite et transmet la puissance nécessaire à l’orientation de la roue arrière, et à la roue avant, par l’intermédiaire de la biellette \textbf{(12)} et de l’axe \textbf{(13)}.

Afin d’assurer la stabilité du chariot en virage pour éviter tout accident, le dérapage en glissement au niveau des roue doit être limité.

La figure suivante (Figure 15) montre le système d’orientation dans deux positions : la figure de gauche montre les roues lorsque le VAG avance en ligne droite, la seconde lorsque le VAG est en virage :
 
Figure 15  Système d’orientation des roues vu de dessus
Partie 2.1 : Etude de la relation entrée/sortie du mécanisme d’orientation des roues

\begin{obj}
L’objectif de cette partie est de déterminer la relation entrée / sortie du système d’orientation des roues afin de valider la solution constructeur.

Une étude préliminaire montre que, pour qu’il n’y ait pas de glissement en virage au niveau des roues, les deux roues doivent être orientées d’un même angle $\alpha_1 = \alpha_4$.
\end{obj}

Afin de déterminer la relation entre les angles d’orientations des roues arrière et avant, on adopte la modélisation suivante :
 
Figure 16 Schéma cinématique simplifié du système d’orientation des roues vu de dessus

Ce schéma cinématique représente le système d’orientation simplifié (vue de dessus du chariot), muni du paramétrage suivant :
\begin{itemize}
\item le repère $\mathcal{R}\left(\vect{x},\vect{y},\vect{z} \right)$ est attaché au carter \textbf{(5)} du VAG;
\item la pièce \textbf{(6)} est en liaison pivot d’axe $\left(K,\vect{y} \right)$ avec \textbf{(5)}. On y attache le repère : $\mathcal{R}_1\left(\vect{x_1},\vect{y},\vect{z_1} \right)$    et on a $\alpha_2 =\left(\vect{x_1},\vect{x} \right)=\left( \vect{z_1},\vect{z} \right)$;
\item la biellette \textbf{(12)} est en liaison rotule de centre $L$ avec \textbf{(6)} d’une part, et en liaison rotule de centre $M$ avec \textbf{(13)} d’autre part. On y attache le repère $\mathcal{R}_{12}$ et on a $\beta=\left(\vect{x_{12}},\vect{x} \right)=\left( \vect{z_{12}},\vect{z} \right)$;
\item la pièce \textbf{(13)} est en liaison pivot d’axe $\left(N,\vect{y}\right)$ avec \textbf{(5)}. On y attache le repère $\mathcal{R}_{13}$ et on a : $\alpha_4=\left(\vect{x},\vect{x_4} \right)=\left( \vect{z},\vect{z_4} \right)$.
\end{itemize}
Les autres données utiles à l’étude sont :
\begin{itemize}
\item $\vect{LK}=L_1 \vect{z_1} $ avec $L_1 = 98\;mm$;
\item $\vect{LM}=L_2 \vect{x_{12}}$ avec $L_2 = 1\,060\;mm$;
\item $\vect{NM}=L_1 \vect{z_4}$;
\item $\vect{KN}=L_3 \vect{x}$ avec $L_3=1\,042\; mm$.
\end{itemize}
On donne ci-dessous les figures géométrales pour les projections :
 
Figure 17 Figures géométrales

L’objectif étant de trouver la relation entre $\alpha_1$ et $\alpha_4$, on se propose d’exploiter la fermeture géométrique $K-N-M-L-K$.

\subparagraph{}\textit{En exploitant la fermeture géométrique proposée, déterminer les deux équations scalaires liant $\alpha_1$, $\alpha_4$ et $\beta$, issues de la projection sur 
$\vect{x}$ et  $\vect{z}$.}
 
\subparagraph{}\textit{En manipulant judicieusement les deux équations précédemment obtenues, donner la relation entre $\alpha_1$ et $\alpha_4$ et montrer qu’elle peut se mettre sous la forme suivante :}
$$
A\cos\alpha_4 - B\sin\alpha_4 = C
$$
Avec :
\begin{itemize}
\item $A = 2L_1^2 \cos\alpha_1$;
\item $B = 2\left(L_1^2 \sin\alpha_1 -L_1L_3\right)$;
\item $C = L_2^2 -2L_1^2 -L_3^2 + 2L_1L_3\sin\alpha_1$.
\end{itemize}
	 
Ces relations pourront être admises par la suite.

 Le terme $A^2+B^2$ ne s’annulant jamais, la relation précédente peut aussi s’écrire :
 $$
 \dfrac{A}{\sqrt{A^2+B^2}} \cos\alpha_4 -  \dfrac{B}{\sqrt{A^2+B^2}} \sin\alpha_4
 =
  \dfrac{C}{\sqrt{A^2+B^2}} 
 $$
 
 On pose $\cos\theta =  \dfrac{A}{\sqrt{A^2+B^2}} $ et $\sin\theta =  \dfrac{B}{\sqrt{A^2+B^2}} $. 


\subparagraph{}\textit{Exprimer $\alpha_4$ en fonction de $\theta$, $A$, $B$ et $C$.}

L’expression obtenue montre une relation non linéaire entre $\alpha_1$ et $\alpha_4$ et par conséquent la condition de roulement sans glissement pour chaque roue, en virage n’est pas strictement respectée.

Cependant, on donne sur le document réponse DR1 la courbe représentative de $\alpha_4$ en fonction de $\alpha_1$, pour $\alpha_1$ variant de 0\textdegree à 50\textdegree.

\subparagraph{}\textit{Tracer en vert, sur le document réponse DR1, la courbe  représentant l’angle $\alpha_{41}$ idéal désiré qui respecte la condition de roulement sans glissement, c'est-à-dire $\alpha_{41}=\alpha_1$. Commenter.}


L’écart entre les courbes est une fonction croissante de $\alpha_1$ : plus l’angle $\alpha_1$ de la roue arrière augmente, plus l’angle $\alpha_1$ de la roue avant s’éloigne de sa valeur idéale.

Pour la suite, on note $\Delta$ l’écart entre l’angle idéal $\alpha_{41}$ et l’angle réel obtenu $\alpha_4$, soit $\Delta = \alpha_{4} - \alpha_{41}$.

La recherche de l’écart maximal $\Delta_{\text{max}}$ nécessite de déterminer l’angle de rotation maximal de la roue avant noté $\alpha_{\text{max}}$.

En phase d’utilisation, le VAG parcourt des trajets qui sont des successions de portions de lignes droites et de portions circulaires. Plus le rayon des portions circulaires est faible, plus les roues du VAG doivent pivoter. Le plus petit rayon relevé sur le trajet au CHU de Dijon est $R_{\text{min}} = 908\;mm$.

On se propose de déterminer l’angle $\alpha_{\text{max}}$ qui permet au VAG de décrire un cercle de rayon $R_{\text{min}}$. On fera l’hypothèse suivante : au regard des résultats de la Q32, on peut admettre en première approche que $\alpha_4\simeq\alpha_1$  et donc $\alpha_{\text{4max}} \simeq\alpha_{\text{1max}}$.

La Figure 18 représente partiellement le VAG, vu de dessus, lorsqu’il décrit un cercle de rayon $R_{\text{min}}$ :
 
Figure 18 Schématisation retenue lors d’un virage

On pose $\vect{||\vect{AC}||}=2a = L_3 =1\,042\; mm$. Le cercle passe par les points $A$ et $C$ et les axes des roues du VAG s’intersectent tous au centre $P$ du cercle de rayon $R_{\text{min}}$.

\subparagraph{}\textit{Par une approche géométrique, exprimer $\alpha_{\text{1max}}$ en fonction de $R_\text{min}$ et $a$. Puis réaliser l’application numérique. Un résultat en degrés est attendu.}
\subparagraph{}\textit{Sur le document réponse DR1, tracer en rouge l’erreur maximale  correspondant à l’angle $\alpha_{1\text{max}}$ trouvé précédemment. Calculer alors l’écart maximal relatif en pourcent (\%) noté $\Delta\%\text{max} = \Delta\text{max}/\alpha1\text{max}$.}

Une erreur maximale relative de 15\% est admise pour garantir un bon comportement en virage du VAG et limiter le glissement entre les roues et le sol.

\subparagraph{}\textit{Conclure quant au respect du cahier des charges au regard de la condition de roulement sans glissement en virage et justifier succinctement le choix du constructeur. Proposer, en une ligne, une autre solution qui permettrait d’orienter la roue avant du VAG en fonction de l’orientation de sa roue arrière.}




\subparagraph{}
\textit{Réaliser le schéma cinématique 3D de la structure articulée en vous aidant du tracé ci-dessous.}
\ifprof
\begin{corrige}
\begin{center}
%\includegraphics[width=.4\textwidth]{images/Corr_02}
\end{center}

\end{corrige}
\else
\begin{center}
%\includegraphics[width=.6\textwidth]{images/Simulateur4}
\end{center}
\fi


\end{document}

