\documentclass[11pt,oneside]{article}
\usepackage[T1]{fontenc}
\usepackage[utf8]{inputenc}
%\DeclareUnicodeCharacter{00A0}{ }
\usepackage[adobe-utopia]{mathdesign}

\usepackage{amsmath}
\usepackage[francais]{babel}
\usepackage[dvips]{graphicx}
%\usepackage{here}
\usepackage{framed}
\usepackage[normalem]{ulem}
\usepackage{fancyhdr}
\usepackage{titlesec}
\usepackage{vmargin}

\usepackage{amsmath}
\usepackage{ifthen}
\usepackage{multirow}
\usepackage{multicol} % Portions de texte en colonnes

%\usepackage{xltxtra} % Logo XeLaTeX
%\usepackage{pst-solides3d}
\usepackage{color}
%\usepackage{colortbl}
\usepackage{titletoc} % Pour la mise en forme de la table des matières

%\usepackage[crop=off]{auto-pst-pdf}
%\usepackage{bclogo}


%\usepackage{longtable}
%\usepackage{flafter}%floatants après la référence
%\usepackage{pst-solides3d}
%\usepackage{pstricks}
%\usepackage{minitoc}
%\setcounter{minitocdepth}{4}
%\usepackage{draftcopy}% "Brouillon"
%\usepackage{floatflt}
%\usepackage{psfrag}
%\usepackage{listings} % Permet d'insérer du code de programmation
%\usepackage{lmodern}
%\usepackage[adobe-utopia,uppercase=upright,greeklowercase=upright]{mathdesign}
%\usepackage{minionpro}
%\usepackage{pifont}
%\usepackage{amssymb}
%\usepackage[francais]{varioref}

\setmarginsrb{1.5cm}{1cm}{1cm}{1.5cm}{1cm}{1cm}{1cm}{1cm}

\definecolor{gris25}{gray}{0.75}
\definecolor{bleu}{RGB}{18,33,98}
\definecolor{bleuf}{RGB}{42,94,171}
\definecolor{bleuc}{RGB}{231,239,247}
\definecolor{rougef}{RGB}{185,18,27}
\definecolor{rougec}{RGB}{255,230,231}
\definecolor{vertf}{RGB}{103,126,82}
\definecolor{vertc}{RGB}{220,255,191}
\definecolor{violetf}{RGB}{112,48,160}
\definecolor{violetc}{RGB}{230,224,236}
\definecolor{jaunec}{RGB}{220,255,191}

\usepackage[%
    pdftitle={TD Cinématique},
    pdfauthor={Xavier Pessoles},
    colorlinks=true,
    linkcolor=blue,
    citecolor=magenta]{hyperref}



% \makeatletter \let\ps@plain\ps@empty \makeatother
%% DEBUT DU DOCUMENT
%% =================
\sloppy
\hyphenpenalty 10000

\newcommand{\Pointilles}[1][3]{%
\multido{}{#1}{\makebox[\linewidth]{\dotfill}\\[\parskip]
}}


\begin{document}


\newboolean{prof}
\setboolean{prof}{true}
%------------- En tetes et Pieds de Pages ------------
\pagestyle{fancy}
\renewcommand{\headrulewidth}{0pt}

\fancyhead{}
\fancyhead[L]{%
\begin{minipage}[c]{1.6cm}
\includegraphics[width=1.4cm]{png/logo_ptsi.png}%
\end{minipage}
\rule{2cm}{.5pt}
}

\fancyhead[C]{\rule{12cm}{.5pt}}

\fancyhead[R]{%
\begin{minipage}[c]{3cm}
\begin{flushright}
\footnotesize{\textit{\textsf{Sciences Industrielles\\ pour l'Ingénieur}}}%
\end{flushright}
\end{minipage}
}

\renewcommand{\footrulewidth}{0.2pt}

\fancyfoot[C]{\footnotesize{\bfseries \thepage}}
%\fancyfoot[L]{\footnotesize{2011 -- 2012} \\ X. \textsc{Pessoles}}
\ifthenelse{\boolean{prof}}{%
%\fancyfoot[R]{\footnotesize{TD -- CI 2 : Cinématique -- P}}
}{%
%\fancyfoot[R]{\footnotesize{TD -- CI 2 : Cinématique}}
}


%\begin{center}
%\textit{Centre d'intérêt}
%\end{center}

\begin{center}
 \LARGE\textsc{CI 2 -- Cinématique : Modélisation, prévision et vérification du comportement cinématiques des systèmes}
\end{center}

\begin{center}
 \Large\textsc{Chapitre 5 -- Étude graphique des mouvements plans} 
\end{center}

\vspace{1cm}

\section*{Barrière de régulation de la Tamise}

Le système proposé est une barrière destinée à protéger Londres contre des remontées
d'eaux de mers lors des grandes marées. En effet, l'ensemble de la région de Londres est soumis à un
risque très important d'inondations accentué avec les montées récentes du niveau de la mer dues au
réchauffement climatique.


\begin{center}
\includegraphics[width=.8\textwidth]{png/img1}
\end{center}

La barrière, mise en place sur la tamise depuis 1982, est longue de 520m et est constituée de
6 portes pivotantes actionnées par des vérins hydrauliques. Au repos, les portes 1 (voir schéma ci-après)
de forme circulaire reposent au fond de la tamise. Les plus grandes portes font 61 m de long
et 20 m de haut pour une masse de 3700 tonnes. Elles sont capables de supporter des charges de
plus de 9000 tonnes.

L'objectif est de calculer la vitesse de rotation des portes connaissant la vitesse de translation des
vérins dans la configuration dessinée. Les vérins sont alimentés sous une pression hydraulique de
valeur $P_{alim}$.
Données :
\begin{itemize}
\item La vitesse de translation de la tige du vérin 5 par rapport à 0 : $||\vect{V(5/0)}||=5\cdot10^{-3} m/s$;
\item $CD=10,25m$;
\item toutes les liaisons sont supposées parfaites.
\end{itemize}

\paragraph{}
\textit{En tenant compte de l'alimentation en énergie des vérins, tracer la vitesse de $\vect{(M,4/ 0)}$.}

\paragraph{}
\textit{Tracer la direction de la vitesse de $\vect{(I,3/ 0)}$.}

\paragraph{}
\textit{Déterminer la vitesse de $\vect{(I,3/ 0)}$.}

\paragraph{}
\textit{Tracer la direction de la vitesse $\vect{(E,3/ 0)}$.}

\paragraph{}
\textit{Déterminer la vitesse de $\vect{(E,2/ 0)}$.}

\paragraph{}
\textit{Tracer la direction de la vitesse $\vect{(D,1/ 0)}$.}

\paragraph{}
\textit{Déterminer la vitesse de $\vect{(D,1/ 0)}$.}

\paragraph{}
\textit{En déduire la valeur de la vitesse instantanée de rotation $\omega(1/0)$.}

\paragraph{}
\textit{Déterminer le centre instantané de rotation de 2/0 en utilisant le théorème des trois
CIR alignés.}

\newpage

Représentation graphique des vitesses : $1 \; cm$ pour $2,5\cdot10^{-3} \; m/s$

\vfill

\begin{center}
\includegraphics[width=.9\textwidth]{png/img2}
\end{center}



\end{document}