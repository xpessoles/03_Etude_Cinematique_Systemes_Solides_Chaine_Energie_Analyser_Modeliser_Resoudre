\documentclass[10pt,oneside]{article}
\usepackage[T1]{fontenc}
\usepackage[utf8]{inputenc}
%\DeclareUnicodeCharacter{00A0}{ }
\usepackage[adobe-utopia]{mathdesign}

\usepackage{amsmath}
\usepackage[francais]{babel}
\usepackage[dvips]{graphicx}
%\usepackage{here}
\usepackage{framed}
\usepackage[normalem]{ulem}
\usepackage{fancyhdr}
\usepackage{titlesec}
\usepackage{vmargin}

\usepackage{amsmath}
\usepackage{ifthen}
\usepackage{multirow}
\usepackage{multicol} % Portions de texte en colonnes

%\usepackage{xltxtra} % Logo XeLaTeX
%\usepackage{pst-solides3d}
\usepackage{color}
%\usepackage{colortbl}
\usepackage{titletoc} % Pour la mise en forme de la table des matières

%\usepackage[crop=off]{auto-pst-pdf}
%\usepackage{bclogo}


%\usepackage{longtable}
%\usepackage{flafter}%floatants après la référence
%\usepackage{pst-solides3d}
%\usepackage{pstricks}
%\usepackage{minitoc}
%\setcounter{minitocdepth}{4}
%\usepackage{draftcopy}% "Brouillon"
%\usepackage{floatflt}
%\usepackage{psfrag}
%\usepackage{listings} % Permet d'insérer du code de programmation
%\usepackage{lmodern}
%\usepackage[adobe-utopia,uppercase=upright,greeklowercase=upright]{mathdesign}
%\usepackage{minionpro}
%\usepackage{pifont}
%\usepackage{amssymb}
%\usepackage[francais]{varioref}

\setmarginsrb{1.5cm}{1cm}{1cm}{1.5cm}{1cm}{1cm}{1cm}{1cm}

\definecolor{gris25}{gray}{0.75}
\definecolor{bleu}{RGB}{18,33,98}
\definecolor{bleuf}{RGB}{42,94,171}
\definecolor{bleuc}{RGB}{231,239,247}
\definecolor{rougef}{RGB}{185,18,27}
\definecolor{rougec}{RGB}{255,230,231}
\definecolor{vertf}{RGB}{103,126,82}
\definecolor{vertc}{RGB}{220,255,191}
\definecolor{violetf}{RGB}{112,48,160}
\definecolor{violetc}{RGB}{230,224,236}
\definecolor{jaunec}{RGB}{220,255,191}


%Si le boolen xp est vrai : compilation pour xabi
%Sinon compilation Damien
\newboolean{xp}
\setboolean{xp}{true}

\newboolean{prof}
\setboolean{prof}{false}

\def\xxtitre{\ifthenelse{\boolean{xp}}{
CI 3 -- CIN : Étude du comportement cinématique des systèmes}{
}}

\def\xxsoustitre{\ifthenelse{\boolean{xp}}{
Chapitre 6 -- Cinématique du point immatériel dans un solide en mouvement}{
}}


\def\xxauteur{\ifthenelse{\boolean{xp}}{
\noindent 2013 -- 2014 \\
Xavier \textsc{Pessoles}}{
}}


\def\xxpied{\ifthenelse{\boolean{xp}}{
CI 3 : CIN -- Cours \\
Ch 6 : Cinématique du point -- TD Transmission -- \ifthenelse{\boolean{prof}}{P}{E}%
}{
}}

\usepackage[%
    pdftitle={CIN : Cinématique du point},
    pdfauthor={Xavier Pessoles},
    colorlinks=true,
    linkcolor=blue,
    citecolor=magenta]{hyperref}


\usepackage{pifont}
\sloppy
\hyphenpenalty 10000


\begin{document}





% \makeatletter \let\ps@plain\ps@empty \makeatother
%% DEBUT DU DOCUMENT
%% =================




%------------- En tetes et Pieds de Pages ------------


\pagestyle{fancy}
\ifthenelse{\boolean{xp}}{%
\renewcommand{\headrulewidth}{0pt}}{%
\renewcommand{\headrulewidth}{0.2pt}} %pour mettre le trait en haut
%\renewcommand{\headrulewidth}{0.2pt}

\fancyhead{}
\fancyhead[L]{%
\ifthenelse{\boolean{xp}}{%
\noindent\begin{minipage}[c]{2.6cm}%
\includegraphics[width=2cm]{png/logo_ptsi.png}%
\end{minipage}%
}{%
\footnotesize{\textit{\textsf{Lycée François Premier}}}
}}

\ifthenelse{\boolean{xp}}{%
\fancyhead[C]{\rule{12cm}{.5pt}}}{
}


\fancyhead[R]{%
\noindent\begin{minipage}[c]{3cm}
\begin{flushright}
\footnotesize{\textit{\textsf{Sciences Industrielles \\ de l'ingénieur}}}%
\end{flushright}
\end{minipage}
}


\ifthenelse{\boolean{xp}}{%
\fancyhead[C]{\rule{12cm}{.5pt}}}{
}

\renewcommand{\footrulewidth}{0.2pt}

\fancyfoot[C]{\footnotesize{\bfseries \thepage}}
\fancyfoot[L]{%
\begin{minipage}[c]{.2\linewidth}
\noindent\footnotesize{{\xxauteur}}
\end{minipage}
\ifthenelse{\boolean{xp}}{}{%
\begin{minipage}[c]{.15\linewidth}
\includegraphics[width=2cm]{png/logoCC.png}
\end{minipage}}
}


\fancyfoot[R]{\footnotesize{\xxpied}}



\begin{center}
 \Large\textsc{\xxtitre}

\end{center}

\begin{center}
 \large\textsc{\xxsoustitre}
\end{center}

\begin{center}
 \large\textsc{Travail Dirigé : Calcul de de rapport de vitesse}
\end{center}

\begin{flushright}
 \textit{Ressources de Florestan Mathurin\footnote{\url{http://florestan.mathurin.free.fr}}.}
\end{flushright}



\subsection*{Exercice 1 : Réducteur Demag}
\begin{minipage}[c]{.7\linewidth}

\subparagraph{}
\textit{Identifier les classes d'équivalence cinématique sur le plan et trouver l'arbre d'entrée.}


\subparagraph{}
\textit{Construire le schéma cinématique du réducteur dans le même plan que le dessin.}


\subparagraph{}
\textit{Déterminer le rapport de réduction du réducteur.}

\textbf{Données :}
$Z_6=16$, $Z_{9a}=46$, $Z_{9b}=19$, $Z_{11a}=59$, $Z_{11b}=17$, $Z_{16}=85$.
$m_6=m_{9a}=m_{9b}=m_{11a}=1\; mm$, $m_{11b}=m_{16}=1,25\; mm$.

\end{minipage} \hfill
\begin{minipage}[c]{.25\linewidth}
\begin{center}
 \includegraphics[width=.85\textwidth]{png/demag1}
\end{center}
\end{minipage}

%\subparagraph{}
%\textit{Indiquer une différence majeure entre le système réel et le système modélisé.}


\ifthenelse{\boolean{prof}}{%
\begin{corrige}

\end{corrige}
}{%
}


\begin{center}
 \includegraphics[width=.9\textwidth]{png/demag2}
\end{center}

\subsection*{Exercice 2 : Sécateur Pellenc}
\setcounter{subparagraph}{0}


\begin{minipage}[c]{.25\linewidth}
\begin{center}
 \includegraphics[width=.85\textwidth]{png/secateur1}
\end{center}
\end{minipage} \hfill
\begin{minipage}[c]{.7\linewidth}

La période de taille de la vigne dure 2 mois environ. Les viticulteurs coupent 9 à 10 heures par jour. Ils répètent donc le même geste des millions de fois avec un sécateur. Les sociétés réalisant le du matériel agricole ont imaginé un sécateur électrique capable de réduire la fatigue de la main et du bras tout en laissant au viticulteur la commande de la coupe et sa liberté de mouvement. Le sécateur développé par la société Pellenc permet notamment de réaliser 60 coupes de diamètre 22 mm par minute. L’ensemble sécateur Pellenc est constitué d’un sécateur électronique, d’une mallette source d’énergie, d’une sacoche avec harnais et ceinture et d’un chargeur de batterie.

\end{minipage}




\begin{minipage}[c]{.55\linewidth}
Lorsque l’utilisateur appuie sur la gâchette, le moteur transmet par l’intermédiaire d’un réducteur à train épicycloïdal un mouvement de rotation à la vis à billes. L’écrou se déplace en translation par rapport à la vis et par l’intermédiaire d’une biellette met en rotation la lame mobile générant ainsi le mouvement de coupe. 
\end{minipage} \hfill
\begin{minipage}[c]{.4\linewidth}
\begin{center}
 \includegraphics[width=.95\textwidth]{png/secateur2}
\end{center}
\end{minipage} 


\vspace{.25cm}

Le moteur tourne à la vitesse de rotation $N_1=1\,400\;tr/min$ le (le rotor est lié au planétaire 1). La vis à billes liée au porte-satellite 4 tourne à la vitesse de rotation $N_4=350^; tr/min$. On note $Z_1$ le nombre dents du planétaire 1, $Z_2$ celui du satellite 2 et $Z_3$ celui de la couronne liée au bâti.

\vspace{.25cm}

\begin{minipage}[c]{.32\linewidth}
\begin{center}
 \includegraphics[width=.85\textwidth]{png/secateur3}
\end{center}
\end{minipage} \hfill
\begin{minipage}[c]{.55\linewidth}


\subparagraph{}
\textit{Déterminer le rapport de réduction du train épicycloïdal $\omega(4/0)/\omega(1/0)$ en fonction de $Z_1$ et $Z_3$.}

\subparagraph{}
\textit{Faire l’application numérique et déterminer une relation entre $Z_1$ et $Z_3$. Sachant que $Z_1=19$ en déduire $Z_3$.}

\subparagraph{}
\textit{Sachant que les roues dentées du train ont les mêmes modules, déterminer une relation géométrique entre les diamètres des éléments dentés $d_1$, $d_2$, $d_3$ puis en déduire une relation entre $Z_2$, $Z_1$, $Z_3$ (condition d’entraxe). Calculer la valeur de $Z_2$.}


\end{minipage}




\end{document}
