\documentclass[11pt,oneside]{article}
\usepackage[T1]{fontenc}
\usepackage[utf8]{inputenc}
%\DeclareUnicodeCharacter{00A0}{ }
\usepackage[adobe-utopia]{mathdesign}

\usepackage{amsmath}
\usepackage[francais]{babel}
\usepackage[dvips]{graphicx}
%\usepackage{here}
\usepackage{framed}
\usepackage[normalem]{ulem}
\usepackage{fancyhdr}
\usepackage{titlesec}
\usepackage{vmargin}

\usepackage{amsmath}
\usepackage{ifthen}
\usepackage{multirow}
\usepackage{multicol} % Portions de texte en colonnes

%\usepackage{xltxtra} % Logo XeLaTeX
%\usepackage{pst-solides3d}
\usepackage{color}
%\usepackage{colortbl}
\usepackage{titletoc} % Pour la mise en forme de la table des matières

%\usepackage[crop=off]{auto-pst-pdf}
%\usepackage{bclogo}


%\usepackage{longtable}
%\usepackage{flafter}%floatants après la référence
%\usepackage{pst-solides3d}
%\usepackage{pstricks}
%\usepackage{minitoc}
%\setcounter{minitocdepth}{4}
%\usepackage{draftcopy}% "Brouillon"
%\usepackage{floatflt}
%\usepackage{psfrag}
%\usepackage{listings} % Permet d'insérer du code de programmation
%\usepackage{lmodern}
%\usepackage[adobe-utopia,uppercase=upright,greeklowercase=upright]{mathdesign}
%\usepackage{minionpro}
%\usepackage{pifont}
%\usepackage{amssymb}
%\usepackage[francais]{varioref}

\setmarginsrb{1.5cm}{1cm}{1cm}{1.5cm}{1cm}{1cm}{1cm}{1cm}

\definecolor{gris25}{gray}{0.75}
\definecolor{bleu}{RGB}{18,33,98}
\definecolor{bleuf}{RGB}{42,94,171}
\definecolor{bleuc}{RGB}{231,239,247}
\definecolor{rougef}{RGB}{185,18,27}
\definecolor{rougec}{RGB}{255,230,231}
\definecolor{vertf}{RGB}{103,126,82}
\definecolor{vertc}{RGB}{220,255,191}
\definecolor{violetf}{RGB}{112,48,160}
\definecolor{violetc}{RGB}{230,224,236}
\definecolor{jaunec}{RGB}{220,255,191}

\usepackage[%
    pdftitle={Cinématique - Modélisation des systèmes mécaniques},
    pdfauthor={Xavier Pessoles},
    colorlinks=true,
    linkcolor=blue,
    citecolor=magenta]{hyperref}
\usepackage{schemabloc}



% \makeatletter \let\ps@plain\ps@empty \makeatother
%% DEBUT DU DOCUMENT
%% =================
\sloppy
\hyphenpenalty 10000

\newcommand{\Pointilles}[1][3]{%
\multido{}{#1}{\makebox[\linewidth]{\dotfill}\\[\parskip]
}}


\colorlet{shadecolor}{orange!15}

\newtheorem{theorem}{Theorem}


\begin{document}


\newboolean{prof}
\setboolean{prof}{true}
%------------- En tetes et Pieds de Pages ------------
\pagestyle{fancy}
\renewcommand{\headrulewidth}{0pt}

\fancyhead{}
\fancyhead[L]{%
\begin{minipage}[c]{1.6cm}
\includegraphics[width=1.4cm]{png/logo_jh_ptsi.png}%
\end{minipage}
\rule{2cm}{.5pt}
}

\fancyhead[C]{\rule{12cm}{.5pt}}

\fancyhead[R]{%
\begin{minipage}[c]{3cm}
\begin{flushright}
\footnotesize{\textit{\textsf{Sciences Industrielles\\ pour l'Ingénieur}}}%
\end{flushright}
\end{minipage}
}

\renewcommand{\footrulewidth}{0.2pt}

\fancyfoot[C]{\footnotesize{\bfseries \thepage}}
\fancyfoot[L]{\footnotesize{2011 -- 2012} \\ X. \textsc{Pessoles}}
\ifthenelse{\boolean{prof}}{%
\fancyfoot[R]{\footnotesize{Cours -- CI 7 : SLCI -- P}}
}{%
\fancyfoot[R]{\footnotesize{Cours -- CI 1 : Cinématique}}
}

%\begin{center}
%\textit{Centre d'intérêt}
%\end{center}

\begin{center}
 \huge\textsc{CI 2 -- Cinématique : Modélisation, prévision et vérification du comportement cinématiques des systèmes}
\end{center}

\begin{center}
 \LARGE\textsc{Chapitre 1 -- Modélisation des systèmes mécaniques} 
\end{center}

\begin{center}
 \Large\textsc{Système de climatisation d'une automobile} 
\end{center}
\vspace{.5cm}


\section*{Mise en situation}
\begin{minipage}[c]{.5\linewidth}
Le système de climatisation d'une automobile permet
d'obtenir à l'intérieur de l'habitacle une température
agréable quelles que soient les conditions climatiques
extérieures. Il est composé :
\begin{itemize}
\item d'un dispositif de chauffage qui réchauffe l'air pulsé
à travers les éléments d'un radiateur alimenté par
l'eau de refroidissement du moteur ;
\item d'un dispositif de réfrigération qui refroidit l'air pulsé
dans l'habitacle tout en lui retirant une partie de son
humidité et de ses poussières.
\end{itemize}
Ce dispositif de réfrigération (voir figure 1) se
compose principalement d'un compresseur 1, de deux
échangeurs (un condenseur 2 et un évaporateur 5),
d'un filtre receveur 3 et d'une soupape d'expansion 4
qui fait fonction de détendeur.

\end{minipage}\hfill
\begin{minipage}[c]{.45\linewidth}
\begin{center}
\includegraphics[width=.9\textwidth]{png/compresseur}
\end{center}
\end{minipage}

\vspace{.25cm}

L'objet de cette étude est essentiellement le
compresseur à pistons axiaux (voir document 1) intégré dans un système de réfrigération.

\section*{Fonctionnement}
\subsection*{Le système de réfrigération}
Entraîné par le moteur thermique au moyen d'une courroie, le compresseur aspire le fluide réfrigérant à basse pression et à
l'état gazeux, et le refoule à haute pression. Le fluide réfrigérant traverse alors le condensent, d'où il ressort à l'état liquide
avant de passer dans le filtre. Celui-ci amortit les excès pendant les phases de charges variables et filtre les particules
solides. La soupape d'expansion, réglée au montage et pilotée par une sonde, assure le débit et abaisse la pression du
fluide à l'entrée de l'évaporateur.
L'évaporateur a un rôle primordial. Le fluide réfrigérant qui le traverse absorbe la chaleur de l'air ambiant extérieur, qui est
pulsé vers l'habitacle. L'air, qui pénètre à l'intérieur de l'habitacle, est donc refroidi. De plus la capacité réfrigérante de
l'évaporateur permet la déshumidification de l'air, ce qui accroît notablement le bien-être dans l'habitacle. Le réglage de
l'installation est tel que le fluide réfrigérant sort de l'évaporateur à l'état gazeux.

\subsection*{Le compresseur}
Le compresseur est représenté, sur le document 1, en coupe longitudinale dans le plan (C,x,y) fixe par rapport au corps 1. Il
est composé de cinq pistons 13 identiques, de diamètre 35 mm, disposés axialement. Lorsque la bobine 18 de l'embrayage
électromagnétique est alimentée, le champ magnétique fait adhérer la rondelle 20 sur la poulie 19 qui est alors en liaison
encastrement avec l'arbre d'entrée 23. Le plateau came 2 et le plateau oscillant 3 transforment le mouvement de rotation
continue de l'arbre d'entrée 23 en un mouvement de translation alternatif des pistons 13.

\section*{Nomenclature}
\begin{minipage}[c]{.46\linewidth}
\begin{center}
\begin{tabular}{|c|c|p{6cm}|}
\hline 
32 &1& Fourreau \\ \hline
31 &5 &Vis HM 515 \\ \hline
30 &1 &Goupille élastique \\ \hline
29 &6 &Rondelle \\ \hline
28 &6 &Rivet \\ \hline
27 &1 &Circlips \\ \hline
26 &1 &Rondelle butée \\ \hline
25 &1 &Rondelle ressort \\ \hline
24 &1 &Rondelle de réglage \\ \hline
23 &1 &Arbre d'entrée \\ \hline
22 &1 &Butée à aiguilles \\ \hline
21 &1 &Moyeu \\ \hline
20 &1 &Rondelle flasque d'embrayage \\ \hline
19 &1 &Poulie d'entraînement \\ \hline
18 &1 &Bobine \\ \hline
17 &2 &Roulement à billes \\ \hline
\textbf{Rep} & \textbf{Nb} & \textbf{Désignation} \\ \hline
\end{tabular}
\end{center}
\end{minipage}\hfill
\begin{minipage}[c]{.46\linewidth}
\begin{center}
\begin{tabular}{|c|c|p{6cm}|}
\hline 
16 &1 &Couvercle moyeu \\ \hline
15 &1 &Roulement à aiguilles \\ \hline
14 &5 &Bielle \\ \hline
13 &5 &Piston \\ \hline
12 &1 &Clapet \\ \hline
11 &1 &Culasse \\ \hline
10& 1 &Clapet \\ \hline
9 &1 &Couvercle de culasse \\ \hline
8 &1 &Ressort \\ \hline
7 &1 &Clavette \\ \hline
6 &1 &Pignon fixe 17 dents \\ \hline
5& 1 & Bille de poussée \\ \hline
4& 1 &Roue conique 17 dents \\ \hline
3& 1 &Plateau oscillant \\ \hline
2& 2 &Plateau came \\ \hline
1& 1 &Corps \\ \hline
\textbf{Rep} & \textbf{Nb} & \textbf{Désignation} \\ \hline
\end{tabular}
\end{center}

\end{minipage}
\end{document}
