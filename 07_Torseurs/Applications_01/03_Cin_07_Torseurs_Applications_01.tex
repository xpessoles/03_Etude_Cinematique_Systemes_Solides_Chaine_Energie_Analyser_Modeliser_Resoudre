\documentclass[10pt,oneside]{article}
\usepackage[T1]{fontenc}
\usepackage[utf8]{inputenc}
%\DeclareUnicodeCharacter{00A0}{ }
\usepackage[adobe-utopia]{mathdesign}

\usepackage{amsmath}
\usepackage[francais]{babel}
\usepackage[dvips]{graphicx}
%\usepackage{here}
\usepackage{framed}
\usepackage[normalem]{ulem}
\usepackage{fancyhdr}
\usepackage{titlesec}
\usepackage{vmargin}

\usepackage{amsmath}
\usepackage{ifthen}
\usepackage{multirow}
\usepackage{multicol} % Portions de texte en colonnes

%\usepackage{xltxtra} % Logo XeLaTeX
%\usepackage{pst-solides3d}
\usepackage{color}
%\usepackage{colortbl}
\usepackage{titletoc} % Pour la mise en forme de la table des matières

%\usepackage[crop=off]{auto-pst-pdf}
%\usepackage{bclogo}


%\usepackage{longtable}
%\usepackage{flafter}%floatants après la référence
%\usepackage{pst-solides3d}
%\usepackage{pstricks}
%\usepackage{minitoc}
%\setcounter{minitocdepth}{4}
%\usepackage{draftcopy}% "Brouillon"
%\usepackage{floatflt}
%\usepackage{psfrag}
%\usepackage{listings} % Permet d'insérer du code de programmation
%\usepackage{lmodern}
%\usepackage[adobe-utopia,uppercase=upright,greeklowercase=upright]{mathdesign}
%\usepackage{minionpro}
%\usepackage{pifont}
%\usepackage{amssymb}
%\usepackage[francais]{varioref}

\setmarginsrb{1.5cm}{1cm}{1cm}{1.5cm}{1cm}{1cm}{1cm}{1cm}

\definecolor{gris25}{gray}{0.75}
\definecolor{bleu}{RGB}{18,33,98}
\definecolor{bleuf}{RGB}{42,94,171}
\definecolor{bleuc}{RGB}{231,239,247}
\definecolor{rougef}{RGB}{185,18,27}
\definecolor{rougec}{RGB}{255,230,231}
\definecolor{vertf}{RGB}{103,126,82}
\definecolor{vertc}{RGB}{220,255,191}
\definecolor{violetf}{RGB}{112,48,160}
\definecolor{violetc}{RGB}{230,224,236}
\definecolor{jaunec}{RGB}{220,255,191}

\usepackage[%
    pdftitle={CIN -- Torseurs -- Applications},
    pdfauthor={Xavier Pessoles},
    colorlinks=true,
    linkcolor=blue,
    citecolor=magenta]{hyperref}



% \makeatletter \let\ps@plain\ps@empty \makeatother
%% DEBUT DU DOCUMENT
%% =================
\sloppy
\hyphenpenalty 10000

\newcommand{\Pointilles}[1][3]{%
\multido{}{#1}{\makebox[\linewidth]{\dotfill}\\[\parskip]
}}


\begin{document}


\newboolean{prof}
\setboolean{prof}{false}
%------------- En tetes et Pieds de Pages ------------
\pagestyle{fancy}
\renewcommand{\headrulewidth}{0pt}

\fancyhead{}
\fancyhead[L]{%
\noindent\noindent\begin{minipage}[c]{2.6cm}
%Lycée Rouvière PTSI
\includegraphics[width=2cm]{png/logo_ptsi.png}%
\end{minipage}
}

\fancyhead[C]{\rule{12cm}{.5pt}}

\fancyhead[R]{%
\begin{minipage}[c]{3cm}
\begin{flushright}
\footnotesize{\textit{\textsf{Sciences Industrielles\\ de l'Ingénieur}}}%
\end{flushright}
\end{minipage}
}

\renewcommand{\footrulewidth}{0.2pt}

\fancyfoot[C]{\footnotesize{\bfseries \thepage}}
\fancyfoot[L]{\footnotesize{2013 -- 2014} \\ X. \textsc{Pessoles}}
\ifthenelse{\boolean{prof}}{%
\fancyfoot[R]{\footnotesize{CI 3 : CIN -- Applications} \\ \footnotesize{Ch 7 : Torseurs -- P}}
}{%
\fancyfoot[R]{\footnotesize{CI 3 : CIN -- Applications} \\ \footnotesize{Ch 7 : Torseurs -- E}}
}


%\begin{center}
%\textit{Centre d'intérêt}
%\end{center}



\begin{center}
 \Large\textsc{CI 3 -- CIN : Étude du comportement cinématique des systèmes}
\end{center}

\begin{center}
 \large\textsc{Chapitre 7 -- Torseurs}
\end{center}

\begin{center}
\textsc{Exercices d'application} 
\end{center}
\begin{flushright}
\textit{D'après ressources de Jean-Pierre Pupier.} 
\end{flushright}
\vspace{.5cm}

\subsection*{Exercice 1}
\setcounter{subparagraph}{0}
Soit le torseur suivant :
$\torseur{\mathcal{T}} = \torseurcol{1}{-3}{2}{13}{5}{1}{A,\mathcal{B}}$. On donne $\vect{OA}=\left[\begin{array}{c} 1 \\ 2 \\ 1\end{array}\right]_{\mathcal{B}}$.

\subparagraph{}
\textit{Calculer les éléments de réduction de ce torseur au point $B$ tel que $\vect{OB}=3\vect{y}$.}
\ifthenelse{\boolean{prof}}{
\begin{corrige}
\end{corrige}
}{}

\subparagraph{}
\textit{Est-ce un torseur particulier ? Si oui, quel est son nom ?}
\ifthenelse{\boolean{prof}}{
\begin{corrige}
\end{corrige}
}{}

\subsection*{Exercice 2}
\setcounter{subparagraph}{0}
Soit le torseur suivant :
$\torseur{\mathcal{T}} = \torseurcol{1}{-3}{2}{13}{5}{3}{A,\mathcal{B}}$. On donne $\vect{OA}=\left[\begin{array}{c} 1 \\ 2 \\ 1\end{array}\right]_{\mathcal{B}}$.

\subparagraph{}
\textit{Est-ce un torseur particulier ? Si non, calculer son pas, puis son moment central.}
\ifthenelse{\boolean{prof}}{
\begin{corrige}
\end{corrige}
}{}

\subsection*{Exercice 3}
\setcounter{subparagraph}{0}
On donne le torseur :
$\torseur{\mathcal{T}} = \torseurcol{1}{2}{-3}{4}{-1}{2}{A,\mathcal{B}}$. On donne $\vect{OA}=\left[\begin{array}{c} 2 \\ 1 \\ -1\end{array}\right]_{\mathcal{B}}$ et 
$\vect{OB}=\left[\begin{array}{c} 3 \\ -2 \\ -2\end{array}\right]_{\mathcal{B}}$.

\subparagraph{}
\textit{Vérifier que le champ de vecteur est bien équiprojectif.}
\ifthenelse{\boolean{prof}}{
\begin{corrige}
\end{corrige}
}{}


\subsection*{Exercice 4}
\setcounter{subparagraph}{0}
On donne les torseurs suivants :
$\torseur{\mathcal{T}_1} = \torseurcol{1}{-4}{1}{1}{-2}{3}{A,\mathcal{B}}$ et 
$\torseur{\mathcal{T}_2} = \torseurcol{3}{-4}{-1}{-3}{-5}{1}{B,\mathcal{B}}$. On a par ailleurs 
$\vect{OA}=\left[\begin{array}{c} 1 \\ 2 \\ -5\end{array}\right]_{\mathcal{B}}$ et 
$\vect{OB}=\left[\begin{array}{c} 4 \\ -2 \\ 3\end{array}\right]_{\mathcal{B}}$.

\subparagraph{}
\textit{En utilisant les notations adaptées, calculer au point $A$ $\torseur{\mathcal{T}}=\torseur{\mathcal{T}_1}+\torseur{\mathcal{T}_2}$.}
\ifthenelse{\boolean{prof}}{
\begin{corrige}
\end{corrige}
}{}

\subparagraph{}
\textit{Calculer le comoment des deux torseurs.}
\ifthenelse{\boolean{prof}}{
\begin{corrige}
\end{corrige}
}{}


\subsection*{Exercice 5}
\setcounter{subparagraph}{0}
\subparagraph{}
\textit{Écrire un exemple numérique des torseurs suivants : 
\begin{itemize}
\item torseur couple; 
\item torseur glisseur en un point central;
\item le même en un point non central;
\item un torseur quelconque.
\end{itemize}}
\ifthenelse{\boolean{prof}}{
\begin{corrige}
\end{corrige}
}{}

\subsection*{Exercice 6 -- Copeau coincé}
\setcounter{subparagraph}{0}
Une plaque 1 est en appui plan sur une table de machine $R$ et est placée dans ce plan par trois appuis ponctuels $A$, $B$ et $C$.

 Lors de la mise en place de la plaque sur ces points un copeau d'épaisseur $e$ se coince entre le point $B$ et la pièce.

\begin{minipage}[c]{.47\linewidth}
\begin{center}
\includegraphics[width=.9\textwidth]{png/fig1}
\end{center}
\end{minipage} \hfill
\begin{minipage}[c]{.47\linewidth}
\begin{center}
\includegraphics[width=.9\textwidth]{png/fig2}
\end{center}
\end{minipage}

\vspace{.25cm}

\textit{Application numérique} :
$a=10\; mm$, $b=35\; mm$, $c= 20\; mm$, $x=53\; mm$, $y=28\; mm$, $e=0,3\; mm$.

\subparagraph{}
\textit{Calculer le petit déplacement du point M que subit ce point du fait de la présence du copeau par rapport à la position idéale qu’il occuperait s’il n’y avait pas de copeau.}
\ifthenelse{\boolean{prof}}{
\begin{corrige}
\end{corrige}
}{}

\subparagraph{}
\textit{Faire les applications numériques.}
\ifthenelse{\boolean{prof}}{
\begin{corrige}
\end{corrige}
}{}


\end{document}

\subparagraph{}
\textit{}
\ifthenelse{\boolean{prof}}{
\begin{corrige}
\end{corrige}
}{}

\subparagraph{}
\textit{}
\ifthenelse{\boolean{prof}}{
\begin{corrige}
\end{corrige}
}{}

\subparagraph{}
\textit{}
\ifthenelse{\boolean{prof}}{
\begin{corrige}

\end{corrige}
}{}

\subparagraph{}
\textit{}
\ifthenelse{\boolean{prof}}{
\begin{corrige}
\end{corrige}
}{}


\end{document}