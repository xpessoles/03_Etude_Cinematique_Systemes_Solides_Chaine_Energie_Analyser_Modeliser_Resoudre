\documentclass[10pt]{article}
\usepackage[T1]{fontenc}
\usepackage[utf8]{inputenc}
%\DeclareUnicodeCharacter{00A0}{ }
\usepackage[adobe-utopia]{mathdesign}

\usepackage{amsmath}
\usepackage[francais]{babel}
\usepackage[dvips]{graphicx}
%\usepackage{here}
\usepackage{framed}
\usepackage[normalem]{ulem}
\usepackage{fancyhdr}
\usepackage{titlesec}
\usepackage{vmargin}

\usepackage{amsmath}
\usepackage{ifthen}
\usepackage{multirow}
\usepackage{multicol} % Portions de texte en colonnes

%\usepackage{xltxtra} % Logo XeLaTeX
%\usepackage{pst-solides3d}
\usepackage{color}
%\usepackage{colortbl}
\usepackage{titletoc} % Pour la mise en forme de la table des matières

%\usepackage[crop=off]{auto-pst-pdf}
%\usepackage{bclogo}


%\usepackage{longtable}
%\usepackage{flafter}%floatants après la référence
%\usepackage{pst-solides3d}
%\usepackage{pstricks}
%\usepackage{minitoc}
%\setcounter{minitocdepth}{4}
%\usepackage{draftcopy}% "Brouillon"
%\usepackage{floatflt}
%\usepackage{psfrag}
%\usepackage{listings} % Permet d'insérer du code de programmation
%\usepackage{lmodern}
%\usepackage[adobe-utopia,uppercase=upright,greeklowercase=upright]{mathdesign}
%\usepackage{minionpro}
%\usepackage{pifont}
%\usepackage{amssymb}
%\usepackage[francais]{varioref}

\setmarginsrb{1.5cm}{1cm}{1cm}{1.5cm}{1cm}{1cm}{1cm}{1cm}

\definecolor{gris25}{gray}{0.75}
\definecolor{bleu}{RGB}{18,33,98}
\definecolor{bleuf}{RGB}{42,94,171}
\definecolor{bleuc}{RGB}{231,239,247}
\definecolor{rougef}{RGB}{185,18,27}
\definecolor{rougec}{RGB}{255,230,231}
\definecolor{vertf}{RGB}{103,126,82}
\definecolor{vertc}{RGB}{220,255,191}
\definecolor{violetf}{RGB}{112,48,160}
\definecolor{violetc}{RGB}{230,224,236}
\definecolor{jaunec}{RGB}{220,255,191}
%\usepackage{algorithm}
%\usepackage{algorithmic}
\usepackage[french]{algorithm2e}

\SetKwBlock{Fonction}{Début Fonction}{Fin Fonction}
\SetKwComment{Comment}{start}{end}
% Python sources

\usepackage{listings}
\lstloadlanguages{R}   % pour regler les pb d accent utf8 dans les codes
\lstset{language=R} % pour regler les pb d accent utf8 dans les codes

\usepackage{textcomp}
\usepackage{setspace}
%\usepackage{palatino}

%\usepackage{color}
\definecolor{Bleu}{rgb}{0.1,0.1,1.0}
\definecolor{Noir}{rgb}{0,0,0}
\definecolor{Grau}{rgb}{0.5,0.5,0.5}
\definecolor{DunkelGrau}{rgb}{0.15,0.15,0.15}
\definecolor{Hellbraun}{rgb}{0.5,0.25,0.0}
\definecolor{Magenta}{rgb}{1.0,0.0,1.0}
\definecolor{Gris}{gray}{0.5}
\definecolor{Vert}{rgb}{0,0.5,0}
\definecolor{SourceHintergrund}{rgb}{1,1.0,0.95}


%
\renewcommand{\lstlistlistingname}{Listings}
\renewcommand{\lstlistingname}{Listing}

\lstnewenvironment{python}[1][]{
\lstset{
%escapeinside={\%*}{*)},
%inputencoding=utf8,   % pour regler les pb d accent utf8 dans les codes
%extendedchars=true,   % pour regler les pb d accent utf8 dans les codes
language=python,
basicstyle=\sffamily\footnotesize, 	
stringstyle=\color{red}, 
showstringspaces=false, 
alsoletter={1234567890},
otherkeywords={\ , \}, \{},
keywordstyle=\color{blue},
emph={access,and,break,class,continue,def,del,elif ,else,
except,exec,finally,for,from,global,if,import,in,i s,
lambda,not,or,pass,print,raise,return,try,while},
emphstyle=\color{black}\bfseries,
emph={[2]True, False, None, self},
emphstyle=[2]\color{olive},
emph={[3]from, import, as},
emphstyle=[3]\color{blue},
upquote=true,
columns=flexible, % pour empecher d'avoir un espacement mono
morecomment=[s]{"""}{"""},
commentstyle=\color{Hellbraun}\slshape, 
%emph={[4]1, 2, 3, 4, 5, 6, 7, 8, 9, 0},
emphstyle=[4]\color{blue},
literate=*{:}{{\textcolor{blue}:}}{1}
{=}{{\textcolor{blue}=}}{1}
{-}{{\textcolor{blue}-}}{1}
{+}{{\textcolor{blue}+}}{1}
{*}{{\textcolor{blue}*}}{1}
{!}{{\textcolor{blue}!}}{1}
{(}{{\textcolor{blue}(}}{1}
{)}{{\textcolor{blue})}}{1}
{[}{{\textcolor{blue}[}}{1}
{]}{{\textcolor{blue}]}}{1}
{<}{{\textcolor{blue}<}}{1}
{>}{{\textcolor{blue}>}}{1}
{COMPLETER}{{\textcolor{red}COMPLETER}}{1},
literate=%
            {é}{{\'{e}}}1
            {è}{{\`{e}}}1
            {ê}{{\^{e}}}1
            {ë}{{\¨{e}}}1
            {û}{{\^{u}}}1
            {ù}{{\`{u}}}1
            {â}{{\^{a}}}1
            {à}{{\`{a}}}1
            {î}{{\^{i}}}1
            {ç}{{\c{c}}}1
            {Ç}{{\c{C}}}1
            {É}{{\'{E}}}1
            {Ê}{{\^{E}}}1
            {À}{{\`{A}}}1
            {Â}{{\^{A}}}1
            {Î}{{\^{I}}}1, % pour regler les pb d accent utf8 dans les codes
%framexleftmargin=1mm, framextopmargin=1mm, frame=shadowbox, rulesepcolor=\color{blue},#1
%backgroundcolor=\color{SourceHintergrund}, 
%framexleftmargin=1mm, framexrightmargin=1mm, framextopmargin=1mm, frame=single, framerule=1pt, rulecolor=\color{black},#1
}}{}



\lstnewenvironment{scilab}[1][]{
\lstset{
language=scilab,
basicstyle=\sffamily\footnotesize, 	
stringstyle=\color{red}, 
showstringspaces=false, 
alsoletter={1234567890},
otherkeywords={\ , \}, \{},
keywordstyle=\color{blue},
emph={access,and,break,class,continue,def,del,elif ,else,
except,exec,finally,for,from,global,if,import,in,i s,
lambda,not,or,pass,print,raise,return,try,while,Debut},
emphstyle=\color{black}\bfseries,
emph={[2]True, False, None, self},
emphstyle=[2]\color{olive},
emph={[3]from, import, as},
emphstyle=[3]\color{blue},
upquote=true,
columns=flexible, % pour empecher d'avoir un espacement mono
morecomment=[s]{"""}{"""},
commentstyle=\color{Hellbraun}\slshape, 
%emph={[4]1, 2, 3, 4, 5, 6, 7, 8, 9, 0},
emphstyle=[4]\color{blue},
literate=*{:}{{\textcolor{blue}:}}{1}
{=}{{\textcolor{blue}=}}{1}
{-}{{\textcolor{blue}-}}{1}
{+}{{\textcolor{blue}+}}{1}
{*}{{\textcolor{blue}*}}{1}
{!}{{\textcolor{blue}!}}{1}
{(}{{\textcolor{blue}(}}{1}
{)}{{\textcolor{blue})}}{1}
{[}{{\textcolor{blue}[}}{1}
{]}{{\textcolor{blue}]}}{1}
{<}{{\textcolor{blue}<}}{1}
{>}{{\textcolor{blue}>}}{1},
%framexleftmargin=1mm, framextopmargin=1mm, frame=shadowbox, rulesepcolor=\color{blue},#1
%backgroundcolor=\color{SourceHintergrund}, 
%framexleftmargin=1mm, framexrightmargin=1mm, framextopmargin=1mm, frame=single, framerule=1pt, rulecolor=\color{black},#1
}}{}


\lstdefinestyle{stylepython}{%
escapeinside={\%*}{*)},
inputencoding=utf8,   % pour regler les pb d accent utf8 dans les codes
extendedchars=true,   % pour regler les pb d accent utf8 dans les codes
language=python,
basicstyle=\sffamily\footnotesize, 	
stringstyle=\color{red}, 
showstringspaces=false, 
alsoletter={1234567890},
otherkeywords={\ , \}, \{},
keywordstyle=\color{blue},
emph={access,and,break,class,continue,def,del,elif ,else,
except,exec,finally,for,from,global,if,import,in,i s,
lambda,not,or,pass,print,raise,return,try,while},
emphstyle=\color{black}\bfseries,
emph={[2]True, False, None, self},
emphstyle=[2]\color{green},
emph={[3]from, import, as},
emphstyle=[3]\color{blue},
upquote=true,
columns=flexible, % pour empecher d'avoir un espacement mono
morecomment=[s]{"""}{"""},
commentstyle=\color{Hellbraun}\slshape, 
%emph={[4]1, 2, 3, 4, 5, 6, 7, 8, 9, 0},
emphstyle=[4]\color{blue},
literate=*{:}{{\textcolor{blue}:}}{1}
{=}{{\textcolor{blue}=}}{1}
{-}{{\textcolor{blue}-}}{1}
{+}{{\textcolor{blue}+}}{1}
{*}{{\textcolor{blue}*}}{1}
{!}{{\textcolor{blue}!}}{1}
{(}{{\textcolor{blue}(}}{1}
{)}{{\textcolor{blue})}}{1}
{[}{{\textcolor{blue}[}}{1}
{]}{{\textcolor{blue}]}}{1}
{<}{{\textcolor{blue}<}}{1}
{>}{{\textcolor{blue}>}}{1}
{COMPLETER}{{\textcolor{red}COMPLETER}}{1},
literate=%
            {é}{{\'{e}}}1
            {è}{{\`{e}}}1
            {ê}{{\^{e}}}1
            {ë}{{\¨{e}}}1
            {û}{{\^{u}}}1
            {ù}{{\`{u}}}1
            {â}{{\^{a}}}1
            {à}{{\`{a}}}1
            {î}{{\^{i}}}1
            {ç}{{\c{c}}}1
            {Ç}{{\c{C}}}1
            {É}{{\'{E}}}1
            {Ê}{{\^{E}}}1
            {À}{{\`{A}}}1
            {Â}{{\^{A}}}1
            {Î}{{\^{I}}}1,
%numbers=left,                    % where to put the line-numbers; possible values are (none, left, right)
%numbersep=5pt,                   % how far the line-numbers are from the code
%numberstyle=\tiny\color{mygray}, % the style that is used for the line-numbers
}

%
%\renewcommand{\algorithmicrequire} {\textbf{\textsc{Entrées:}}}
%\renewcommand{\algorithmicensure}  {\textbf{\textsc{Sorties:}}}
%\renewcommand{\algorithmicwhile}   {\textbf{tantque}}
%\renewcommand{\algorithmicdo}      {\textbf{faire}}
%\renewcommand{\algorithmicendwhile}{\textbf{fin tantque}}
%\renewcommand{\algorithmicend}     {\textbf{fin}}
%\renewcommand{\algorithmicif}      {\textbf{si}}
%\renewcommand{\algorithmicendif}   {\textbf{finsi}}
%\renewcommand{\algorithmicelse}    {\textbf{sinon}}
%\renewcommand{\algorithmicthen}    {\textbf{alors}}
%\renewcommand{\algorithmicfor}     {\textbf{pour}}
%\renewcommand{\algorithmicforall}  {\textbf{pour tout}}
%\renewcommand{\algorithmicdo}      {\textbf{faire}}
%\renewcommand{\algorithmicendfor}  {\textbf{fin pour}}
%\renewcommand{\algorithmicloop}    {\textbf{boucler}}
%\renewcommand{\algorithmicendloop} {\textbf{fin boucle}}
%\renewcommand{\algorithmicrepeat}  {\textbf{répéter}}
%\renewcommand{\algorithmicuntil}   {\textbf{jusqu'à}}

\lstnewenvironment{termi}[1][]{
\lstset{
language=scilab,
basicstyle=\sffamily\footnotesize, 	
stringstyle=\color{red}, 
showstringspaces=false, 
alsoletter={1234567890},
otherkeywords={\ , \}, \{},
keywordstyle=\color{blue},
emph={access,and,break,class,continue,def,del,elif ,else,
except,exec,finally,for,from,global,if,import,in,i s,
lambda,not,or,pass,print,raise,return,try,while,Debut},
emphstyle=\color{black}\bfseries,
emph={[2]True, False, None, self},
emphstyle=[2]\color{green},
emph={[3]from, import, as},
emphstyle=[3]\color{blue},
upquote=true,
columns=flexible, % pour empecher d'avoir un espacement mono
morecomment=[s]{"""}{"""},
commentstyle=\color{Hellbraun}\slshape, 
%emph={[4]1, 2, 3, 4, 5, 6, 7, 8, 9, 0},
emphstyle=[4]\color{blue},
literate=*{:}{{\textcolor{blue}:}}{1}
{=}{{\textcolor{blue}=}}{1}
{-}{{\textcolor{blue}-}}{1}
{+}{{\textcolor{blue}+}}{1}
{*}{{\textcolor{blue}*}}{1}
{!}{{\textcolor{blue}!}}{1}
{(}{{\textcolor{blue}(}}{1}
{)}{{\textcolor{blue})}}{1}
{[}{{\textcolor{blue}[}}{1}
{]}{{\textcolor{blue}]}}{1}
{<}{{\textcolor{blue}<}}{1}
{>}{{\textcolor{blue}>}}{1},
%framexleftmargin=1mm, framextopmargin=1mm, frame=shadowbox, rulesepcolor=\color{blue},#1
%backgroundcolor=\color{SourceHintergrund}, 
%framexleftmargin=1mm, framexrightmargin=1mm, framextopmargin=1mm, frame=single, framerule=1pt, rulecolor=\color{black},#1
}}{}


%
%\renewcommand{\algorithmicrequire} {\textbf{\textsc{Entrées:}}}
%\renewcommand{\algorithmicensure}  {\textbf{\textsc{Sorties:}}}
%\renewcommand{\algorithmicwhile}   {\textbf{tantque}}
%\renewcommand{\algorithmicdo}      {\textbf{faire}}
%\renewcommand{\algorithmicendwhile}{\textbf{fin tantque}}
%\renewcommand{\algorithmicend}     {\textbf{fin}}
%\renewcommand{\algorithmicif}      {\textbf{si}}
%\renewcommand{\algorithmicendif}   {\textbf{finsi}}
%\renewcommand{\algorithmicelse}    {\textbf{sinon}}
%\renewcommand{\algorithmicthen}    {\textbf{alors}}
%\renewcommand{\algorithmicfor}     {\textbf{pour}}
%\renewcommand{\algorithmicforall}  {\textbf{pour tout}}
%\renewcommand{\algorithmicdo}      {\textbf{faire}}
%\renewcommand{\algorithmicendfor}  {\textbf{fin pour}}
%\renewcommand{\algorithmicloop}    {\textbf{boucler}}
%\renewcommand{\algorithmicendloop} {\textbf{fin boucle}}
%\renewcommand{\algorithmicrepeat}  {\textbf{répéter}}
%\renewcommand{\algorithmicuntil}   {\textbf{jusqu'à}}
%%%%%%%%%%%%
% Définition des vecteurs 
%%%%%%%%%%%%
 \newcommand{\vect}[1]{\overrightarrow{#1}}
\newcommand{\axe}[2]{\left(#1,\vect{#2}\right)}

\newcommand{\rep}[1]{\mathcal{R}_{#1}}
\newcommand{\vx}[1]{\vect{x_{#1}}}
\newcommand{\vy}[1]{\vect{y_{#1}}}
\newcommand{\vz}[1]{\vect{z_{#1}}}

%%%%%%%%%%%%
% Définition des torseurs 
%%%%%%%%%%%%

 \newcommand{\torseur}[1]{%
\left\{{#1}\right\}
}

\newcommand{\torseurcin}[3]{%
\left\{\mathcal{#1} \left(#2/#3 \right) \right\}
}

\newcommand{\torseurstat}[3]{%
\left\{\mathcal{#1} \left(#2\rightarrow #3 \right) \right\}
}

 \newcommand{\torseurc}[8]{%
%\left\{#1 \right\}=
\left\{
{#1}
\right\}
 = 
\left\{%
\begin{array}{cc}%
{#2} & {#5}\\%
{#3} & {#6}\\%
{#4} & {#7}\\%
\end{array}%
\right\}_{#8}%
}

 \newcommand{\torseurcol}[7]{
\left\{%
\begin{array}{cc}%
{#1} & {#4}\\%
{#2} & {#5}\\%
{#3} & {#6}\\%
\end{array}%
\right\}_{#7}%
}

 \newcommand{\torseurl}[3]{%
%\left\{\mathcal{#1}\right\}_{#2}=%
\left\{%
\begin{array}{l}%
{#1} \\%
{#2} %
\end{array}%
\right\}_{#3}%
}

 \newcommand{\vectv}[3]{%
\vect{V\left( {#1} \in {#2}/{#3}\right)}
}


\newcommand{\vectf}[2]{%
\vect{R\left( {#1} \rightarrow {#2}\right)}
}

\newcommand{\vectm}[3]{%
\vect{\mathcal{M}\left( {#1}, {#2} \rightarrow {#3}\right)}
}


 \newcommand{\vectg}[3]{%
\vect{\Gamma \left( {#1} \in {#2}/{#3}\right)}
}

 \newcommand{\vecto}[2]{%
\vect{\Omega\left( {#1}/{#2}\right)}
}
% }$$\left\{\mathcal{#1} \right\}_{#2} =%
% \left\{%
% \begin{array}{c}%
%  #3 \\%
%  #4 %
% \end{array}%
% \right\}_{#5}}
\setcounter{tocdepth}{2}
% \mtcselectlanguage{french} 


%  ------------------------------------------
% | Modification du formatage des sections : | 
%  ------------------------------------------

% Grands titres :
% ---------------

\newcommand{\titre}[1]{%
\begin{center}
      \bigskip
      \rule{\textwidth}{1pt}
      \par\vspace{0.1cm}
      
      \textbf{\large #1}
      \par\rule{\textwidth}{1pt}
    \end{center}
    \bigskip
  }

% Supprime le numéro du chapitre dans la numérotation des sections:
% -----------------------------------------------------------------
\makeatletter
\renewcommand{\thesection}{\@arabic\c@section}
\makeatother


% \titleformat{\chapter}[display]
% {\normalfont\Large\filcenter}
% {}
% {1pc}
% {\titlerule[1pt]
%   \vspace{1pc}%
%   \Huge}[\vspace{1ex}%
% \titlerule]


%%%% Chapitres Comme PY Pechard %%%%%%%%%
% numéro du chapitre
\DeclareFixedFont{\chapnumfont}{OT1}{phv}{b}{n}{80pt}
% pour le mot « Chapitre »
\DeclareFixedFont{\chapchapfont}{OT1}{phv}{m}{it}{40pt}
% pour le titre
\DeclareFixedFont{\chaptitfont}{T1}{phv}{b}{n}{25pt}

\definecolor{gris}{gray}{0.75}
\titleformat{\chapter}[display]%
	{\sffamily}%
	{\filleft\chapchapfont\color{gris}\chaptertitlename\
	\\
	\vspace{12pt}
	\chapnumfont\thechapter}%
	{16pt}%
	{\filleft\chaptitfont}%
	[\vspace{6pt}\titlerule\titlerule\titlerule]

%%%%  Fin Chapitres Comme PY Pechard %%%%%%%%%


% Section, subsection, subsubsection sans serifs :
% % ----------------------------------------------

% \makeatletter
% \renewcommand{\section}{\@startsection{section}{0}{0mm}%
% {\baselineskip}{.3\baselineskip}%
% {\normalfont\sffamily\Large\textbf}}%
% \makeatother

\makeatletter
\renewcommand{\@seccntformat}[1]{{\textcolor{bleu}{\csname
the#1\endcsname}\hspace{0.5em}}}
\makeatother

\makeatletter
\renewcommand{\section}{\@startsection{section}{1}{\z@}%
                       {-4ex \@plus -1ex \@minus -.4ex}%
                       {1ex \@plus.2ex }%
                       {\normalfont\Large\sffamily\bfseries}}%
\makeatother
 
\makeatletter
\renewcommand{\subsection}{\@startsection {subsection}{2}{\z@}
                          {-3ex \@plus -0.1ex \@minus -.4ex}%
                          {0.5ex \@plus.2ex }%
                          {\normalfont\large\sffamily\bfseries}}
\makeatother
 
\makeatletter
\renewcommand{\subsubsection}{\@startsection {subsubsection}{3}{\z@}
                          {-2ex \@plus -0.1ex \@minus -.2ex}%
                          {0.2ex \@plus.2ex }%
                          {\normalfont\large\sffamily\bfseries}}
\makeatother
 
\makeatletter             
\renewcommand{\paragraph}{\@startsection{paragraph}{4}{\z@}%
                                    {-2ex \@plus-.2ex \@minus .2ex}%
                                    {0.1ex}%               
{\normalfont\sffamily\bfseries}}
\makeatother
 
\makeatletter
\renewcommand{\subparagraph}{\@startsection{subparagraph}{5}{\z@}%
                                       {-2ex \@plus-.1ex \@minus .2ex}%
                                       {0.1ex}%
				    {\normalfont\normalsize\sffamily\bfseries}}
\makeatletter
% \makeatletter
% \renewcommand{\subsection}{\@startsection{subsection}{1}{2mm}%
% {\baselineskip}{.3\baselineskip}%
% {\normalfont\sffamily\large\textbf}}%
% \makeatother
% 
% \makeatletter
% \renewcommand{\subsubsection}{\@startsection{subsubsection}{2}{4mm}%
% {\baselineskip}{.15\baselineskip}%
% {\normalfont\sffamily\large\textbf}}%
% \makeatother
% 
% \makeatletter
% \renewcommand{\paragraph}{\@startsection{paragraph}{3}{6mm}%
% {\baselineskip}{.15\baselineskip}%
% {\normalfont\sffamily\large\textbf}}%
% \makeatother
 
\setcounter{secnumdepth}{4}





% Formatage de la table des matières 
% Paquets nécessaires : titletoc ?

% Chapitre spéciaux écrits dans un nombre cerclé dans la table des matières.
\titlecontents{chapter}[+3pc]
  {\addvspace{10pt}\sffamily\bfseries}
{\contentslabel[{\pscirclebox[fillstyle=solid,fillcolor=gray!25,
linecolor=gray!25,framesep=4pt]{\textcolor{white}{\thecontentslabel}}}]{2.5pc}}
  {}
  {\dotfill \normalfont\thecontentspage\ }

\titlecontents{section}[3pc]
  {\addvspace{2pt}\sffamily}
  {\contentslabel[\thecontentslabel]{1.8pc}}
  {}
  {\dotfill \normalfont\thecontentspage\ }

\titlecontents{subsection}[5pc]
  {\addvspace{2pt}\sffamily}
  {\contentslabel[\thecontentslabel]{1.8pc}}
  {}
  {\dotfill \normalfont\thecontentspage\ }

\titlecontents{subsubsection}[8pc]
  {\addvspace{2pt}\sffamily}
  {\contentslabel[\thecontentslabel]{3pc}}
  {}
  {\dotfill \normalfont\thecontentspage\ }
%{\;\titlerule\;\normalfont\thecontentspage\ }

\titlecontents{paragraph}[9pc]
  {\addvspace{2pt}\sffamily}
  {\contentslabel[\thecontentslabel]{3.5pc}}
  {}
  {\dotfill \normalfont\thecontentspage\ }

%pour avoir l indentation dans minipage
\newdimen\oldparindent\oldparindent=\parindent

\makeatletter
\def\@iiiminipage#1#2[#3]#4{%
  \noindent
  \leavevmode
  \@pboxswfalse
  \setlength\@tempdima{#4}%
  \def\@mpargs{{#1}{#2}[#3]{#4}}%
  \setbox\@tempboxa\vbox\bgroup
    \color@begingroup
      \hsize\@tempdima
      \textwidth\hsize \columnwidth\hsize
      \@parboxrestore
      \parindent=\oldparindent
      \def\@mpfn{mpfootnote}\def\thempfn{\thempfootnote}\c@mpfootnote\z@
      \let\@footnotetext\@mpfootnotetext
      \let\@listdepth\@mplistdepth \@mplistdepth\z@
      \@minipagerestore
      \@setminipage}
\makeatother

%Definition de la commande question
\newcounter{Qu}
\newcommand{\Question}[2][0]{
\ifthenelse{\equal{#1}{0}}                      %demande-t-on une minipage ?
{\medskip\noindent {\refstepcounter{Qu}\textbf{Q\theQu .\hspace{0,7mm}}#2}\ifshowanswers \else \smallskip \fi}  %non donc on balance le texte
{\ifshowanswers                                 %oui minipage en mode problem
\noindent {\refstepcounter{Qu}\textbf{Q\theQu .\hspace{0,7mm}}#2}    %mode solution
\else                                           %mode problem
\noindent\begin{minipage}{#1}\noindent {\refstepcounter{Qu}\textbf{Q\theQu .\hspace{0,7mm}}#2}\end{minipage}\smallskip
\fi }
}

\newcommand{\Questionpb}[2][0]{%le premier argument entre [] est par défaut à 0
\begin{onlyproblem}\Question[#1]{#2}\end{onlyproblem}
}

\newcommand{\Onlyproblem}[2][0]{%le premier argument entre [] est par défaut à 0
%si le 2e arguement est 0
\ifthenelse{\equal{#1}{0}}
%on demande un environnement pb classique
{\begin{onlyproblem}#2\end{onlyproblem}}
%sinon on demande à faire une minipage
{\begin{onlyproblem}\noindent\begin{minipage}{#1}\parskip2ex #2\end{minipage}\smallskip \end{onlyproblem} }
}

\newcounter{Sl}
\addtocounter{Sl}{+1}
\newcommand{\Solutioncnt}[1]{\bigskip\noindent \textbf{R\theSl .\hspace{0,7mm}}\addtocounter{Sl}{+1} #1}
\newcommand{\Solutionnorm}[1]{#1}

\newif\ifmixte
\let\mixte\mixtetrue
\let\nomix\mixtefalse
\nomix

\newcommand{\Solution}[1]{
\noindent
\ifmixte
\noindent\rule[0.1cm]{17cm}{0.8pt}\\
  \begin{solution}
    \ifnum\theQu>0
    \Solutionnorm{#1}
    \else
    \Solutioncnt{#1}
    \fi
    \smallskip
  \end{solution}

\noindent\rule[0.1cm]{17cm}{0.8pt}
\else
  \begin{onlysolution}
\fbox{\parbox{\linewidth-2\fboxrule-2\fboxsep}{
    \ifnum\theQu>0
    \Solutionnorm{#1}
    \else
    \Solutioncnt{#1}
    \fi
    \smallskip
}}
  \end{onlysolution}
\fi
}
% Paquets requis : 

\definecolor{gris25}{gray}{0.75}
\definecolor{bleu}{RGB}{18,33,98}
\definecolor{bleuf}{RGB}{42,94,171}
\definecolor{bleuc}{RGB}{231,239,247}
\definecolor{rougef}{RGB}{185,18,27}
\definecolor{rougec}{RGB}{255,230,231}
\definecolor{vertf}{RGB}{103,126,82}
\definecolor{vertc}{RGB}{220,255,191}
\definecolor{violetf}{RGB}{112,48,160}
\definecolor{violetc}{RGB}{230,224,236}
\definecolor{jaunec}{RGB}{220,255,191}



\newenvironment{corrige}[1][\hsize]%
{%
    \def\FrameCommand%
    {%
\rotatebox{90}{\textit{\textsf{Corrigé}}} 
        {\color{violetf}\vrule width 3pt}%
        \hspace{0pt}%must no space.
        \fboxsep=\FrameSep\colorbox{violetc}%
    }%
    \MakeFramed{\hsize #1 \advance\hsize-\width\FrameRestore}%
}%
{\endMakeFramed}%

\newenvironment{sci}[1][\hsize]%
{%
    \def\FrameCommand%
    {%
%\rotatebox{90}{\textit{\textsf{Scilab}}\includegraphics[height=.8cm]{png/logo_scilab}} 
\rotatebox{90}{\includegraphics[height=.6cm]{png/logo_scilab}} 
        {\color{violetf}\vrule width 3pt}%
        \hspace{0pt}%must no space.
        \fboxsep=\FrameSep\colorbox{violetc}%
    }%
    \MakeFramed{\hsize #1 \advance\hsize-\width\FrameRestore}%
}%
{\endMakeFramed}%

\newenvironment{pseudo}[1][\hsize]%
{%
    \def\FrameCommand%
    {%
\rotatebox{90}{\textit{\textsf{Pseudo Code}}} 
        {\color{violetf}\vrule width 3pt}%
        \hspace{0pt}%must no space.
        \fboxsep=\FrameSep\colorbox{violetc}%
    }%
    \MakeFramed{\hsize #1 \advance\hsize-\width\FrameRestore}%
}%
{\endMakeFramed}%

\newenvironment{py}[1][\hsize]%
{%
    \def\FrameCommand%
    {%
%\rotatebox{90}{\textit{\textsf{Python}}} 
\rotatebox{90}{\includegraphics[height=.6cm]{png/logo_python}} 
        {\color{violetf}\vrule width 3pt}%
        \hspace{0pt}%must no space.
        \fboxsep=\FrameSep\colorbox{violetc}%
    }%
    \MakeFramed{\hsize #1 \advance\hsize-\width\FrameRestore}%
}%
{\endMakeFramed}%


\newenvironment{term}[1][\hsize]%
{%
    \def\FrameCommand%
    {%
\rotatebox{90}{\textit{\textsf{Terminal}}} 
        {\color{violetf}\vrule width 3pt}%
        \hspace{0pt}%must no space.
        \fboxsep=\FrameSep\colorbox{violetc}%
    }%
    \MakeFramed{\hsize #1 \advance\hsize-\width\FrameRestore}%
}%
{\endMakeFramed}%


\newenvironment{rem}[1][\hsize]%
{%
    \def\FrameCommand
    {%
\rotatebox{90}{\textit{\textsf{Remarque}}} 
        {\color{bleuf}\vrule width 3pt}%
        \hspace{0pt}%must no space.
        \fboxsep=\FrameSep\colorbox{bleuc}%
    }%
    \MakeFramed{\hsize#1\advance\hsize-\width\FrameRestore}%
}%
{\endMakeFramed}%


\newenvironment{savoir}[1][\hsize]%
{%
    \def\FrameCommand
    {%
\rotatebox{90}{\textit{\textsf{Savoir}}} 
        {\color{bleuf}\vrule width 3pt}%
        \hspace{0pt}%must no space.
        \fboxsep=\FrameSep\colorbox{bleuc}%
    }%
    \MakeFramed{\hsize#1\advance\hsize-\width\FrameRestore}%
}%
{\endMakeFramed}%

\newenvironment{Objectif}[1][\hsize]%
{%
    \def\FrameCommand
    {%
\rotatebox{90}{\textit{\textsf{Objectif}}} 
        {\color{bleuf}\vrule width 3pt}%
        \hspace{0pt}%must no space.
        \fboxsep=\FrameSep\colorbox{bleuc}%
    }%
    \MakeFramed{\hsize#1\advance\hsize-\width\FrameRestore}%
}%
{\endMakeFramed}%

\newenvironment{prob}[1][\hsize]%
{%
    \def\FrameCommand%
    {%
\rotatebox{90}{\textit{\textsf{ Problématique}}} 
        {\color{rougef}\vrule width 3pt}%
        \hspace{0pt}%must no space.
        \fboxsep=\FrameSep\colorbox{rougec}%
    }%
    \MakeFramed{\hsize#1\advance\hsize-\width\FrameRestore}%
}%
{\endMakeFramed}%

\newenvironment{obj}[1][\hsize]%
{%
    \def\FrameCommand%
    {%
\rotatebox{90}{\textit{\textsf{ $\;$}}} 
        {\color{rougef}\vrule width 3pt}%
        \hspace{0pt}%must no space.
        \fboxsep=\FrameSep\colorbox{rougec}%
    }%
    \MakeFramed{\hsize#1\advance\hsize-\width\FrameRestore}%
}%
{\endMakeFramed}%

\newenvironment{defi}[1][\hsize]%
{%
    \def\FrameCommand%
    {%
\rotatebox{90}{\textit{\textsf{Définition\\}}} 
        {\color{bleuf}\vrule width 3pt}%
        \hspace{0pt}%must no space.
        \fboxsep=\FrameSep\colorbox{bleuc}%
    }%
    \MakeFramed{\hsize#1\advance\hsize-\width\FrameRestore}%
}%
{\endMakeFramed}%


\newenvironment{demo}[1][\hsize]%
{%
    \def\FrameCommand%
    {%
\rotatebox{90}{\textit{\textsf{Démonstration\\}}} 
        {\color{bleuf}\vrule width 3pt}%
        \hspace{0pt}%must no space.
        \fboxsep=\FrameSep\colorbox{bleuc}%
    }%
    \MakeFramed{\hsize#1\advance\hsize-\width\FrameRestore}%
}%
{\endMakeFramed}%


\newenvironment{hypo}[1][\hsize]%
{%
    \def\FrameCommand%
    {%
\rotatebox{90}{\textit{\textsf{Hypothèse\\}}} 
        {\color{bleuf}\vrule width 3pt}%
        \hspace{0pt}%must no space.
        \fboxsep=\FrameSep\colorbox{bleuc}%
    }%
    \MakeFramed{\hsize#1\advance\hsize-\width\FrameRestore}%
}%
{\endMakeFramed}%


\newenvironment{prop}[1][\hsize]%
{%
    \def\FrameCommand%
    {%
\rotatebox{90}{\textit{\textsf{Propriété\\}}} 
        {\color{bleuf}\vrule width 3pt}%
        \hspace{0pt}%must no space.
        \fboxsep=\FrameSep\colorbox{bleuc}%
    }%
    \MakeFramed{\hsize#1\advance\hsize-\width\FrameRestore}%
}%
{\endMakeFramed}%

\newenvironment{props}[1][\hsize]%
{%
    \def\FrameCommand%
    {%
\rotatebox{90}{\textit{\textsf{Propriétés\\}}} 
        {\color{bleuf}\vrule width 3pt}%
        \hspace{0pt}%must no space.
        \fboxsep=\FrameSep\colorbox{bleuc}%
    }%
    \MakeFramed{\hsize#1\advance\hsize-\width\FrameRestore}%
}%
{\endMakeFramed}%

\newenvironment{exemple}[1][\hsize]%
{%
    \def\FrameCommand%
    {%
\rotatebox{90}{\textit{\textsf{Exemple\\}}} 
        {\color{vertf}\vrule width 3pt}%
        \hspace{0pt}%must no space.
        \fboxsep=\FrameSep\colorbox{vertc}%
    }%
    \MakeFramed{\hsize#1\advance\hsize-\width\FrameRestore}%
}%
{\endMakeFramed}%

\newenvironment{exercice}[1][\hsize]%
{%
    \def\FrameCommand%
    {%
\rotatebox{90}{\textit{\textsf{Exercice\\}}} 
        {\color{vertf}\vrule width 3pt}%
        \hspace{0pt}%must no space.
        \fboxsep=\FrameSep\colorbox{vertc}%
    }%
    \MakeFramed{\hsize#1\advance\hsize-\width\FrameRestore}%
}%
{\endMakeFramed}%

\newenvironment{Support}[1][\hsize]%
{%
    \def\FrameCommand%
    {%
\rotatebox{90}{\textit{\textsf{Support de cours\\}}} 
        {\color{vertf}\vrule width 3pt}%
        \hspace{0pt}%must no space.
        \fboxsep=\FrameSep\colorbox{jaunec}%
    }%
    \MakeFramed{\hsize#1\advance\hsize-\width\FrameRestore}%
}%
{\endMakeFramed}%

\newenvironment{resultat}[1][\hsize]%
{%
    \def\FrameCommand%
    {%
\rotatebox{90}{\textit{\textsf{Résultat\\}}} 
        {\color{rougef}\vrule width 3pt}%
        \hspace{0pt}%must no space.
        \fboxsep=\FrameSep\colorbox{rougec}%
    }%
    \MakeFramed{\hsize#1\advance\hsize-\width\FrameRestore}%
}%
{\endMakeFramed}%

\newenvironment{methode}[1][\hsize]%
{%
    \def\FrameCommand%
    {%
\rotatebox{90}{\textit{\textsf{Méthode\\}}} 
        {\color{rougef}\vrule width 3pt}%
        \hspace{0pt}%must no space.
        \fboxsep=\FrameSep\colorbox{rougec}%
    }%
    \MakeFramed{\hsize#1\advance\hsize-\width\FrameRestore}%
}%
{\endMakeFramed}%

\newenvironment{theo}[1][\hsize]%
{%
    \def\FrameCommand%
    {%
\rotatebox{90}{\textit{\textsf{Théorème\\}}} 
        {\color{rougef}\vrule width 3pt}%
        \hspace{0pt}%must no space.
        \fboxsep=\FrameSep\colorbox{rougec}%
    }%
    \MakeFramed{\hsize#1\advance\hsize-\width\FrameRestore}%
}%
{\endMakeFramed}%

\newenvironment{warn}[1][\hsize]%
{%
    \def\FrameCommand%
    {%
\rotatebox{90}{\textit{\textsf{Attention\\}}} 
        {\color{rougef}\vrule width 3pt}%
        \hspace{0pt}%must no space.
        \fboxsep=\FrameSep\colorbox{rougec}%
    }%
    \MakeFramed{\hsize#1\advance\hsize-\width\FrameRestore}%
}%
{\endMakeFramed}%

%Si le boolen xp est vrai : compilation pour xabi
%Sinon compilation Damien
\newboolean{xp}
\setboolean{xp}{true}

\newboolean{prof}
\setboolean{prof}{false}

\usepackage[%
    pdftitle={CI3 - CIN -- Torseurs -- TD},
    pdfauthor={Xavier Pessoles},
    colorlinks=true,
    linkcolor=blue,
    citecolor=magenta]{hyperref}

\def\discipline{Sciences Industrielles de l'Ingénieur}
\def\xxtitre{\ifthenelse{\boolean{xp}}{%CI 3 -- CIN : Étude du comportement cinématique des systèmes
}{
Chapitre  -- }}

\def\xxsoustitre{\ifthenelse{\boolean{xp}}{
CI 3 -- CIN : Étude du comportement cinématique des systèmes}{
Partie  -- }}
\def\xxauteur{\ifthenelse{\boolean{xp}}{
\noindent 2013 -- 2014 \\
Xavier \textsc{Pessoles}}{
Damien \textsc{Iceta} \\ Xavier \textsc{Pessoles}}}

\def\xxpied{\ifthenelse{\boolean{xp}}{
CI 3 : CIN -- TD\\
Ch. 7 : Torseurs -- \ifthenelse{\boolean{prof}}{P}{E}}{
\xxtitre}}

\def\xxcathegorie{\ifthenelse{\boolean{xp}}{
2013 -- 2014 \\
Xavier \textsc{Pessoles}}{
Informatique - Cours}}





%---------------------------------------------------------------------------


\begin{document}

\ifthenelse{\boolean{xp}}{
\sloppy
\hyphenpenalty 10000


%------------- En tetes et Pieds de Pages ------------

\pagestyle{fancy}
\renewcommand{\headrulewidth}{0pt}
\fancyhead{}
\fancyhead[L]{%
\noindent\begin{minipage}[c]{2.6cm}%
\includegraphics[width=2cm]{png/logo_ptsi.png}%
\end{minipage}}


\fancyhead[C]{\rule{12cm}{.5pt}}


\fancyhead[R]{%
\noindent\begin{minipage}[c]{3cm}
\begin{flushright}
\footnotesize{\textit{\textsf{\discipline}}}%
\end{flushright}
\end{minipage}
}



\fancyhead[C]{\rule{12cm}{.5pt}}

\renewcommand{\footrulewidth}{0.2pt}

\fancyfoot[C]{\footnotesize{\bfseries \thepage}}
\fancyfoot[L]{%
\begin{minipage}[c]{.2\linewidth}
\noindent\footnotesize{{\xxauteur}}
\end{minipage}
}

\ifthenelse{\boolean{prof}}{%
\fancyfoot[R]{\footnotesize{\xxpied}}}

\begin{center}
 \huge\textsc{\xxtitre}
\end{center}

\begin{center}
 \LARGE\textsc{\xxsoustitre}
\end{center}

\vspace{.5cm}
}{\ifthenelse{\boolean{xp}}{
\usepackage[%
    pdftitle={OS et Environnement de développement},
    pdfauthor={Xavier Pessoles},
    colorlinks=true,
    linkcolor=blue,
    citecolor=magenta]{hyperref}}{
\usepackage[%
    pdftitle={OS et Environnement de développement},
    pdfauthor={Damien Iceta},
    colorlinks=true,
    linkcolor=blue,
    citecolor=magenta]{hyperref}}

\usepackage{pifont}
\usepackage{lastpage}

% \makeatletter \let\ps@plain\ps@empty \makeatother
%% DEBUT DU DOCUMENT
%% =================
\sloppy
\hyphenpenalty 10000

\newcommand{\Pointilles}[1][3]{%
\multido{}{#1}{\makebox[\linewidth]{\dotfill}\\[\parskip]
}}


\colorlet{shadecolor}{orange!15}

\newtheorem{theorem}{Theorem}


\begin{document}


\newboolean{prof}
\setboolean{prof}{true}
%------------- En tetes et Pieds de Pages ------------


\pagestyle{fancy}
%\renewcommand{\headrulewidth}{0}
\renewcommand{\headrulewidth}{0.2pt} %pour mettre le trait en haut

\fancyhead{}
\fancyhead[L]{
\footnotesize{{{\xxtitre}}}%
%\noindent\noindent\begin{minipage}[c]{2.6cm}
%\includegraphics[width=2.5cm]{png/logo.png}%
%\end{minipage}
}

%\fancyhead[C]{\rule{12cm}{.5pt}}  %pour mettre le petit trait en haut


\fancyhead[R]{%
\noindent\begin{minipage}[c]{3cm}
\begin{flushright}
\footnotesize{{{\xxcathegorie}}}%
\end{flushright}
\end{minipage}
}

\renewcommand{\footrulewidth}{0.2pt}

\fancyfoot[C]{\footnotesize{}}
\fancyfoot[L]{%
\begin{minipage}[l]{.2\linewidth}
\noindent\footnotesize{{\xxauteur}}
\end{minipage}
\begin{minipage}[c]{.15\linewidth}
%\includegraphics[width=2cm]{png/logoCC.png}
\end{minipage}}

\ifthenelse{\boolean{prof}}{%
\fancyfoot[R]{\footnotesize{Page \thepage\   sur  \pageref{LastPage}}}}

\begin{center}
 \huge\textsc{\xxtitre}
\end{center}

\begin{center}
 \LARGE\textsc{\xxsoustitre}
\end{center}

\vspace{.5cm}}

\begin{center}
\large{\textsc{Chapitre 7 -- Torseurs}}
\end{center}

\begin{center}
\textsc{Travaux dirigés}
\end{center}

\normalsize

\begin{flushright}
\textit{D'après ressources ???}
\end{flushright}

 \renewcommand{\baselinestretch}{1.2}
%\setlength{\parskip}{2ex plus 0.5ex minus 0.2ex}




\section{Carrousel au triple mouvement}
\setcounter{subparagraph}{0}
%\textbf{Objectif technologique:} 
%\begin{itemize}
%\item Calculer le rapport de réduction d'un train d'engrenages
%\end{itemize}
\ifthenelse{\boolean{prof}}{}{
\begin{center}
 \includegraphics[width=\textwidth]{images/exo1}
\end{center}

Le carrousel étudié est constitué d'un fût 1 supportant un plateau tournant 2 sur lequel sont articulés des disques 3 auxquels sont liées les nacelles 4. 
\subsubsection*{Données}
\begin{itemize}
\item $\mathcal{R}_0 (O,\vect{i_0},\vect{j_0},\vect{k_{01}})$ repère lié au bâti $0$
\item $\mathcal{R}_1 (O,\vect{i_1},\vect{j_1},\vect{k_1})$ et $\mathcal{R}_1^* (O,\vect{i_1^*},\vect{j_1^*},\vect{k_1^*})$ repères liés à $1$
\item $\mathcal{R}_2 (O,\vect{i_2},\vect{j_2},\vect{k_2})$ repère lié à $2$
\item $\mathcal{R}_3 (C,\vect{i_3},\vect{j_3},\vect{k_3})$ repère lié à $3$
\item $\mathcal{R}_4 (C,\vect{i_4},\vect{j_4},\vect{k_4})$ repère lié à $4$
\end{itemize}

$$
\vect{OA}=L\vect{i_2} \quad \vect{AB}=h\vect{k_1^*} \quad \vect{BC}=R\vect{i_3} \quad \vect{GC}=e\vect{k_4} 
$$

$L$, $h$, $R$ et $e$ sont des constantes positives.

\begin{itemize}
\item Liaison 1--0 : pivot d'axe $(O,\vect{k_{01}})$ : $\alpha = (\vect{i_0},\vect{i_1})$
\item Liaison 1--2 : pivot d'axe $(O,\vect{k_{21}^*})$ : $\beta = (\vect{i_1^*},\vect{i_2})$
\item Liaison 3--2 : pivot d'axe $(A,\vect{k_{321}^*})$ : $\gamma = (\vect{i_2},\vect{i_3})$
\item Liaison 4--3 : pivot d'axe $(C,\vect{j_{43}})$ : $\psi = (\vect{k_{321}^*},\vect{k_4})$
\end{itemize}

Inclinaison du plateau 1 : rotation d'axe $(O,\vect{j_1})$, $\theta_1=(\vect{i_1},\vect{i_1^*})$ où $\theta_1$ est une constante positive. La liaison 1--0 n'est pas animée; donc $\alpha=\dot{\alpha}=0$. Un moteur permet d'animer la liaison 2--1 ($\beta\neq0$). 


\begin{center}
 \includegraphics[width=\textwidth]{images/exo2}
\end{center}
}

\subparagraph{}
%\textit{Exprimer $\torseurcin{V}{2}{1}$ au point $A$.}
\textit{Exprimer $\vecto{2}{1}$ et $\vectv{A}{2}{1}$.}
\ifthenelse{\boolean{prof}}{
\begin{corrige}
Les solides $S_2$ et $S_1$ sont en liaison pivot de centre $O$, d'angle $\beta$ et d'axe $\vect{k_{21}}$. En conséquence en $O$, on a :
$$
\left\{
\mathcal{V}(2/1)
\right\}
=
\left\{
\begin{array}{c}
\vect{\Omega(2/1)} = \dot{\beta} \vect{k_{21}} \\
\vect{V(O,2/1)} = \vect{0}
\end{array}
\right\}_O
=
\left\{
\begin{array}{c}
\vect{\Omega(2/1)} = \dot{\beta} \vect{k_{21}} \\
\vect{V(A,2/1)} = \vect{V(O,2/1)} + \vect{AO} \wedge \vect{\Omega(2/1)}
\end{array}
\right\}_A
$$

$$
\vect{V(A,2/1)} = -L\vect{i_2} \wedge  \dot{\beta} \vect{k_{21^*}} = L\dot{\beta}\vect{j_2}
$$

\end{corrige}}{}


\subparagraph{}
%\textit{Exprimer $\torseurcin{V}{3}{1}$ au point $C$.}
\textit{Exprimer $\vecto{3}{1}$ et $\vectv{C}{3}{1}$.}
\ifthenelse{\boolean{prof}}{
\begin{corrige}

Pour calculer la vitesse relative entre $S_3$ et $S_1$, il faut décomposer le torseur cinématique : 
$\left\{
\mathcal{V}(3/1)
\right\} =
\left\{
\mathcal{V}(3/2)
\right\} +
\left\{
\mathcal{V}(2/1)
\right\} 
$

Les solides $S_3$ et $S_2$ sont en liaison pivot de centre $A$, d'angle $\gamma$ et d'axe $\vect{k_{321^*}}$; donc : 
$$
\left\{
\mathcal{V}(3/2)
\right\}
=
\left\{
\begin{array}{c}
\vect{\Omega(3/2)} = \dot{\gamma} \vect{k_{321^*}} \\
\vect{V(A,3/2)} = \vect{0}
\end{array}
\right\}_A
$$

On a donc :
$$
\left\{
\mathcal{V}(3/1)
\right\}
=
\left\{
\begin{array}{c}
\vect{\Omega(3/1)} = \left( \dot{\beta} +\dot{\gamma} \right)\vect{k_{321^*}} \\
\vect{V(A,3/1)} = L\dot{\beta}\vect{j_2}
\end{array}
\right\}_A
$$

En conséquences, 

$$
\vect{V(C,3/1)} = \vect{V(A,3/1)} + \vect{CA} \wedge \vect{\Omega(3/1)} =L\dot{\beta}\vect{j_2} + \left( -R\vect{i_{3}} - h \vect{k_{1^*}}\right) \wedge \left( \dot{\beta} +\dot{\gamma} \right)\vect{k_{321^*}}
$$

$$
\vect{V(C,3/1)} = L\dot{\beta}\vect{j_2} + R \left( \dot{\beta} +\dot{\gamma} \right)\vect{j_3} 
$$
\end{corrige}}{}


\subparagraph{}
%\textit{Exprimer $\torseurcin{V}{4}{1}$ au point $G$.}
\textit{Exprimer $\vectv{G}{4}{1}$.}
\ifthenelse{\boolean{prof}}{
\begin{corrige}En prenant un peu de recul, il n'est pas forcément indispensable d'écrire entièrement le torseur cinématique. Par exemple, dans le cas de $\vect{V(G,4/1)}$ :
$$
\vect{V(G,4/1)} = \vect{V(C,4/1)} + \vect{GC} \wedge \vect{\Omega(4/1)} =
\underbrace{\vect{V(C,4/3)}}_{\vect{0}} + \vect{V(C,3/1)} + \vect{GC} \wedge \underbrace{\vect{\Omega(4/1)}}_{\vect{\Omega(4/3)}+\vect{\Omega(3/1)}}
$$

$$
\vect{V(G,4/1)} =
 L\dot{\beta}\vect{j_2} + R \left( \dot{\beta} +\dot{\gamma} \right)\vect{j_3} 
+ e \vect{k_4} \wedge \left( 
\left( \dot{\beta} +\dot{\gamma} \right)\vect{k_{321^*}}
+\dot{\psi} \vect{j_{43}}
\right)
$$

On a : 
$$ 
\vect{k_4} \wedge \vect{k_{321^*}} = - \sin \psi \vect{j_{43}}  
 $$

$$
\vect{k_4} \wedge \vect{j_{43}} = -\vect{i_{4}}
$$

On a donc :
$$
\vect{V(G,4/1)} =
 L\dot{\beta}\vect{j_2} + R \left( \dot{\beta} +\dot{\gamma} \right)\vect{j_3} 
- e \left( \dot{\beta} +\dot{\gamma} \right)\sin \psi \vect{j_{43}}  
- e \dot{\psi} \vect{i_{4}}
$$

$$
\vect{V(G,4/1)} =
 L\dot{\beta}\vect{j_2} + \left( R -e \sin \psi \right) \left( \dot{\beta} +\dot{\gamma} \right)\vect{j_3} 
- e \dot{\psi} \vect{i_{4}}
$$
\end{corrige}}{}

%\subparagraph{}
%\textit{Exprimer $\vect{V(A,2/1)}$ en fonction de $\dot{\beta}$ et $L$, $\vect{V(C,3/1)}$ en fonction de $R$, $L$, $\dot{\beta}$ et $\dot{\gamma}$ et $\vect{V(G,4/1)}$ en fonction de $R$, $L$, $e$, $\psi$, $\dot{\psi}$, $\dot{\beta}$et $\dot{\gamma}$.}

\vspace{.5cm}

Le fût 1 est muni d'une poulie de diamètre $D$ sur laquelle s'enroule une courroie qui entraîne en rotation la poulie de diamètre $D/2$ liée au disque 3 lors du mouvement de 2 par rapport à 1. 

On a les hypothèses suivantes : 
\begin{itemize}
\item non glissement entre la courroie et les poulies;
\item la courroie est inextensible.
\end{itemize}

De plus le siège 4 est bloqué dans la position $\psi = -\pi/2$ par rapport au disque 3. 

\subparagraph{}
\textit{En utilisant les hypothèses précédentes, montrer que  $\dot{\gamma}=-2\dot{\beta}$.}

\ifthenelse{\boolean{prof}}{
\begin{corrige}
En considérant l'hypothèse de roulement sans glissement au point $I$, le point $I$ est immobile  lorsqu'on considère le mouvement de la courroie (notée c) par rapport à la poule 1 :
$$
\vect{V(I,c/1)} = \vect{0}
$$

En utilisant la décomposition du vecteur vitesse : 
$$
\vect{V(I,c/1)} = \vect{V(I,c/2)} + \vect{V(I,2/1)}
$$

En conséquence, 
$$
\vect{V(I,c/2)}= -\vect{V(I,2/1)}
$$

Par ailleurs, 
$$
\vect{V(I,2/1)} = \vect{V(O,2/1)} + \vect{IO} \wedge \vect{\Omega(2/1)} = 
0 -\dfrac{D}{2}\vect{i_c} \wedge \dot{\beta} \vect{k_{321^*}} = \dfrac{D}{2}\dot{\beta}\vect{j_c}
$$

De même, en considérant l'hypothèse de roulement sans glissement au point $J$, le point $J$ est immobile  lorsqu'on considère le mouvement de la courroie (notée c) par rapport à la poule 3 :
$$
\vect{V(J,c/3)} = \vect{0}
$$


En utilisant la décomposition du vecteur vitesse : 
$$
\vect{V(J,c/3)} = \vect{V(J,c/2)} + \vect{V(J,2/3)}
$$

En conséquence, 
$$
\vect{V(J,c/2)}= -\vect{V(J,2/3)}
$$
Par ailleurs, 
$$
\vect{V(J,2/3)} = \vect{V(A,2/3)} + \vect{JO} \wedge \vect{\Omega(2/3)} = 
0 -\dfrac{D}{4}\vect{i_c} \wedge -\dot{\gamma} \vect{k_{321^*}} = - \dfrac{D}{4}\dot{\gamma}\vect{j_c}
$$


La courroie étant inextensible, 
$$ \vect{V(I,c/2)}= \vect{V(J,c/2)}$$
Et donc : 
$$
\dot{\gamma} = -2 \dot{\beta}
$$

\end{corrige}}{}

\subparagraph{}
\textit{En déduire la nouvelle expression de $\vect{V(G,4/1)}$ en fonction de $R$, $L$, $e$ et $\dot{\beta}$.}
\ifthenelse{\boolean{prof}}{
\begin{corrige}

On a donc $\dot{\psi}=0$ et :

$$
\vect{V(G,4/1)} =
 L\dot{\beta}\vect{j_2} -\dot{\beta} \left( R +e\right) \vect{j_3} 
$$



\end{corrige}}{}

\subparagraph{}
\textit{Exprimer l'accélération du point $G$ dans le mouvement de 4/1 en fonction de $R$, $L$, $e$, $\dot{\beta}$ si $\dot{\beta}$ est constant.}
\ifthenelse{\boolean{prof}}{
\begin{corrige}

$$
\vect{\Gamma(G,4/1)}  =\left[\dfrac{d\vect{V(G,4/1)}}{dt}\right]_{\mathcal{R}_1}
=
L\underbrace{\ddot{\beta}}_{0}\vect{j_2} 
+ L\dot{\beta}\left[\dfrac{d\vect{j_2}}{dt}\right]_{\mathcal{R}_1}
-\underbrace{\ddot{\beta}}_{0} \left( R +e\right) \vect{j_3} 
 -\dot{\beta} \left( R +e\right) \left[\dfrac{d\vect{j_3}}{dt}\right]_{\mathcal{R}_1}
$$

$$
\left[\dfrac{d\vect{j_2}}{dt}\right]_{\mathcal{R}_1}  =
\left[\dfrac{d\vect{j_2}}{dt}\right]_{\mathcal{R}_2} + \vect{\Omega(2/1)}\wedge \vect{j_2}
=\vect{0} + \dot{\beta}\vect{k_{21^*}} \wedge \vect{j_2}
=- \dot{\beta}\vect{i_2}
$$


$$
\left[\dfrac{d\vect{j_3}}{dt}\right]_{\mathcal{R}_1}  =
\left[\dfrac{d\vect{j_3}}{dt}\right]_{\mathcal{R}_3} + \vect{\Omega(3/1)}\wedge \vect{j_3}
=\vect{0} + (\dot{\beta}+\dot{\gamma})\vect{k_{321^*}} \wedge \vect{j_3}
=\dot{\beta} \vect{i_3}
$$

$$
\vect{\Gamma(G,4/1)} 
=
- L\dot{\beta}^2 \vect{i_2}
 -\dot{\beta}^2  \left( R +e\right) \vect{i_3}
$$

\end{corrige}}{}

\subparagraph{}
\textit{Calculer la valeur maximale de la norme de cette accélération pour $\dot{\beta}=2 rad/s$, $L=5m$, $R=1m$, $e=1m$.}
\ifthenelse{\boolean{prof}}{
\begin{corrige}

On a :
$$
||\vect{\Gamma(G,4/1)}||^2
= L^2\dot{\beta}^4 +\dot{\beta}^4  \left( R +e\right)^2
+2 L\dot{\beta}^4 \left( R +e\right) \cos (\vect{i_2},\vect{i_3})
= L^2\dot{\beta}^4 +\dot{\beta}^4  \left( R +e\right)^2
+2 L\dot{\beta}^4 \left( R +e\right) \cos \gamma
$$

$||\vect{\Gamma(G,4/1)}||^2$ est maximal lorsque $\cos \gamma = 1$; donc
$$||\vect{\Gamma(G,4/1)}|| = \sqrt{L^2\dot{\beta}^4 +\dot{\beta}^4  \left( R +e\right)^2
+2 L\dot{\beta}^4 \left( R +e\right) }
= \dot{\beta}^2(L+R+e) = 28\;m\cdot s^{-2}
$$

\end{corrige}}{}

Le dessin ci-dessous montre le mécanisme permettant de faire varier en fonctionnement l'angle $\theta_1$. L'actionneur de ce mécanisme est le vérin hydraulique 5--6. 

Soit $\vect{FH}=2a\vect{i_7}$, $\vect{FE}=3a\vect{i_1}$ (où $a$ est une constante positive); $\vect{EH}=x(t)\vect{i_{56}}$ et $\varphi(t)=(\vect{i_7},\vect{i_1})$.

\subparagraph{}
\textit{Exprimer $x$ en fonction de $a$ et $\varphi$ puis la vitesse de sortie de la tige du vérin, soit $\vect{V(H,6/5)}$, en fonction de $a$, $\varphi$ et $\dot{\varphi}$.}
\ifthenelse{\boolean{prof}}{
\begin{corrige}

Commençons par écrire la fermeture de chaîne cinématique dans le triangle EFH : 
$$ 
\vect{EF}+\vect{FH}+\vect{HE}=\vect{0}
\Longleftrightarrow
x(t)\vect{i_{56}}=2a\vect{i_7}-3a\vect{i_1}
$$
En élevant cette relation au carré, on a : 
$$
x(t)^2 = 4a^2 + 9a^2 - 12a^2\cos\phi
$$

En conséquence, 
$$
x(t)=a\sqrt{13- 12\cos\phi}
$$

Par ailleurs, 
$$
\vect{V(H,6/5)} = \left[\dfrac{d\vect{EH}}{dt}\right]_{\mathcal{R}_5} = 
\dfrac{d(a\sqrt{13- 12\cos\phi})}{dt} \vect{i_{56}}
$$

$$
\vect{V(H,6/5)} = 
\dfrac{6a\dot{\phi}\sin\phi}{\sqrt{13-12\cos\phi}}
\vect{i_{56}}
$$

\end{corrige}}{}


\subparagraph{}
\textit{En considérant que dans cet intervalle de temps, $\dot{\varphi}$ est constante, déterminer le volume d'huile nécessaire au passage de la position $\varphi=\pi/9$ à la position $\varphi=\pi/3$, si $S$ est la section du piston sur laquelle agit l'huile.}
\ifthenelse{\boolean{prof}}{
\begin{corrige}
$$
Vol = S\cdot\left(x\left(\dfrac{\pi}{9}\right) - x\left(\dfrac{\pi}{6}\right)\right) = 0,187\;m^3
$$

\end{corrige}}{}


AN : $a=2m$, $S=700cm^2$.

\begin{center}
 \includegraphics[width=\textwidth]{images/exo3}
\end{center}


\end{document}


